\documentclass[../math.tex]{subfiles}
\externaldocument{../math.tex}

\begin{document}

\chapter{Foundations} \label{chap_foundations}

\section{Calculus of Inductive Constructions}

The basic foundation for math used here is the Calculus of Inductive
Constructions (CIC) used by Coq.  I actually don't know how to describe CIC
rigorously, so in this section I will only describe things in non-precise terms.

The fundamental entity is types.  Types have values, and a value $a$ having a
type $\mathbb U$ is written as $a : \mathbb U$.  There are two main categories
of types: \verb|Prop| and \verb|Set|.  \verb|Prop| is impredicative, while I
don't actually know what's special about \verb|Set|.  Case analysis (described
below) on values of type \verb|Prop| is not allowed when constructing values of
a type other than \verb|Prop|.  Other than types explicitly in \verb|Prop|, most
types will be written in blackboard bold: $\mathbb A$, $\mathbb B$, etc.

Types form a hierarchy: there exist types \verb|Type|$_0$, \verb|Type|$_1$,
\ldots such that $\verb|Type|_n : \verb|Type|_{n + 1}$.  \verb|Prop| and
\verb|Set| are both values in some \verb|Type|$_n$.  While you do not need to
worry about this hierarchy most of the time, every once in a while certain
constructions are prohibited by the hierarchy, because $\verb|Type| :
\verb|Type|$ is inconsistent.

Functions are also first-class objects, and are values in \verb|Type|.  A
function $f : \mathbb A \rightarrow \mathbb B$ produces a unique value $f(a) :
\mathbb B$ for all $a : \mathbb A$.  I think the exact type of $f$ is related to
the types of $A$ and $B$.

The main way of constructing new types is through inductive definitions.  These
produce values/functions called constructors that create values of the new type.
These constructors are allowed to be recursive.  As an example, the natural
numbers are defined as an inductive type with two constructors: A single value
called $0_{\mathbb N}$, and a function $S_{\mathbb N} : \mathbb N \rightarrow
\mathbb N$.  The values from each constructor are considered distinct, as are
the values from a single constructor with different inputs (i.e. constructors
are injective).  To build things from an inductive type, one can do a case
analysis and produce a different value for every possible constructor.  To prove
things from an inductive type, one can do case analysis while further assuming
that the property holds for all inputs to the constructors, which corresponds to
a proof by induction.

Propositions are types.  Values of that type correspond to proofs of that
proposition.  So given a function like $A \rightarrow B$ where $A$ and $B$ are
propositions, it can be thought of as saying that given a proof of $A$, we can
get a proof of $B$, aka $A$ implies $B$.  Predicates $P$ on a type $\mathbb A$
are thought of as functions $\mathbb A \rightarrow \vtt{Prop}$.

Sets are thought of as predicates.  As such, we often use the notation $a \in P$
to mean $P(a)$.  We can also use the usual set builder notation $\{x : \U \mid
S(x)\}$ to describe a set $S$ built from values in $\U$.

Basic logical propositions are defined in Coq and will not be defined in detail
here.  Here is a list of the ones used here:
\begin{itemize}
    \item True, which is a proposition that has a single constructor with no
    inputs.  Notation: \verb|True|.
    \item False, which is a proposition that has no constructors.  Notation:
    \verb|False|.
    \item Not, which takes in a \verb|Prop| $A$ and is defined as $A \rightarrow
    \verb|False|$.  Notation: $\neg$.
    \item And, which is a \verb|Prop| that has a single constructor that takes
    in two other \verb|Prop|s.  Notation: $\wedge$.
    \item Or, which is a \verb|Prop| that has two constructors that each take
    in another \verb|Prop|.  Notation: $\vee$.
    \item Iff, which takes in two \verb|Prop|s $A$ and $B$ and is defined as $(A
    \rightarrow B) \wedge (B \rightarrow A)$.  Notation: $\leftrightarrow$.
    \item Equality, which takes in a type and two values of that type and
    returns a \verb|Prop| saying whether the two values are equal.  Notation:
    $=$.
    \item Existence, which is a \verb|Prop| that takes in a predicate on a type
    and a value of that type that satisfies that predicate.  Notation:
    $\exists$.
\end{itemize}

Beyond these basic mathematical propositions, there are a few more that are
useful for us:

\begin{itemize}
    \item Inhabited: A \verb|Prop| that takes in a value of a certain type.  It
    is basically a proof that a type is not empty.  Notation: \verb|inhabited|.
    \item Strong Existence: Like existence but is a \verb|Type| rather than a
    \verb|Prop|.  The reason for having this is that you can produce the value
    when constructing non-\verb|Prop|s, which you can't do with normal
    existence.  Notation: $\{x \mid P\}$, which states that $x$ satisfies
    predicate $P$.
    \item Strong Or: Like or, but is a \verb|Type| rather than a \verb|Prop|.
    This exists for the same reason strong existence exists.  Notation: $\{A\} +
    \{B\}$.
\end{itemize}

\section{Further Axioms}

The system described in the previous section has a few shortcomings when it
comes to doing math in a traditional way, so we will now extend it with some
axioms.

\begin{axiom}[Functional Extensionality]
    For all types $\mathbb A$ and $\mathbb B$, for all function $f, g : \mathbb
    A \rightarrow \mathbb B$,
    \[
        (\forall x, f(x) = g(x)) \rightarrow f = g.
    \]
\end{axiom}

\begin{axiom}[Propositional Extensionality]
    For all propositions $A$ and $B$,
    \[
        (A \leftrightarrow B) \rightarrow A = B.
    \]
\end{axiom}

\begin{axiom}[Indefinite Description]
    For all types $\mathbb A$,
    \[
        \verb|inhabited | \mathbb A \rightarrow \mathbb A.
    \]
\end{axiom}

Functional extensionality and propositional extensionality are pretty
self-explanatory.  They directly give another kind of extensionality:

\begin{theorem}[Predicate extensionality]
    Given a type $\mathbb A$, for all predicates $P, Q : \mathbb A \rightarrow
    \verb|Prop|$,
    \[
        (\forall x, x \in P \leftrightarrow x \in Q) \rightarrow P = Q.
    \]
\end{theorem}
\begin{proof}
    Given that $P$ and $Q$ are functions, we can use functional extensionality
    and reduce this to showing that for all $x$, $x \in P = x \in Q$.  Using
    propositional extensionality, we only need to show that $x \in P
    \leftrightarrow x \in Q$.  This is true by hypothesis.  Thus, $P = Q$.
\end{proof}

Indefinite description is a bit less clear and requires some explaining.  It is
related to the axiom of choice.  The main use of this axiom is to get around the
restriction of case analysis on \verb|Prop|s to create \verb|Type|s, at the cost
of losing all description of that value (hence the name of the axiom).  As an
example, here are some simple but useful theorems that indefinite description
gives us:

\begin{theorem} \label{ex_to_type}
    Given a type $\mathbb A$ and a predicate $P : \mathbb A \rightarrow
    \verb|Prop|$,
    \[
        (\exists x, x \in P) \rightarrow \{x \mid P\}.
    \]
\end{theorem}
\begin{proof}
    Due to the restriction of case analysis on \verb|Prop|s when creating
    \verb|Type|s, we can't explicitly produce the value $x$ from $\exists x,
    P(x)$.  However, we can apply indefinite description to instead make us
    need to construct $\verb|inhabited | \{x \mid P\}$.  This is now a
    \verb|Prop|, so we can now produce the value $x$ that satisfies $P(x)$.  Now
    that we have the value $x$ such that $P(x)$, we can construct $\{x \mid
    P\}$, so we know that $\verb|inhabited | \{x \mid P\}$.
\end{proof}

\begin{theorem} \label{or_to_strong}
    For all propositions $A$ and $B$,
    \[
        A \vee B \rightarrow \{A\} + \{B\}.
    \]
\end{theorem}
\begin{proof}
    The proof is very similar to the proof of Theorem \ref{ex_to_type} and won't
    be repeated.
\end{proof}

Several other useful facts can be derived from these axioms, but the proofs are
more complicated.  I didn't work out the proofs myself so I will not put them
here.

\begin{theorem}[Proof Irrelevance]
    For a given proposition $P$, for all $A, B : P$, $A = B$.
\end{theorem}

\begin{theorem}[Excluded Middle]
    For all propositions $P$, $P \vee \neg P$.
\end{theorem}
\noindent By Theorem \ref{or_to_strong}, we also have $\{P\} + \{\neg P\}$.

These last several theorems make many of the usual constructions in math work
and will be used implicitly in the future.

\section{Logic}

We now have the basic machinery necessary to prove many basic logical theorems.
Many of these proofs will implicitly be using propositional extensionality.

\begin{theorem} \label{false_rect}
    For all types $\mathbb A$, $\verb|False| \rightarrow \mathbb A$.
\end{theorem}
\begin{proof}
    Assuming we have an instance of $\verb|False|$, we can do case analysis on
    it.  Because $\verb|False|$ has no constructors, there are no cases, so we
    vacuously get a value of $\mathbb A$.
\end{proof}
\noindent This is how we formalize some proofs by contradiction.

\begin{theorem} \label{not_not}
    For all propositions $P$, $P = \neg\neg P$.
\end{theorem}
\begin{proof}
    Assume that $P$ is true.  By definition, $\neg\neg P$ is the same as $(P
    \rightarrow \verb|False|) \rightarrow \verb|False|$, so we can assume that
    $P \rightarrow \verb|False|$.  Because we know that $P$ is true, this means
    that $\verb|False|$ is true, which is a contradiction.

    Now assume that $\neg \neg P$ is true.  By the law of the excluded middle,
    we know that either $P$ or $\neg P$.  If $P$, then we are done, so consider
    the case $\neg P$.  Because $\neg \neg P$ is equivalent to $\neg P
    \rightarrow \verb|False|$, we now have a proof of $\verb|False|$, which is a
    contradiction.
\end{proof}
\noindent The second part of this theorem is how we formalize some other proofs
by contradiction.  We will no longer be this careful in how we deal with proofs
by contradiction and will deal with them in a more traditional way.

\begin{theorem}
    For all propositions $A$ and $B$, we have $(A \to B) \leftrightarrow (\neg B
    \to \neg A)$.
\end{theorem}
\begin{proof}
    First assume that $A \to B$ and $\neg B$, and for a contradiction assume
    that $A$.  Then we have $B$, which is a contradiction.  Thus we have $\neg
    A$, so when $A \to B$, we have $\neg B \to \neg A$.

    To prove that $(\neg B \to \neg A) \to (A \to B)$, notice that by the
    previous theorem this is the same as $(\neg B \to \neg A) \to (\neg \neg A
    \to \neg \neg B)$, which follows from the preceding paragraph.
\end{proof}
\noindent This is how we formalize proofs by contrapositive.

\begin{theorem}
    For all propositions $A$ and $B$, $\neg (A \rightarrow B) = A \wedge \neg
    B$.
\end{theorem}
\begin{proof}
    Assume that $\neg (A \rightarrow B)$.  We must prove both $A$ and $\neg B$.
    If $\neg A$, then $A \rightarrow B$ would be true because any proof of $A$
    would cause a contradiction with $\neg A$.  But $\neg (A \rightarrow B)$, so
    we must have $A$.  Now if $B$, then $A \rightarrow B$ would be true because
    $B$ is always true anyway.  But $\neg (A \rightarrow B)$, so we must have
    $\neg B$.  Thus, both $A$ and $\neg B$, so $\neg (A \rightarrow B)
    \rightarrow A \wedge \neg B$.

    Now assume that $A \wedge \neg B$, and for a contradiction assume that $A
    \rightarrow B$.  Since $A$ and $A \rightarrow B$, we get $B$, but $\neg B$,
    so we have a contradiction.  Thus, $A \wedge \neg B \rightarrow \neg(A
    \rightarrow B)$.
\end{proof}

\begin{theorem} \label{not_and}
    For all propositions $A$ and $B$, $\neg(A \wedge B) = \neg A \vee \neg B$.
\end{theorem}
\begin{proof}
    Assume that $\neg(A \wedge B)$.  If we had both $A$ and $B$, it would be a
    contradiction, so we have at least one of $\neg A$ and $\neg B$, which means
    that $\neg A \vee \neg B$.

    Now assume that $\neg A \vee \neg B$.  For the sake of a contradiction,
    assume that $A \wedge B$.  Then in both cases of $\neg A \vee \neg B$ we
    would have a contradiction, so $\neg(A \wedge B)$.
\end{proof}

\begin{theorem} \label{not_or}
    For all propositions $A$ and $B$, $\neg(A \vee B) = \neg A \wedge \neg B$.
\end{theorem}
\begin{proof}
    We can prove this using the previous theorems:
    \begin{align*}
        \neg(A \vee B) &= \neg A \wedge \neg B \\
        \neg(A \vee B) &= \neg \neg (\neg A \wedge \neg B) \\
        \neg(A \vee B) &= \neg (\neg \neg A \vee \neg \neg B) \\
        \neg(A \vee B) &= \neg (A \vee B).
    \end{align*}
\end{proof}

\begin{theorem}
    For all types $\mathbb U$ and predicates $P$ on $\mathbb U$,
    \[
        \neg(\exists a, a \in P) = (\forall a, a \notin P).
    \]
\end{theorem}
\begin{proof}
    Assume that $\neg(\exists a, a \in P)$.  Now let $a : \mathbb U$ and assume
    for a contradiction that $a \in P$.  Then we have $\exists a, a \in P$,
    which is a contradiction, so we must have $a \notin P$.

    Now assume that $\forall a, a \notin P$, and assume for a contradiction that
    $\exists a, a \in P$.  This is instantly a contradiction, so $\neg(\exists
    a, a \in P)$.
\end{proof}

\begin{theorem}
    For all types $\mathbb U$ and predicates $P$ on $\mathbb U$,
    \[
        \neg(\forall a, a \in P) = (\exists a, a \notin P).
    \]
\end{theorem}
\begin{proof}
    We can prove this using the previous theorems:
    \begin{align*}
        \neg(\forall a, a \in P) &= (\exists a, a \notin P) \\
        \neg(\forall a, a \in P) &= \neg\neg(\exists a, a \notin P) \\
        \neg(\forall a, a \in P) &= \neg(\forall a, \neg \neg (a \in P)) \\
        \neg(\forall a, a \in P) &= \neg(\forall a, a \in P).
    \end{align*}
\end{proof}

\begin{theorem}
    For all propositions $A$ and $B$, $\neg(A \wedge B) = (A \rightarrow \neg
    B)$.
\end{theorem}
\begin{proof}
    \begin{align*}
        \neg(A \wedge B) &= (A \rightarrow \neg B) \\
        \neg(A \wedge B) &= \neg \neg (A \rightarrow \neg B) \\
        \neg(A \wedge B) &= \neg (A \wedge \neg \neg B) \\
        \neg(A \wedge B) &= \neg (A \wedge B).
    \end{align*}
\end{proof}

\begin{theorem} \label{and_assoc}
    For all propositions $A$, $B$, and $C$, $A \wedge (B \wedge C) = (A \wedge
    B) \wedge C$.
\end{theorem}
\begin{proof}
    Trivial.
\end{proof}

\begin{theorem} \label{or_assoc}
    For all propositions $A$, $B$, and $C$, $A \vee (B \vee C) = (A \vee B)
    \wedge C$.
\end{theorem}
\begin{proof}
    Trivial.
\end{proof}

\begin{theorem} \label{and_comm}
    For all propositions $A$ and $B$, $A \wedge B = B \wedge A$.
\end{theorem}
\begin{proof}
    Trivial.
\end{proof}

\begin{theorem} \label{or_comm}
    For all propositions $A$ and $B$, $A \vee B = B \vee A$.
\end{theorem}
\begin{proof}
    Trivial.
\end{proof}

\begin{theorem} \label{and_or_ldist}
    For all propositions $A$, $B$, and $C$, $A \wedge (B \vee C) = (A \wedge B)
    \vee (A \wedge C)$.
\end{theorem}
\begin{proof}
    Assume that $A \wedge (B \vee C)$.  Then $A$ and $B \vee C$.  In the case
    where $B$ is true, we have $A \wedge B$, and in the case where $C$ is true,
    we have $A \wedge C$.  Either way, we have $(A \wedge B) \vee (A \wedge C)$.

    Now assume that $(A \wedge B) \vee (A \wedge C)$.  In the case where $A
    \wedge B$ is true, we have $B \vee C$, so $A \wedge (B \vee C)$ is true.  In
    the case where $A \wedge C$ is true, we have $B \vee C$, so $A \wedge (B
    \vee C)$ is true.  Either way, we have $A \wedge (B \vee C)$.
\end{proof}

\begin{theorem} \label{and_or_rdist}
    For all propositions $A$, $B$, and $C$, $(A \vee B) \wedge C = (A \wedge C)
    \vee (B \wedge C)$.
\end{theorem}
\begin{proof}
    This follows from Theorems \ref{and_comm} and \ref{and_or_ldist}.
\end{proof}

\begin{theorem} \label{or_and_ldist}
    For all propositions $A$, $B$, and $C$, $A \vee (B \wedge C) = (A \vee B)
    \wedge (A \vee C)$.
\end{theorem}
\begin{proof}
    Assume that $A \vee (B \wedge C)$.  In the case where $A$ is true, we have
    both $A \vee B$ and $A \vee C$, so $(A \vee B) \wedge (A \vee C)$ is true.
    In the case where $B \wedge C$ is true, we again have both $A \vee B$ and $A
    \vee C$, so $(A \vee B) \wedge (A \vee C)$ is true.  Either way, we have $(A
    \vee B) \wedge (A \vee C)$.

    Now assume that $(A \vee B) \wedge (A \vee C)$.  If $A$ is true, then $A
    \vee (B \wedge C)$ is true.  If $A$ is not true, then $A \vee B$ implies
    $B$ and $A \vee C$ implies $C$, so we have both $B$ and $C$.  Thus, $B
    \wedge C$, so $A \vee (B \wedge C)$ is true.  Either way, we have $A \vee (B
    \wedge C)$.
\end{proof}

\begin{theorem} \label{or_and_rdist}
    For all propositions $A$, $B$, and $C$, $(A \wedge B) \vee C = (A \vee C)
    \wedge (B \vee C)$.
\end{theorem}
\begin{proof}
    This follows from Theorems \ref{or_comm} and \ref{or_and_ldist}.
\end{proof}

\begin{theorem} \label{or_lfalse}
    For all propositions $P$, $\vtt{False} \vee P = P$.
\end{theorem}
\begin{proof}
    Assume that $\vtt{False} \vee P$.  Then because $\vtt{False}$ can't be true,
    we must have $P$.

    Now assume $P$.  Then $\vtt{False} \vee P$ because $P$ is true.
\end{proof}

\begin{theorem} \label{or_ltrue}
    For all propositions $P$, $\vtt{True} \vee P = \vtt{True}$.
\end{theorem}
\begin{proof}
    The forward implication is true because \vtt{True} is always true.  The
    reverse implication is true because $A \rightarrow A \vee B$.
\end{proof}

\begin{theorem} \label{and_lfalse}
    For all propositions $P$, $\vtt{False} \wedge P = \vtt{False}$.
\end{theorem}
\begin{proof}
    Both implications assume \vtt{False}, so the theorem is true vacuously.
\end{proof}

\begin{theorem} \label{and_ltrue}
    For all propositions $P$, $\vtt{True} \wedge P = P$.
\end{theorem}
\begin{proof}
    Assuming $\vtt{True} \wedge P$, we automatically get $P$, so the forward
    implication is true.  Assuming $P$, we already have \vtt{True} for free so
    we get $\vtt{True} \wedge P$, so the reverse implication is true as well.
\end{proof}

\begin{theorem} \label{or_idemp}
    For all propositions $P$, $P \vee P = P$.
\end{theorem}
\begin{proof}
    If $P \vee P$ is true, then in both cases we have $P$, so the forward
    implication is true.  Now if $P$ is true, then we have $P \vee P$, so the
    reverse implication is true.
\end{proof}

\begin{theorem} \label{and_idemp}
    For all propositions $P$, $P \wedge P = P$.
\end{theorem}
\begin{proof}
    If $P \wedge P$ is true, then we have $P$, so the forward implication is
    true.  Now if $P$ is true, then we have $P$ and $P$, so $P \wedge P$, so the
    reverse implication is true.
\end{proof}

\begin{theorem} \label{or_both}
    For all propositions $P$, $P \vee \neg P = \vtt{True}$.
\end{theorem}
\begin{proof}
    The forward implication is trivial, while the reverse implication follows
    from the law of the excluded middle.
\end{proof}

\begin{theorem} \label{and_both}
    For all propositions $P$, $P \wedge \neg P = \vtt{False}$.
\end{theorem}
\begin{proof}
    For the forward implication $P \wedge \neg P$ is a contradiction.  For the
    reverse implication, \vtt{False} itself is a contradiction.  Thus, the
    theorem is true vacuously.
\end{proof}

We can also prove some theorems about the nature of \verb|Prop| itself.

\begin{theorem} \label{not_true}
    $\neg \verb|True| = \verb|False|.$
\end{theorem}
\begin{proof}
    Assume $\neg \verb|True|$.  However, because we know that \verb|True| is
    true, we have a contradiction.  The reverse implication follows from Theorem
    \ref{false_rect}.
\end{proof}

\begin{theorem} \label{not_false}
    $\neg \verb|False| = \verb|True|.$
\end{theorem}
\begin{proof}
    \begin{align*}
        \neg \vtt{False} &= \vtt{True} \\
        \neg \vtt{False} &= \neg \neg \vtt{True} \\
        \neg \vtt{False} &= \neg \vtt{False}.
    \end{align*}
\end{proof}

\begin{theorem} \label{prop_eq_true}
    For all propositions $P$, $P = (P = \vtt{True})$.
\end{theorem}
\begin{proof}
    Assume $P$.  Then we must prove that $P = \vtt{True}$.  $P \rightarrow
    \vtt{True}$ is trivial because \vtt{True} is always true.  $\vtt{True}
    \rightarrow P$ is true because we are assuming $P$.  Thus, $P \rightarrow (P
    = \vtt{True})$.

    Now assume that $P = \vtt{True}$.  Then to prove $P$ we just need to prove
    $\vtt{True}$, which is trivial.  Thus, $(P = \vtt{True}) \rightarrow P$.
\end{proof}

\begin{theorem} \label{prop_eq_false}
    For all propositions $P$, $\neg P = (P = \vtt{False})$.
\end{theorem}
\begin{proof}
    Assume $\neg P$.  Then we must prove that $P = \vtt{False}$.  $P \rightarrow
    \vtt{False}$ is precisely the statement $\neg P$.  $\vtt{False} \rightarrow
    P$ is true vacuously.  Thus, $\neg P \rightarrow (P = \vtt{False})$.

    Now assume that $P = \vtt{False}$.  Then $\neg P$ is equivalent to $\neg
    \vtt{False}$, which is equal to $\vtt{True}$, which is trivial.
\end{proof}

\begin{theorem} \label{prop_split}
    For all propositions $P$, $(P = \vtt{True}) \vee (P = \vtt{False})$.
\end{theorem}
\begin{proof}
    By theorems \ref{prop_eq_true} and \ref{prop_eq_false}, this is equivalent
    to $P \vee \neg P$, which is just the law of the excluded middle.
\end{proof}

\begin{theorem} \label{prop_neq}
    $\vtt{True} \neq \vtt{False}$.
\end{theorem}
\begin{proof}
    This is equivalent to $\vtt{True} = \vtt{False} \rightarrow \vtt{False}$.
    So assuming that $\vtt{True} = \vtt{False}$, we must prove \vtt{False}.  But
    because $\vtt{True} = \vtt{False}$, this is the same as proving \vtt{True},
    which is trivial.
\end{proof}

\begin{theorem}
    For all propositions $P$, $P \neq (\neg P)$.
\end{theorem}
\begin{proof}
    By Theorem \ref{prop_split}, $P$ is either \vtt{True} or \vtt{False}.  When
    $P$ is \vtt{True}, we then get $(\vtt{True} \neq (\neg \vtt{True})) =
    (\vtt{True} \neq \vtt{False})$, which is Theorem \ref{prop_neq}.  When $P$
    is \vtt{False}, we then get $(\vtt{False} \neq (\neg \vtt{False})) =
    (\vtt{False} \neq \vtt{True})$, which is again Theorem \ref{prop_neq}.
\end{proof}
\noindent The last few theorems will be used later to prove that the cardinality
of \vtt{Prop} is two.

\begin{theorem} \label{not_eq_eq}
    For all propositions $A$ and $B$, if $(\neg A) = (\neg B)$, then $A = B$.
\end{theorem}
\begin{proof}
    We can apply $\neg$ to both sides of $(\neg A) = (\neg B)$ to get $(\neg
    \neg A) = (\neg \neg B)$, so by Theorem \ref{not_not} we have $A = B$.
\end{proof}

\begin{theorem} \label{not_eq_iff}
    For all propositions $A$ and $B$, $(A \leftrightarrow B) \leftrightarrow
    (\neg A \leftrightarrow \neg B)$.
\end{theorem}
\begin{proof}
    By propositional extensionality, we can reduce this to proving that $(A = B)
    \leftrightarrow ((\neg A) = (\neg B))$.  The forward direction is trivial,
    and the reverse direction is the previous theorem.
\end{proof}

\begin{theorem} \label{prop_is_true}
    For all propositions $P$, $P \to P = \verb|True|$.
\end{theorem}
\begin{proof}
    This is a direct consequence of Theorem \ref{prop_eq_true}.
\end{proof}

\begin{theorem} \label{prop_is_false}
    For all propositions $P$, $\neg P \to P = \verb|False|$.
\end{theorem}
\begin{proof}
    This is a direct consequence of Theorem \ref{prop_eq_false}.
\end{proof}

\begin{theorem} \label{or_left}
    For all propositions $A$ and $B$, $(\neg B \to A) \to A \vee B$.
\end{theorem}
\begin{proof}
    If $B$, then we have $A \vee B$.  If $\neg B$, then we have $A$ from $\neg B
    \to A$, so $A \vee B$.
\end{proof}

\begin{theorem} \label{or_right}
    For all propositions $A$ and $B$, $(\neg A \to B) \to A \vee B$.
\end{theorem}
\begin{proof}
    If $A$, then we have $A \vee B$.  If $\neg A$, then we have $B$ from $\neg A
    \to B$, so $A \vee B$.
\end{proof}

\section{Functions}

\begin{definition}
    Given a type $\mathbb U$, we define the identity function $1_\mathbb U :
    \mathbb U \rightarrow \mathbb U$ as $1_\mathbb U(a) = a$ for all $a :
    \mathbb U$.
\end{definition}

\begin{definition}
    Given types $\mathbb A$ and $\mathbb B$ and a proof that $\mathbb A
    \rightarrow \vtt{False}$, we define the empty function $\varnothing_{\mathbb
    A \mathbb B} : \mathbb A \rightarrow \mathbb B$ by utilizing Theorem
    \ref{false_rect} on all inputs.
\end{definition}

\begin{definition}
    Given three types $\A$, $\B$, and $\C$ and two functions $f : \A \rightarrow
    \B$ and $g : \B \rightarrow \C$, we define their composition $g \circ f$
    with $(g \circ f)(a) = g(f(a))$ for all $a : \A$.
\end{definition}

\begin{class}
    Given a function $f : \A \rightarrow \B$, we say that the function is
    injective if $\forall a, b : \A, f(a) = f(b) \rightarrow a = b$.
\end{class}

\begin{class}
    Given a function $f : \A \rightarrow \B$, we say that the function is
    surjective if $\forall b : \B, \exists a : \A, f(a) = b$.
\end{class}

\begin{class}
    Given a function $f : \A \rightarrow \B$, we say that the function is
    bijective if it is injective and surjective.
\end{class}

\begin{instance}
    The identity function is bijective.
\end{instance}
\begin{proof}
    Because $1_\U(a) = 1_\U(b)$ is identical to $a = b$, $1_\U(a) = 1_\U(b)
    \rightarrow a = b$, so $1_\U$ is injective.  For any $x : \U$, $1_\U(x) =
    x$, so $1_\U$ is surjective.  Because $1_\U$ is injective and surjective, it
    is bijective.
\end{proof}

\begin{instance} \label{inj_comp}
    The composition of injective functions is injective.
\end{instance}
\begin{proof}
    Let $\A$, $\B$, and $\C$ be types and $f : \A \rightarrow \B$ and $g : \B
    \rightarrow \C$ be injective functions.  Let $a, b : \A$ such that $g(f(a))
    = g(f(b))$.  Because $g$ is injective, we have $f(a) = f(b)$, and because
    $f$ is injective, we have $a = b$ as required.
\end{proof}

\begin{instance} \label{sur_comp}
    The composition of surjective functions is surjective.
\end{instance}
\begin{proof}
    Let $\A$, $\B$, and $\C$ be types and $f : \A \rightarrow \B$ and $g : \B
    \rightarrow \C$ be surjective functions.  Let $c : \C$.  Then because $g$ is
    surjective, there exists a $b : \B$ such that $g(b) = c$.  Furthermore,
    because $f$ is surjective, there exists an $a : \A$ such that $f(a) = b$.
    Thus, $g(f(a)) = g(b) = c$, so $g \circ f$ is surjective.
\end{proof}

\begin{instance} \label{bij_comp}
    The composition of bijective functions is bijective.
\end{instance}
\begin{proof}
    This is just the combination of Theorems \ref{inj_comp} and \ref{sur_comp}.
\end{proof}

\begin{instance} \label{empty_inj}
    The empty function is injective.
\end{instance}
\begin{proof}
    Let $\A$ and $\B$ be types such that $\A \rightarrow \vtt{False}$.  Then for
    all $a, b : \A$ such that $\varnothing_{\A\B}(a) = \varnothing_{\A\B}(b)$,
    we already have a contradiction because no such values exist in the first
    place.  Thus, the theorem is vacuously true.
\end{proof}

\begin{instance}
    For any types $\A$ and $\B$ and function $f : \A \rightarrow \B$, if $\B
    \rightarrow \vtt{False}$, then $f$ is surjective.
\end{instance}
\begin{proof}
    Let $b : \B$.  Then because $\B \rightarrow \vtt{False}$, we already have a
    contradiction, so the theorem is vacuously true.
\end{proof}

\begin{theorem}[Partition Principle]
    For all types $\A$ and $\B$, if there is a surjective function from $\A$ to
    $\B$, there exists an injective function from $\B$ to $\A$.
\end{theorem}
\begin{proof}
    Let $f$ be a surjective function from $\A$ to $\B$.  Then for any value $b :
    \B$, we have a value $a : \A$ such that $f(a) = b$.  Define a new function
    $g : \B \rightarrow \A$ that is given by this value.  Now let $x, y : \B$
    such that $g(x) = g(y)$.  Because $f$ is surjective, we have an $a$ such
    that $f(a) = x$, and by the definition of $g$ we have $g(x) = a$.
    Similarly, there is a $b$ such that $f(b) = y$ and $g(y) = b$.  But since
    $g(x) = g(y)$, we have $a = b$.  Thus, $f(a) = f(b)$, so $x = y$, showing
    that $g$ is an injective function from $\B$ to $\A$.
\end{proof}

The previous theorem implicitly used indefinite description (in particular,
Theorem \ref{ex_to_type}) in its definition of $g$.  These uses of indefinite
description will no longer be pointed out in the future.

\begin{definition}
    Given types $\A$ and $\B$ and functions $f : \A \rightarrow \B$ and $g : \B
    \rightarrow \A$, $g$ is said to be the inverse of $f$ if
    \[
        \forall x : \A, g(f(x)) = x
    \]
    and
    \[
        \forall x : \B, f(g(x)) = x.
    \]
\end{definition}

\begin{theorem}
    Bijective functions have an inverse.
\end{theorem}
\begin{proof}
    Let $f$ be a bijective function from $\A$ to $\B$.  Because $f$ is
    surjective, given any $b : \B$, there exists an $a : \A$ such that $f(a) =
    b$.  Let $g : \B \rightarrow \A$ be the function that gets this value.  It
    is claimed that $g$ is the inverse of $f$.  First, let $x : \A$.  Then
    $g(f(x)) = y$ with $f(y) = f(x)$.  Because $f$ is injective, we have $y =
    x$, so $g(f(x)) = x$.  Second, let $x : \B$.  Then $g(x) = y$ with $f(y) =
    x$.  Thus, $f(g(x)) = f(y) = x$.
\end{proof}

\begin{instance} \label{inverse_ex_bijective}
    A function that has an inverse is bijective.
\end{instance}
\begin{proof}
    Let $f : \A \rightarrow \B$ and $g : \B \rightarrow \A$ be inverses.  First,
    let $a$ and $b$ be values in $\A$ such that $f(a) = f(b)$.  Then we also
    have $g(f(a)) = g(f(b))$, and because $f$ and $g$ are inverses, we get $a =
    b$.  Thus, $f$ is injective.  Now let $b : \B$.  Then letting $a = g(b)$,
    $f(a) = b$, so $f$ is surjective.
\end{proof}

\section{Basic Types} \label{sec_types}

There are a few basic types that we can create that are useful.

\begin{definition}
    Given a \Prop\ $A$ and a dependent \Prop\ $B : A \rightarrow \Prop$, we
    define the dependent conjunction $A \curlywedge B$ with a single
    constructor $\forall a : A, B(a) \rightarrow A \curlywedge B$.
\end{definition}

This is like a normal conjunction except that the second \Prop depends on the
first \Prop being true.

\begin{definition}
    Define an inductive type $\S$ with a single value constructor $I$.  This is
    called the singleton type.
\end{definition}

\begin{definition}
    Given two types $\A$ and $\B$, define an inductive type with a single
    constructor that takes in a value of each type.  Call this type the product
    of $\A$ and $\B$, denoted $\A \times \B$.  We write values of this type as
    $(a, b)$, where $a : \A$ and $b : \B$.  We also can define the projection
    operators $P_1 : \A \times \B \rightarrow \A$ and $P_2 : \A \times \B
    \rightarrow \B$ that acquire the first and second component of a value in
    $\A \times \B$.
\end{definition}

\begin{definition}
    Given two type $\A$ and $\B$, define an inductive type with two
    constructors: one which takes in a value of $\A$, and another which takes in
    a value of $\B$.  Call this type the sum of $\A$ and $\B$, denoted $\A +
    \B$.
\end{definition}

\begin{definition}
    Given a type $\A$, define the optional type with two constructors: one
    called \vtt{val} which takes in a value of $\A$, and one value constructor
    called \verb|nil|.  Call this type $\O(\A)$.
\end{definition}

\begin{class}
    Given a type $\A$, we say that the type is nontrivial if there are two
    values $a$ and $b$ in $\A$ with $a \neq b$.
\end{class}

\begin{theorem} \label{not_trivial2}
    If $\A$ is a nontrivial type, then for all $c : \A$, there exists a $d : \A$
    such that $c \neq d$.
\end{theorem}
\begin{proof}
    Because $\A$ is not trivial, we have some $a$ and $b$ in $\A$ such that $a
    \neq b$.  Now let $c : \A$.  If $c = a$, then $b$ is an element such that $c
    \neq b$.  If $c \neq a$, then $a$ is an element such that $c \neq a$.
\end{proof}

\begin{class}
    Given a type $\A$, we say that the type is a singleton type if there exists
    an $x : \A$ such that for all $a : \A$, $a = x$.
\end{class}

\begin{theorem}
    If $\A$ is a singleton type, then for all $a$ and $b$ in $\A$, $a = b$.
\end{theorem}
\begin{proof}
    Because both $a = x$ and $b = x$, we have $a = b$.
\end{proof}

\begin{theorem}
    If any type $\A$ is inhabited and if for all $a$ and $b$ in $\A$ we have $a
    = b$, then $\A$ is a singleton type.
\end{theorem}
\begin{proof}
    Because $\A$ is inhabited there is some $x : \A$.  Because $a = b$ for all
    $a$ and $b$ in $\A$, we know that $a = x$ for all $a : \A$, so $\A$ is a
    singleton type.
\end{proof}

\end{document}
