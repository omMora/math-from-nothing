\documentclass[../math.tex]{subfiles}
\externaldocument{../math.tex}
\externaldocument{set}

\begin{document}

\setcounter{chapter}{4}

\chapter{Lists}

\begin{definition}
    Let $\A$ be a type.  Then we can define a new inductive type $\L(\A)$ with
    two constructors: a value constructor $[]$, and a constructor that takes in
    a value $x : \A$ and a value $a : \L(\A)$, denoted $x : a$.  We call a
    value in $\L(\A)$ a list of $\A$, and we call $[]$ the empty list.  We use
    the notation $[a, b, \cdots, n]$ to mean $a : b : \cdots : n : []$.
\end{definition}

By the properties of inductive types, we can perform induction on lists.  Note
that we are not performing induction on the size of the list.  We are performing
induction on the list itself.  This will be an important distinction to make
when we are talking about the size of the list.

Note that by the properties of constructors, we have $x : a \neq []$.  We also
know that if $x_1 : a_1 = x_2 : a_2$, then $x_1 = x_2$ and $a_1 = a_2$.

Throughout this chapter, let $\U$ be a type.  Unless said otherwise, all values
that are not lists will be assumed to be of this type.

\section{Basic Operations}

\begin{instance}
    Let $a$ and $b$ be two lists.  We define their sum $a + b$ recursively on
    $a$:
    \begin{align*}
              [] + b &= b \\
        (x : a') + b &= x : (a' + b).
    \end{align*}
\end{instance}
\noindent Notice that addition is concatenation of lists.

\begin{definition}
    Let $a$ be a list.  Then we define its reverse $a^{-1}$ recursively:
    \begin{align*}
              []^{-1} &= [] \\
        (x : a')^{-1} &= a'^{-1} + [x].
    \end{align*}
\end{definition}

\begin{definition}
    Let $\A$ and $\B$ be types, and $f$ a function from $\A$ to $\B$.  Then for
    a list $a : \L(\A)$, we can define a new list $f(a) : \L(\B)$ recursively:
    \begin{align*}
            f([]) &= [] \\
        f(x : a') &= f(x) : f(a').
    \end{align*}
\end{definition}

\begin{instance}
    Define $0 = []$.  Note that we will still use the notation $[]$ instead of
    $0$.
\end{instance}

\begin{instance}
    $[]$ is a left additive identity.
\end{instance}
\begin{proof}
    This follows directly from the definition of $+$.
\end{proof}

\begin{theorem}
    For all values $x$ and lists $a$ and $b$,
    \[
        (x : a) + b = x : (a + b).
    \]
\end{theorem}
\begin{proof}
    This follows directly from the definition of $+$.
\end{proof}

\begin{theorem}
    For all values $x$ and lists $a$,
    \[
        [x] + a = x : a.
    \]
\end{theorem}
\begin{proof}
    \[
        [x] + a = (x : []) + a = x : ([] + a) = x : a.
    \]
\end{proof}

\begin{instance}
    $[]$ is a right additive identity.
\end{instance}
\begin{proof}
    The proof will be by induction on $a$.  When $a = []$, then $[] + [] = []$.
    Now assume that $a + [] = a$.  We must prove that for all $x$, $(x : a) + []
    = x : a$.  This is true because
    \[
        (x : a) + [] = x : (a + []) = x : a,
    \]
    where the last equality follows from the inductive hypothesis.
\end{proof}

\begin{instance}
    Concatenation of lists is associative.
\end{instance}
\begin{proof}
    Let $a$, $b$, and $c$ be lists.  We will prove that $a + (b + c) =
    (a + b) + c$ by induction on $a$.  First, when $a = []$, then
    \[
        [] + (b + c) = b + c = ([] + b) + c,
    \]
    so the base case is true.  Now assume that $a + (b + c) = (a +
    b) + c$, and let $x$ be any value.  Then
    \begin{align*}
        &(x : a) + (b + c) \\
        ={}& x : (a + (b + c) \\
        ={}& x : ((a + b) + c) \\
        ={}& (x : (a + b)) + c \\
        ={}& ((x : a) + b) + c,
    \end{align*}
    showing that the inductive case is true.  Thus, by induction, concatenation
    of lists is associative.
\end{proof}

\begin{instance}
    Concatenation of lists is left cancellative.
\end{instance}
\begin{proof}
    Let $a$, $b$, and $c$ be lists such that $c + a = c + b$.  The proof will be
    by induction on $c$.  When $c = []$, we have
    \[
        a = [] + a = [] + b = b,
    \]
    so the base case is true.  Now assume that if $c + a = c + b$, then $a = b$.
    Let $x$ be any value.  Then
    \begin{align*}
        (x : c) + a &= (x : c) + b \\
        x : (c + a) &= x : (c + b) \\
        c + a &= c + b,
    \end{align*}
    so the result follows from the inductive hypothesis.  Thus, by induction,
    concatenation of lists is left cancellative.
\end{proof}

\begin{theorem}
    $[]^{-1} = []$.
\end{theorem}
\begin{proof}
    This follows directly from the definition of the reverse of a list.
\end{proof}

\begin{theorem}
    For all values $x$ and lists $a$, $(x : a)^{-1} = a^{-1} + [x]$.
\end{theorem}
\begin{proof}
    This follows directly from the definition of the reverse of a list.
\end{proof}

\begin{theorem}
    For all values $x$, $[x]^{-1} = [x]$.
\end{theorem}
\begin{proof}
    \[
        [x]^{-1} = (x : [])^{-1} = [] + [x] = [x].
    \]
\end{proof}

\begin{theorem}
    For all lists $a$ and $b$, $(a + b)^{-1} = b^{-1} + a^{-1}$.
\end{theorem}
\begin{proof}
    The proof will be by induction on $a$.  When $a = []$, we have
    \[
        ([] + b)^{-1} = b^{-1} = b^{-1} + [],
    \]
    so the base case is true.  Now assume that $(a + b)^{-1} = b^{-1} +
    a^{-1}$.  Then for all $x : \U$,
    \begin{align*}
           & ((x : a) + b)^{-1} \\
        ={}& (x : (a + b))^{-1} \\
        ={}& (a + b)^{-1} + [x] \\
        ={}& (b^{-1} + a^{-1}) + [x] \\
        ={}& b^{-1} + (a^{-1} + [x]) \\
        ={}& b^{-1} + (x : a)^{-1},
    \end{align*}
    so the inductive case is true.  Thus, by induction, $(a + b)^{-1} =
    b^{-1} + a^{-1}$ for all lists $a$ and $b$.
\end{proof}

\begin{theorem} \label{list-reverse-reverse}
    For all lists $a$, $(a^{-1})^{-1} = a$.
\end{theorem}
\begin{proof}
    The proof will be by induction on $a$.  When $a = []$, we have
    $([]^{-1})^{-1} = []^{-1} = []$, so the base case is true.  Now assume that
    $(a^{-1})^{-1} = a$ and let $x$ be any value of $\U$.  Then
    \begin{align*}
           & ((x : a)^{-1})^{-1} \\
        ={}& (a^{-1} + [x])^{-1} \\
        ={}& [x]^{-1} + (a^{-1})^{-1} \\
        ={}& [x] + a \\
        ={}& x : a,
    \end{align*}
    se the inductive case is true.  Thus, by induction $(a^{-1})^{-1} = a$ for
    all lists $a$.
\end{proof}

\begin{theorem}
    For all lists $a$ and $b$, $a = b$ if and only if $a^{-1} = b^{-1}$.
\end{theorem}
\begin{proof}
    The forward direction is trivial.  For the reverse direction, assume that
    $a^{-1} = b^{-1}$.  Then we can apply the reverse to both sides to get
    $(a^{-1})^{-1} = (b^{-1})^{-1}$.  By the previous theorem this means that $a
    = b$, as required.
\end{proof}

\begin{instance}
    Concatenation of lists is right cancellative.
\end{instance}
\begin{proof}
    Let $a$, $b$, and $c$ be lists such that $a + c = b + c$.  Then
    \begin{align*}
        a + c &= b + c \\
        (a + c)^{-1} &= (b + c)^{-1} \\
        c^{-1} + a^{-1} &= c^{-1} + b^{-1} \\
        a^{-1} &= b^{-1} \\
        a &= b.
    \end{align*}
\end{proof}

\begin{theorem}
    For all lists $a$, if $a^{-1} = []$, then $a = []$ as well.
\end{theorem}
\begin{proof}
    Because $[]^{-1} = []$, we have $a^{-1} = []^{-1}$, and the result follows
    from the previous theorem.
\end{proof}

\begin{theorem}
    $f([]) = []$.
\end{theorem}
\begin{proof}
    This follows directly from the definition.
\end{proof}

\begin{theorem}
    For all $a : \L(\A)$ and $x : \A$, $f(x : a) = f(x) : f(a)$.
\end{theorem}
\begin{proof}
    This follows directly from the definition.
\end{proof}

\begin{theorem}
    For all $x : \A$, $f([x]) = [f(x)]$.
\end{theorem}
\begin{proof}
    \[
        f([x]) = f(x : []) = f(x) : [] = f(x).
    \]
\end{proof}

\begin{theorem}
    For all lists $a$ and $b$, $f(a + b) = f(a) + f(b)$.
\end{theorem}
\begin{proof}
    The proof will be by induction on $a$.  When $a = []$,
    \[
        f([] + b) = f(b) = [] + f(b) = f([]) + f(b),
    \]
    so the base case is true.  Now assume that $f(a + b) = f(a) + f(b)$.
    Then for all $x : \A$,
    \begin{align*}
           & f((x : a) + b) \\
        ={}& f(x : (a + b)) \\
        ={}& f(x) : f(a + b) \\
        ={}& f(x) : f(a) + f(b) \\
        ={}& (f(x) : f(a)) + f(b) \\
        ={}& f(x : a) + f(b),
    \end{align*}
    so the inductive hypothesis is true.  Thus, $f(a + b) = f(a) + f(b)$
    by induction.
\end{proof}

\begin{theorem}
    Let $\A$, $\B$, and $\C$ be types, and $f$ a function from $\A$ to $\B$ and
    $g$ a function from $\B$ to $\C$.  Then for all lists $a : \L(\A)$,
    \[
        g(f(a)) = (g \circ f)(a).
    \]
\end{theorem}
\begin{proof}
    The proof will be by induction on $a$.  When $a = []$,
    \[
        g(f([])) = g([]) = [] = (g \circ f)([]),
    \]
    so the base case is true.  Now assume that $g(f(a)) = (g \circ f)(a)$.  Then
    for all $x : \A$,
    \begin{align*}
           & g(f(x : a)) \\
        ={}& g(f(x) : f(a)) \\
        ={}& g(f(x)) : g(f(a)) \\
        ={}& (g \circ f)(x) : (g \circ f)(a) \\
        ={}& (g \circ f)(x : a),
    \end{align*}
    so the inductive hypothesis is true.  Thus, $g(f(a)) = (g \circ f)(a)$ by
    induction.
\end{proof}

\section{Values in Lists}

\begin{definition}
    Let $a$ be a list and $x$ be a value.  Then we define the $x \in a$
    recursively on $a$:
    \begin{align*}
              x \in [] &= \vtt{False} \\
        x \in (y : a') &= (x = y \vee x \in a').
    \end{align*}
\end{definition}

\begin{definition}
    Let $a$ be a list.  Then we define $a$ having unique elements recursively:
    \begin{align*}
            \text{$[]$ has unique elements} &= \vtt{True} \\
        \text{$x : a'$ has unique elements} &= x \notin a' \wedge \text{$a'$ has
        unique elements.}
    \end{align*}
\end{definition}
\noindent Notice that because $(a : l)$ having unique elements implies that $l$
has unique elements, when doing induction on a list, it having unique elements
does not need to be considered as a part of the inductive hypothesis.

\begin{theorem}
    For all elements $x$, $x \notin []$.
\end{theorem}
\begin{proof}
    Because $\neg \vtt{False} = \vtt{True}$, this is true by definition.
\end{proof}

\begin{theorem}
    For all elements $x$ and $y$ and lists $a$, $x \in (y : a) \leftrightarrow x
    = y \vee x \in a$.
\end{theorem}
\begin{proof}
    This is true by definition.
\end{proof}

\begin{theorem}
    For all elements $x$ and $y$, $x \in [y] \leftrightarrow x = y$.
\end{theorem}
\begin{proof}
    \[
        x \in [y] \leftrightarrow x = y \vee x \in []
        \leftrightarrow x = y \vee \vtt{False}
        \leftrightarrow x = y.
    \]
\end{proof}

\begin{theorem} \label{in-list-conc}
    For all lists $a$ and $b$ and for all values $x$, if $x \in (a + b)$, then
    either $x \in a$ or $x \in b$.
\end{theorem}
\begin{proof}
    Assume that $x \notin b$.  Then the proof will be by induction on $a$.  When
    $a = []$, then we have $x \in b$, contradicting $x \notin b$, so the base
    case is true.  Now assume that if $x \in (a + b)$, then $x \in a$, and
    let $y$ be a value such that $x \in (y : a + b)$.  We must prove that $x \in
    (y : a)$.  If $x = y$, then we have $x \in (y : a)$.  If $x \neq y$, then we
    have $x \in (a + b)$, so by the inductive hypothesis we have $x \in a$.
    Thus, $x \in (y : a)$.  Either way, $x \in (y : a)$, showing that the
    theorem is true by induction.
\end{proof}

\begin{theorem} \label{in-list-rconc}
    For all lists $a$ and $b$ and values $x$, if $x \in b$, then $x \in (a +
    b)$.
\end{theorem}
\begin{proof}
    The proof will be by induction on $a$.  When $a = []$, then $x \in b$ by
    hypothesis, so the base case is true.  Now assume that $x \in (a + b)$.  We
    must prove that $x \in (y : a + b)$ for any value $y$.  But $x \in (a + b)$
    is precisely one of the ways that we can have $x \in (y : a + b)$, so the
    result is trivial.  Thus, the theorem is true by induction.
\end{proof}

\begin{theorem} \label{in-list-lconc}
    For all lists $a$ and $b$ and values $x$, if $x \in a$, then $x \in (a +
    b)$.
\end{theorem}
\begin{proof}
    The proof will be by induction on $a$.  When $a = []$, we $x \in []$, a
    contradiction.  Thus the base case is true vacuously.  Now assume that if $x
    \in a$, then $x \in (a + b)$.  We must prove that for some value $y$, if $x
    \in (y : a)$, then $x \in (y : a + b)$.  Now if $x = y$, then we have $x \in
    (y : a + b)$.  If $x \neq y$, then we have $x \in a$, so by the inductive
    hypothesis we have $x \in (a + b)$, showing that $x \in (y : a + b)$.  Thus,
    the theorem is true by induction.
\end{proof}

\begin{theorem} \label{in-list-comm}
    For all lists $a$ and $b$ and values $x$, if $x \in (a + b)$, then $x \in (b
    + a)$.
\end{theorem}
\begin{proof}
    By Theorem \ref{in-list-conc}, we have either $x \in a$ or $x \in b$.  If $x
    \in a$, the result follows from Theorem \ref{in-list-lconc}, and if $x \in
    b$, the result follows from Theorem \ref{in-list-rconc}.
\end{proof}

\begin{theorem} \label{in-list-split}
    For all lists $a$ and values $x$, in $x \in a$, then there exist lists $b$
    and $c$ such that $a = b + x : c$.
\end{theorem}
\begin{proof}
    The proof will be by induction on $a$.  $x \in []$ is a contradiction, so
    the base case is vacuously true.  Now assume that if $x \in a$, then there
    exist lists $b$ and $c$ such that $a = b + x : c$, and assume that $x \in
    (y : a)$ for some $y$.  We must prove that there exist lists $b$ and $c$
    such that $y : a = b + x : c$.  If $x = y$, then we have $y : a = [] + x :
    a$ as required.  If $x \neq y$, then $x \in a$, so by the inductive
    hypothesis we have lists $b$ and $c$ such that $a = b + x : c$, so $y : a
    = y : b + x : c$.  Either way, the inductive step is true, so the theorem is
    true by induction.
\end{proof}

\begin{theorem} \label{in-list-image}
    For all lists $a$, values $x$, and functions $f : \U \to \V$, if $x \in a$,
    then $f(x) \in f(a)$.
\end{theorem}
\begin{proof}
    The proof will be by induction on $a$.  $x \in []$ is a contradiction, so
    the base case is vacuously true.  Now assume that if $x \in a$, then $f(x)
    \in f(a)$, and that $x \in y : a$ for some value $y$.  We must prove that
    $f(x) \in f(y : a) = f(y) : f(a)$.  If $x = y$, then $f(x) \in f(y) : f(a)$.
    If $x \neq y$, then $x \in a$, so by the inductive hypothesis, $f(x) \in
    f(a)$, so $f(x) \in f(y) : f(a)$.  Either way, the inductive step is true,
    so the theorem is true by induction.
\end{proof}

\begin{theorem} \label{image-in-list}
    For all lists $a$, values $y$, and functions $f : \U \to \V$, if $y \in
    f(a)$, then there exists an $x \in a$ such that $f(x) = y$.
\end{theorem}
\begin{proof}
    The proof will be by induction on $a$.  When $a = []$, $y \in f([]) = []$ is
    a contradiction, so the base case is vacuously true.  Now assume that if $y
    \in f(a)$, then there exists an $x \in a$ such that $f(x) = y$, and assume
    that there is some value $z$ such that $y \in f(z : a)$.  This means that
    either $f(z) = y$ or $y \in f(a)$.  If $f(z) = y$, then $z$ itself is an
    element $x$ such that $f(x) = y$ and $x \in (z : a)$.  If $y \in f(a)$, then
    by the inductive hypothesis, there is some $x$ such that $f(x) = y$ and $x
    \in l$, so $x \in (z : a)$ as well.  Either way, the inductive step is true,
    so the theorem is true by induction.
\end{proof}

\begin{theorem}
    $[]$ has unique elements.
\end{theorem}
\begin{proof}
    This is true by definition.
\end{proof}

\begin{theorem}
    For all values $a$ and lists $l$, $a : l$ has unique elements iff $a \notin
    l$ and $l$ has unique elements.
\end{theorem}
\begin{proof}
    This is true by definition.
\end{proof}

\begin{theorem}
    For all values $a$, $[a]$ has unique elements.
\end{theorem}
\begin{proof}
    $[a]$ having unique elements is the same as $a \notin []$ and $[]$ having
    unique elements.  Both of these are trivial.
\end{proof}

\begin{lemma} \label{list-unique-comm-add}
    For all lists $a$ and values $x$, if $x : a$ has unique elements, then $a +
    [x]$ has unique elements.
\end{lemma}
\begin{proof}
    Because $x : a$ has unique elements, $x \notin a$ and $a$ has unique
    elements.  The proof will be by induction on $a$.  When $a = []$, then $[] +
    [x] = [x]$, which has unique elements, so the base case is true.  Now assume
    that if $x \notin a$, then $(a + [x])$ has unique elements, and assume that
    there is a $y$ such that $y \notin (x : a)$.  This means that $y \neq x$ and
    $y \notin a$.  We must prove that $y : a + [x]$ has unique elements, that
    is, that $y \notin (a + [x])$ and that $(a + [x])$ has unique elements.  The
    latter follows from the inductive hypothesis.  To prove that $y \notin (a +
    [x])$, assume that $y \in (a + [x])$.  Then by Theorem \ref{in-list-conc},
    either $y \in a$ or $y \in [x]$.  If $y \in a$, that contradicts $y \notin
    a$ from before.  If $y \in [x]$, then $y = x$ contradicting $y \neq x$ from
    before.  Both cases have a contradiction, so our assumption that $y \in (a +
    [x])$ was false, meaning that $y \notin (a + [x])$.  Thus, the inductive
    step is true, so by induction, the theorem is true.
\end{proof}

\begin{theorem} \label{list-unique-comm}
    For all lists $a$ and $b$, $a + b$ has unique elements if and only if $b +
    a$ has unique elements.
\end{theorem}
\begin{proof}
    By symmetry we only need to prove the forward implication.  We will use
    induction on $a$ to prove the statement ``For all lists $b$, if $a + b$ has
    unique elements, then $b + a$ has unique elements.''  First, when $a = []$,
    then when $[] + b = b$ has unique elements, then $b + [] = b$ has unique
    elements, so the base case is true.

    Now assume that for all lists $b$, if $a + b$ has unique elements, then $b +
    a$ has unique elements.  We need to prove that if $x : a + b$ has unique
    elements, then $b + x : a$ has unique elements.  By Lemma
    \ref{list-unique-comm-add}, we know that $a + b + [x] = a + (b + [x])$ has
    unique elements.  Then by the inductive hypothesis, $(b + [x]) + a = b + x :
    a$ has unique elements.  Thus, by induction, the theorem is true.
\end{proof}

\begin{theorem} \label{list-unique-lconc}
    For all lists $a$ and $b$, if $a + b$ has unique elements, then $a$ has
    unique elements.
\end{theorem}
\begin{proof}
    The proof will be by induction on $a$.  $[]$ has unique elements, so the
    base case is true.  Now assume that $a$ has unique elements and let $x$ be
    such that $x : a + b$ has unique elements.  Because $x : a + b$ has unique
    elements, we know that $x \notin a + b$.  By the contrapositive of Theorem
    \ref{in-list-lconc}, we know that $x \notin a$.  Thus, $x \notin a$ and $a$
    has unique elements, showing that $x : a$ has unique elements.  Thus, the
    theorem is true by induction.
\end{proof}

\begin{theorem} \label{list-unique-rconc}
    For all lists $a$ and $b$, if $a + b$ has unique elements, then $b$ has
    unique elements.
\end{theorem}
\begin{proof}
    This is just the combination of Theorems \ref{list-unique-lconc} and
    \ref{list-unique-comm}.
\end{proof}

\begin{theorem} \label{list-unique-conc}
    For all lists $a$ and $b$, if $a + b$ has unique elements, then for every $x
    \in a$, we have $x \notin b$.
\end{theorem}
\begin{proof}
    Assume that there is an $x$ such that $x \in a$ and $x \in b$.  We will
    derive a contradiction by induction on $a$.  $x \in []$ is a contradiction,
    so the base case is true.  Now assume that $x \notin a$, and that there's a
    $y$ such that $x \in y : a$ and $y : a + b$ has unique elements.  Because $y
    : a + b$ has unique elements, we have $y \notin (a + b)$.  Because $x \in y
    : a$, we have two cases: when $x = y$, and when $x \in a$.  When $x = y$, we
    have $x \in b$, and by Theorem \ref{in-list-rconc} we have $x \in (a + b)$,
    contradicting $x \notin (a + b)$.  When $x \in a$, this directly
    contradicts the inductive hypothesis.  Thus, by induction, we have a
    contradiction, meaning that we can't have both $x \in a$ and $x \in b$.
\end{proof}

\begin{theorem} \label{list-image-unique}
    For all lists $a$ and functions $f : \U \to \V$, if $f(a)$ has unique
    elements, then $a$ does as well.
\end{theorem}
\begin{proof}
    The proof will be by induction on $a$.  When $a = []$, it has unique
    elements, so the base case is true.  Now assume that $a$ has unique
    elements.  We must prove that if $f(x : a) = f(x) : f(a)$ has unique
    elements, then $x : a$ has unique elements.  Because $f(x) : f(a)$ has
    unique elements, we know that $f(x) \notin f(a)$ and that $f(a)$ has unique
    elements.  Because $f(x) \notin f(a)$, by the contrapositive of Theorem
    \ref{in-list-image}, we know that $x \notin a$.  By the inductive
    hypothesis, $a$ has unique elements.  Thus, $x : a$ has unique elements, so
    the theorem is true by induction.
\end{proof}

\section{Lists and Sets}

\begin{definition}
    Let $a$ be a list, and let $S$ be a set.  We will define the filtered list
    $a^S$ recursively:
    \begin{align*}
             []^S &= [] \\
        (x : a)^S &= \text{If $x \in S$ then $x : a^S$ else $a^S$}.
    \end{align*}
\end{definition}

\begin{definition} \label{list-prop}
    Let $a$ be a list, and let $S$ be a set.  We will define $a \in S$, meaning
    that every element in $a$ is in $S$, recursively:
    \begin{align*}
           [] \in S &= \vtt{True} \\
        x : a \in S &= x \in S \wedge a \in S.
    \end{align*}
\end{definition}
\noindent Notice that like a list having unique elements, because $(a : l) \in
S$ implies that $l \in S$, when doing induction on a list, it being in $S$ does
not need to be considered as a part of the inductive hypothesis.

\begin{definition}
    Given a relation $\usim$, let $x \usim$ represent the set of all elements
    $y$ such that $x \sim y$.  Then for a list $a$, we define $a \in \usim$
    recursively:
    \begin{align*}
           [] \in \usim &= \vtt{True} \\
        x : a \in \usim &= a \in (x \usim) \wedge a \in \usim.
    \end{align*}
    Note that in the last expression, the first $\in$ is being used in the sense
    of definition \ref{list-prop}, while the second $\in$ is the recursive one.
\end{definition}

Note that this definition only checks that elements are related in one
direction, and that elements are not checked against themselves.  To be honest
part of this reason is that this definition is used in a grand total of one
situations (at least currently), in a place where the relation is symmetric and
shouldn't be checked against itself.  However, you can derive other forms that
may be more suitable for other situations from the last two definitions, so not
much is lost by using this definition.

It is important to note that we now have four different usages of $\in$.
Context can distinguish between them all.  Given a value $x$, a list $a$, a set
$S$, and a relation $\usim$, $x \in S$ is traditional set membership, $x \in a$
checks if $x$ occurs in the list $a$, $a \in S$ checks if every elements of $a$
is in $S$, and $a \in \usim$ checks if all elements of $a$ are related to each
other.  This confusion won't be as bad in future chapters because these concepts
will usually just be written in English rather than using this notation (e.g.
explicitly saying that every value in a list satisfies a predicate).

\begin{theorem}
    For all sets $S$, $[]^S = []$.
\end{theorem}
\begin{proof}
    This is true by definition.
\end{proof}

\begin{theorem}
    For all sets $S$, values $x$, and lists $a$, if $x \in S$, then $(x : a)^S =
    x : a^S$.
\end{theorem}
\begin{proof}
    This is true by definition.
\end{proof}

\begin{theorem}
    For all sets $S$, values $x$, and lists $a$, if $x \notin S$, then $(x :
    a)^S = a^S$.
\end{theorem}
\begin{proof}
    This is true by definition.
\end{proof}

\begin{theorem}
    For all sets $S$ and values $x$, if $x \in S$, then $[x]^S = [x]$.
\end{theorem}
\begin{proof}
    \[
        [x]^S = x : [] = [x].
    \]
\end{proof}

\begin{theorem}
    For all sets $S$ and values $x$, if $x \in S$, then $[x]^S = []$.
\end{theorem}
\begin{proof}
    Trivial.
\end{proof}

\begin{theorem}
    For all sets $S$ and lists $a$ and $b$, $(a + b)^S = a^S + b^S$.
\end{theorem}
\begin{proof}
    The proof will be by induction on $a$.  When $a = []$,
    \[
        ([] + b)^S = b^S = [] + b^S = []^S + b^S,
    \]
    so the base case is true.  Now assume that $(a + b)^S = a^S + b^S$.  We must
    prove that for any $x$, $(x : a + b)^S = (x : a)^S + b^S$.  There are two
    cases: when $x \in S$, and when $x \notin S$.  When $x \in S$,
    \[
        (x : a + b)^S = x : (a + b)^S = x : a^S + b^S = (x : a)^S + b^S.
    \]
    When $x \notin S$,
    \[
        (x : a + b)^S = (a + b)^S = a^S + b^S = (x : a)^S + b^S.
    \]
    Either way, the inductive step is true, so the theorem is true by induction.
\end{proof}

\begin{lemma} \label{list-filter-in-both}
    For all sets $S$, lists $a$, and values $x$, if $x \in a^S$, then $x \in a$
    and $x \in S$.
\end{lemma}
\begin{proof}
    The proof will be by induction on $a$.  When $a = []$, $x \in []^S = []$ is
    a contradiction, so the base case is vacuously true.  Now assume that for
    any $x \in a^S$, we have $x \in a$ and $x \in S$, and that $x \in (y : a)^S$ for some
    value $y$.  We must prove that $x \in y : a$ and that $x \in S$.  We have two cases: when $y
    \in S$, and when $y \notin S$.  When $y \notin S$, we have $x \in a^S$, so
    by the inductive hypothesis we have $x \in S$ and $x \in a$, meaning that $x
    \in (y : a)$.  When $y \in S$, we have $x \in y : a^S$, so we have two more
    cases here: when $x = y$, and when $x \in a^S$.  When $x = y$, we easily
    have $x \in y : a$ and $y \in S$.  When $x \in a^S$, by the inductive
    hypothesis we have $x \in S$ and $x \in a$, so $x \in y : a$.  The inductive
    step is true in all cases, so the theorem is true by induction.
\end{proof}

\begin{theorem} \label{list-filter-in}
    For all sets $S$, lists $a$, and values $x$, if $x \in a^S$, then $x \in a$.
\end{theorem}
\begin{proof}
    This is just a special case of Lemma \ref{list-filter-in-both}.
\end{proof}

\begin{theorem} \label{list-filter-in-set}
    For all sets $S$, lists $a$, and values $x$, if $x \in a^S$, then $x \in S$.
\end{theorem}
\begin{proof}
    This is just a special case of Lemma \ref{list-filter-in-both}.
\end{proof}

\begin{theorem} \label{list-filter-unique}
    For all sets $S$ and lists $a$, if $a$ has unique elements, then $a^S$ has
    unique elements.
\end{theorem}
\begin{proof}
    The proof will be by induction on $a$.  When $a = []$, then $[]^S = []$ has
    unique elements, so the base case is true.  Now assume that $a^S$ has unique
    elements and that $x : a$ has unique elements for some value $x$.  We must
    prove that $(x : a)^S$ has unique elements.  Because $x : a$ has unique
    elements, we know that $x \notin a$ and that $a$ has unique elements.  There
    are two cases: when $x \in S$, and when $x \notin S$.  When $x \in S$, we
    must prove that $x : a^S$ has unique elements.  $x \notin a^S$ follows from
    the contrapositive of Theorem \ref{list-filter-in}, and $a^S$ having unique
    elements is the inductive hypothesis.  When $x \notin S$, we must prove that
    $a^S$ has unique elements, which is the inductive hypothesis.  Either way,
    $(x : a)^S$ has unique elements, so the theorem is true by induction.
\end{proof}

\begin{theorem} \label{list-filter-image-in}
    For all sets $S : \set{\U}$, functions $f : \U \to \V$, lists $a$, and
    values $y : \V$, if $y \in f(a^S)$, then $y \in f(a)$.
\end{theorem}
\begin{proof}
    By Theorem \ref{image-in-list}, there is some $x \in a^S$ such that $f(x) =
    y$.  Because $x \in a^S$, we know that $x \in a$ by Theorem
    \ref{in-list-image}.  Thus, by Theorem \ref{list-filter-in}, $f(x) = y \in
    f(a)$.
\end{proof}

\begin{theorem} \label{list-filter-image-unique}
    For all sets $S : \set{\U}$, function $f : \U \to \V$, and lists $a$, if
    $f(a)$ has unique elements, then $f(a^S)$ has unique elements.
\end{theorem}
\begin{proof}
    The proof will be by induction on $a$.  When $a = []$, then $f([]^S) = f([])
    = []$ has unique elements, so the base case is true.  Now assume that
    $f(a^S)$ has unique elements, and let $x$ be a value such that $f(x : a)$
    has unique elements.  This means that $f(x) \notin f(a)$.  We must prove
    that $f((x : a)^S)$ has unique elements.  There are two cases: when $x \in
    S$, and when $x \notin S$.  When $x \in S$, we have $f((x : a)^S) = f(x :
    a^S) = f(x) : f(a^S)$.  $f(a^S)$ has unique elements by the inductive
    hypothesis, so we just need to prove that $f(x) \notin f(a^S)$, which
    follows from the contrapositive of Theorem \ref{list-filter-image-in}.  When
    $x \notin S$, we have $f((x : a)^S) = f(a^S)$, which has unique elements by
    the inductive hypothesis.  In both cases, $f((x : a)^S)$ has unique
    elements, so the theorem is true by induction.
\end{proof}

\begin{theorem} \label{list-filter-inter}
    For all sets $S$ and $T$ and lists $a$, $(a^S)^T = a^{S \cap T}$.
\end{theorem}
\begin{proof}
    The proof will be by induction on $a$.  When $a = []$, then $([]^S)^T = []^T
    = [] = []^{S \cap T}$, so the base case is true.  Now assume that $(a^S)^T =
    a^{S \cap T}$.  We must prove that for all values $x$, we have $((x :
    a)^S)^T = (x : a)^{S \cap T}$.  There are three cases: when $x \in S$ and $x
    \in T$, when $x \in S$ and $x \notin T$, and when $x \notin S$.  When $x \in
    S$ and $x \in T$, we also have $x \in S \cap T$, so
    \[
        ((x : a)^S)^T = (x : a^S)^T = x : (a^S)^T = x : a^{S \cap T}
        = (x : a)^{S \cap T}.
    \]
    When $x \in S$ and $x \notin T$, we also have $x \notin S \cap T$, so
    \[
        ((x : a)^S)^T = (x : a^S)^T = (a^S)^T = a^{S \cap T}
        = (x : a)^{S \cap T}.
    \]
    When $x \notin S$, we also have $x \notin S \cap T$, so
    \[
        ((x : a)^S)^T = (a^S)^T = a^{S \cap T} = (x : a)^{S \cap T}.
    \]
    All cases work, so the theorem is true by induction.
\end{proof}

\begin{theorem} \label{list-filter-filter}
    For all sets $S$ and lists $a$, $(a^S)^S = a^S$.
\end{theorem}
\begin{proof}
    This is just the combination of Theorem \ref{list-filter-inter} and Theorem
    \ref{inter-idemp}.
\end{proof}

\begin{theorem}
    For all sets $S$, $[] \in S$.
\end{theorem}
\begin{proof}
    This follows directly from the definition.
\end{proof}

\begin{theorem}
    For all sets $S$, values $x$, and lists $a$, $(x : a) \in S$ if and only if
    $x \in S$ and $a \in S$.
\end{theorem}
\begin{proof}
    This follows directly from the definition.
\end{proof}

\begin{theorem}
    For all sets $S$ and values $x$, $[x] \in S$ if and only if $x \in S$.
\end{theorem}
\begin{proof}
    \[
        [x] \in S \leftrightarrow x \in S \wedge [] \in S
        \leftrightarrow x \in S \wedge \vtt{True}
        \leftrightarrow x \in S.
    \]
\end{proof}

\begin{theorem} \label{list-prop-conc}
    For all sets $S$ and lists $a$ and $b$, $(a + b) \in S$ if and only if $a
    \in S$ and $b \in S$.
\end{theorem}
\begin{proof}
    The proof will be by induction on $a$.  When $a = []$, then
    \begin{align*}
        &([] + b) \in S \\
        \leftrightarrow{}&b \in S \\
        \leftrightarrow{}&\vtt{True} \wedge b \in S \\
        \leftrightarrow{}&[] \in S \wedge b \in S,
    \end{align*}
    so the base case is true.  Now assume that $(a + b) \in S$ if and only if $a
    \in S$ and $b \in S$.  Then for any value $x$,
    \begin{align*}
        &(x : a + b) \in S \\
        \leftrightarrow{}&x \in S \wedge (a + b) \in S \\
        \leftrightarrow{}&x \in S \wedge a \in S \wedge b \in S \\
        \leftrightarrow{}&(x : a) \in S \wedge b \in S,
    \end{align*}
    showing that the inductive step is true.  Thus, the theorem is true by
    induction.
\end{proof}

\begin{theorem} \label{list-prop-sub}
    For all sets $S$ and $T$ with $S \subseteq T$ and all lists $a$, if $a \in
    S$, then $a \in T$.
\end{theorem}
\begin{proof}
    The proof will be by induction on $a$.  When $a = []$, we trivially have $[]
    \in T$, so the base case is true.  Now assume that $a \in T$.  We must prove
    that for all values $x$ with $(x : a) \in S$, we also have $(x : a) \in T$.
    Because $x \in S$ and $S \subseteq T$, we have $x \in T$, and we have $a \in
    T$ by the inductive hypothesis.  Thus, $(x : a) \in T$.
\end{proof}

\begin{theorem} \label{list-prop-filter}
    For all lists $a$ and sets $S$, $a^S \in S$.
\end{theorem}
\begin{proof}
    The proof will be by induction on $a$.  When $a = []$, $[]^S = [] \in S$, so
    the base case is true.  Now assume that $a^S \in S$.  We must prove that for
    all values $x$, $(x : a)^S \in S$.  There are two cases: when $x \in S$, and
    when $x \notin S$.  When $x \in S$, we have $(x : a)^S = x : a^S$, and
    because $x \in S$ and $a^S \in S$ by the inductive hypothesis, we have $(x :
    a)^S \in S$.  When $x \notin S$, we have $(x : a)^S = a^S$, so by the
    inductive hypothesis we have $(x : a)^S \in S$.  Either way, $(x : a)^S \in
    S$, so the theorem is true by induction.
\end{proof}

\begin{theorem} \label{list-prop-in}
    For all lists $a$ and sets $S$, if $a \in S$, then for all $x \in a$, we
    have $x \in S$.
\end{theorem}
\begin{proof}
    The proof will be by induction on $a$.  $x \in []$ is impossible, so the
    base case is vacuously true.  Now assume that if $x \in a$, then $x \in S$.
    We must prove that for all values $y$ such that $(y : a) \in S$ and $x \in
    (y : a)$, we have $x \in S$.  Because $x \in (y : a)$, we have either $x =
    y$ or $x \in a$.  If $x = y$, then because $(y : a) \in S$, we have $x \in
    S$.  If $x \in a$, then $x \in S$ by the inductive hypothesis.  Either way,
    $x \in S$, so the theorem is true by induction.
\end{proof}

\begin{theorem} \label{list-prop-in-sub}
    For all lists $a$ and $b$ and sets $S$, if for all $x \in a$ we have $x \in
    b$, and if $b \in S$, then $a \in S$.
\end{theorem}
\begin{proof}
    The proof will be by induction on $a$.  First $[] \in S$, so the base case
    is true.  Now assume that if for all $x \in a$ we have $x \in b$, then we
    have $a \in S$.  We must prove that for some value $y$ with $x \in (y : a)$
    implying $x \in b$, we have $(y : a) \in S$.  Because $y \in (y : a)$, we
    have $y \in b$, and because $b \in S$, by Theorem \ref{list-prop-in} we have
    $y \in S$.  Also, any $x \in a$ is $(y : a)$ as well, so we have that any $x
    \in a$ is in $b$.  Thus, by the inductive hypothesis, we have $a \in S$.
    Thus, $y \in S$ and $a \in S$, showing that $(y : a) \in S$, showing that
    the theorem is true by induction.
\end{proof}

\begin{theorem} \label{list-prop-other-filter}
    For all lists $a$ and sets $S$ and $T$, if $a \in S$, then $a^T \in S$.
\end{theorem}
\begin{proof}
    By Theorem \ref{list-prop-in-sub}, it suffices to prove that for all $x \in
    a^T$, we have $x \in a$, which follows directly from Theorem
    \ref{list-filter-in}.
\end{proof}

\begin{theorem}
    For all relations $\usim$, $[] \in \usim$.
\end{theorem}
\begin{proof}
    This follows directly from the definition.
\end{proof}

\begin{theorem}
    For all relations $\usim$, lists $a$, and values $x$, $(x : a) \in \usim$ if
    and only if $a \in (x\usim)$ and $a \in \usim$.
\end{theorem}
\begin{proof}
    This follows directly from the definition.
\end{proof}

\begin{theorem} \label{list-prop2-single}
    For all relations $\usim$ and values $x$, $[x] \in \usim$.
\end{theorem}
\begin{proof}
    \[
        [x] \in \usim
        \leftrightarrow [] \in (x\usim) \wedge [] \in \usim
        \leftrightarrow \vtt{True} \wedge \vtt{True}
        \leftrightarrow \vtt{True}.
    \]
\end{proof}

\section{Lists and Natural Numbers}

\section{Folds}

\section{Permutations}

\section{Unordered Lists}

\end{document}
