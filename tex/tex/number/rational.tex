\documentclass[../../math.tex]{subfiles}
\externaldocument{../../math.tex}
\externaldocument{../basics/elementary_algebra}
\externaldocument{../basics/set}
\externaldocument{../basics/category}
\externaldocument{integer}

\begin{document}

\setcounter{chapter}{7}

\chapter{The Rationals}

The rationals can be constructed from the integers by considering pairs of
integers $(a, b)$, thought of as the fraction $\frac{a}{b}$.  However, this
construction is more general than just for constructing the rationals from the
integers and applies to any integral domain.  Thus, in this chapter, most of the
construction will be using an arbitrary integral domain, and the rationals and
their unique properties won't be explored until the end.

\section{Basic Construction}

Let $\U$ be an integral domain, and let $\U^*$ be the type of all nonzero values
in $\U$.

\begin{definition}
    Define a relation $\sim$ on $\U \times \U^*$ where $(a_1, a_2) \sim (b_1,
    b_2)$ is defined to mean $a_1b_2 = b_1a_2$.
\end{definition}

\begin{lemma}
    The relation $\sim$ is an equivalence relation.
\end{lemma}
\begin{proof}
    \textit{Reflexivitity.}  We must check $a_1a_2 = a_1a_2$, which is a
    reflexive equality.

    \textit{Symmetry.}  We must prove that if $a_1b_2 = b_1a_2$, then $b_1a_2 =
    a_1b_2$.  This is true by the symmetry of equality.

    \textit{Transitivity.}  We must prove that if $a_1b_2 = b_1a_2$ and
    $b_1c_2 = c_1b_2$, then $a_1c_2 = c_1a_2$.
    \begin{align*}
        b_1a_2c_2 &= b_1c_2a_2 \\
        a_1b_2c_2 &= c_1b_2a_2 \\
        a_1c_2 &= c_1a_2,
    \end{align*}
    where the cancelling is valid because $b_2 \neq 0$ by definition.
\end{proof}

\begin{definition}
    Define the type $\Frac(\U)$ to be the type $(\U \times \U^*)/\sim$.
\end{definition}

We will use the notation $\frac{a}{b}$ to refer to $[(a, b)]$.  Once division is
defined we will see that this doesn't conflict with the usual fraction notation.

\begin{definition}
    Define the inclusion $\iota_\U : \U \to \frac(\U)$ that takes an $x : \U$
    and bring it to $\frac{x}{1}$.
\end{definition}

\begin{instance}
    $\iota_\U$ is injective.
\end{instance}
\begin{proof}
    If $\iota_\U(a) = \iota_\U(b)$, this means that $a1 = b1$, meaning that $a =
    b$.
\end{proof}

\begin{instance}
    $\Frac(\U)$ is not trivial.
\end{instance}
\begin{proof}
    Because $0 \neq 1$ in $\U$, by $\iota_\U$ being injective we have
    $\iota_\U(0) \neq \iota_\U(1)$.
\end{proof}

\section{Addition}

\begin{definition}
    Define an operation $\oplus : \U \times \U^* \to \U \times \U^* \to \U
    \times \U^*$ given by
    \[
        (a_1, a_2) \oplus (b_1, b_2) = (a_1b_2 + b_1a_2, a_2b_2).
    \]
    $a_2b_2$ is nonzero by Theorem \ref{mult_nz}.
\end{definition}

\begin{lemma}
    The operation $\oplus$ is well-defined under the equivalence relation
    $\sim$.
\end{lemma}
\begin{proof}
    We must prove that for all $a$, $b$, $c$, and $d$, if $a \sim b$ and $c \sim
    d$, we have $a \oplus c \sim b \oplus d$.  Thus, we have
    \[
        a_1b_2 = b_1a_2
    \]
    and
    \[
        c_1d_2 = d_1c_2
    \]
    and must prove that
    \[
        (a_1c_2 + c_1a_2)b_2d_2 = (b_1d_2 + d_1b_2)a_2c_2.
    \]
    Multiplying the first equation by $c_2d_2$ we get
    \[
        a_1c_2b_2d_2 = b_1d_2a_2c_2,
    \]
    and multiplying the second equation by $a_2b_2$ we get
    \[
        c_1a_2b_2d_2 = d_1b_2a_2c_2.
    \]
    Adding these two equations gives us
    \begin{align*}
        a_1c_2b_2d_2 + c_1a_2b_2d_2 &= b_1d_2a_2c_2 + d_1b_2a_2c_2 \\
        (a_1c_2 + c_1a_2)b_2d_2 &= (b_1d_2 + d_1b_2)a_2c_2
    \end{align*}
    as required.
\end{proof}

\begin{instance}
    Define addition in $\Frac(\U)$ as the binary operation given by Theorem
    \ref{binary_op_ex} and the previous lemma.
\end{instance}

\begin{instance}
    Addition in $\Frac(\U)$ is commutative.
\end{instance}
\begin{proof}
    \[
        a + b
        = \frac{a_1}{a_2} + \frac{b_1}{b_2}
        = \frac{a_1b_2 + b_1a_2}{a_2b_2}
        = \frac{b_1a_2 + a_1b_2}{b_2a_2}
        = \frac{b_1}{b_2} + \frac{a_1}{a_2}
        = b + a.
    \]
\end{proof}

\begin{instance}
    Addition in $\Frac(\U)$ is associative.
\end{instance}
\begin{proof}
    \begin{align*}
        a + (b + c)
        &= \frac{a_1}{a_2} + \left( \frac{b_1}{b_2} + \frac{c_1}{c_2} \right) \\
        &= \frac{a_1}{a_2} + \frac{b_1c_2 + c_1b_2}{b_2c_2} \\
        &= \frac{a_1(b_2c_2) + (b_1c_2 + c_1b_2)a_2}{a_2b_2c_2} \\
        &= \frac{a_1b_2c_2 + a_2b_1c_2 + a_2b_2c_1}{a_2b_2c_2} \\
        &= \frac{(a_1b_2 + b_1a_2)c_2 + c_1(a_2b_2)}{a_2b_2c_2} \\
        &= \frac{a_1b_2 + b_1a_2}{a_2b_2} + \frac{c_1}{c_2} \\
        &= \left( \frac{a_1}{a_2} + \frac{b_1}{b_2} \right) \frac{c_1}{c_2} \\
        &= (a + b) + c.
    \end{align*}
\end{proof}

\begin{instance}
    Define zero in $\Frac(\U)$ as $\iota_\U(0)$.
\end{instance}

\begin{instance}
    Zero is an additive identity in $\Frac(\U)$.
\end{instance}
\begin{proof}
    \[
        0 + a =
        \frac{0}{1} + \frac{a_1}{a_2} =
        \frac{0a_2 + a_11}{1a_2} =
        \frac{a_1}{a_2} =
        = a.
    \]
\end{proof}

\begin{definition}
    Define an operation $\ominus : \U \times \U^* \to \U \times \U^*$ given by
    \[
        \ominus (a_1, a_2) = (-a_1, a_2).
    \]
\end{definition}

\begin{lemma}
    The operation $\ominus$ is well-defined.
\end{lemma}
\begin{proof}
    We must prove that $a_1b_2 = b_1a_2$ implies $-a_1b_2 = -b_1a_2$, which can
    be done just by finding the negative of both sides.
\end{proof}

\begin{instance}
    Define additive inverses in $\Frac(\U)$ as the operation given by Theorem
    \ref{unary_op_ex} and the previous lemma.
\end{instance}

\begin{instance}
    Negation is a left inverse in $\Frac(\U)$.
\end{instance}
\begin{proof}
    For all $a$,
    \[
        a - a =
        \frac{a_1}{a_2} + \frac{-a_1}{a_2} =
        \frac{a_1a_2 - a_1a_2}{a_2a_2} =
        \frac{0}{a_2a_2} =
        \frac{0}{1} =
        0.
    \]
\end{proof}

\begin{instance}
    $\iota_\U$ is additive.
\end{instance}
\begin{proof}
    \[
        \iota_\U(a + b) =
        \frac{a + b}{1} =
        \frac{a1 + b1}{(1)(1)} =
        \frac{a}{1} + \frac{b}{1} =
        \iota_\U(a) + \iota_\U(b).
    \]
\end{proof}

\section{Multiplication}

\begin{definition}
    Define an operation $\otimes : \U \times \U^* \to \U \times \U^* \to \U
    \times \U^*$ given by
    \[
        (a_1, a_2) \otimes (b_1, b_2) = (a_1b_1, a_2b_2).
    \]
    $a_2b_2$ is nonzero by Theorem \ref{mult_nz}.
\end{definition}

\begin{lemma}
    The operation $\otimes$ is well-defined under the equivalence relation
    $\sim$.
\end{lemma}
\begin{proof}
    We must prove that for all $a$, $b$, $c$, and $d$, if $a \sim b$ and $c \sim
    d$, we have $a \otimes c \sim b \otimes d$.  Thus, we have
    \[
        a_1b_2 = b_1a_2
    \]
    and
    \[
        c_1d_2 = d_1c_2
    \]
    and must prove that
    \[
        a_1c_1b_2d_2 = b_1d_1a_2c_2.
    \]
    This follows directly by multiplying the previous two equations.
\end{proof}

\begin{instance}
    Define multiplication in $\Frac(\U)$ as the binary operation given by
    Theorem \ref{binary_op_ex} and the previous lemma.
\end{instance}

\begin{instance}
    Multiplication in $\Frac(\U)$ is commutative.
\end{instance}
\begin{proof}
    \[
        ab
        = \frac{a_1}{a_2} \, \frac{b_1}{b_2}
        = \frac{a_1b_1}{a_2b_2}
        = \frac{b_1a_1}{b_2a_2}
        = \frac{b_1}{b_2} \, \frac{a_1}{a_2}
        = ba.
    \]
\end{proof}

\begin{instance}
    Multiplication in $\Frac(\U)$ is associative.
\end{instance}
\begin{proof}
    \[
        a(bc)
        = \frac{a_1}{a_2} \left( \frac{b_1}{b_2} \, \frac{c_1}{c_2} \right)
        = \frac{a_1}{a_2} \, \frac{b_1c_1}{b_2c_2}
        = \frac{a_1b_1c_1}{a_2b_2c_2}
        = \frac{a_1b_1}{a_2b_2} \, \frac{c_1}{c_2}
        = \left( \frac{a_1}{a_2} \, \frac{b_1}{b_2} \right) \frac{c_1}{c_2}
        = (ab)c.
    \]
\end{proof}

\begin{instance}
    Multiplication in $\Frac(\U)$ distributes over addition.
\end{instance}
\begin{proof}
    \begin{align*}
        a(b + c)
        &= \frac{a_1}{a_2} \left( \frac{b_1}{b_2} + \frac{c_1}{c_2} \right) \\
        &= \frac{a_1}{a_2} \, \frac{b_1c_2 + c_1b_2}{b_2c_2} \\
        &= \frac{a_1(b_1c_2 + c_1b_2)}{a_2b_2c_2} \\
        &= \frac{a_1b_1c_2 + a_1c_1b_2}{a_2b_2c_2} \\
        &= \frac{a_2(a_1b_1c_2 + a_1c_1b_2)}{a_2(a_2b_2c_2)} \\
        &= \frac{a_1b_1a_2c_2 + a_1c_1a_2b_2}{a_2b_2a_2c_2} \\
        &= \frac{a_1b_1}{a_2b_2} + \frac{a_1c_1}{a_2c_2} \\
        &= \frac{a_1}{a_2} \, \frac{b_1}{b_2}
            + \frac{a_1}{a_2} \, \frac{c_1}{c_2} \\
        &= ab + ac
    \end{align*}
\end{proof}

\begin{instance}
    Define one in $\Frac(\U)$ as $\iota_\U(1)$.
\end{instance}

\begin{instance}
    One is a multiplicative identity in $\Frac(\U)$.
\end{instance}
\begin{proof}
    \[
        1a
        = \frac{1}{1} \, \frac{a_1}{a_2}
        = \frac{1a_1}{1a_2}
        = \frac{a_1}{a_2}
        = a.
    \]
\end{proof}

\begin{definition}
    Define an operation $\oslash : \U \times \U^* \to \U \times \U^*$ given by
    \[
        \oslash(a_1, a_2) = \begin{cases}
            (a_1, a_2) \text{ if $a_1 = 0$} \\
            (a_2, a_1) \text{ if $a_1 \neq 0$}.
        \end{cases}
    \]
\end{definition}

\begin{lemma}
    The operation $\oslash$ is well-defined.
\end{lemma}
\begin{proof}
    We must prove that if $a \sim b$, then $\oslash a \sim \oslash b$.  There
    will be four cases, depending on if $a_1 = 0$ and $b_1 = 0$.  By symmetry
    the cases $a_1 = 0$, $b_1 \neq 0$ and $a_1 \neq 0$, $b_1 = 0$ are the same,
    so there are three cases.  In each case we have $a_1b_2 = b_1a_2$.
    \begin{case} $a_1 = 0$, $b_1 = 0$.
        In this case we just need to prove that $a \sim b$, which is what we
        started with.
    \end{case}
    \begin{case} $a_1 = 0$, $b_1 \neq 0$.
        From $a_1 = 0$ and $a_1b_2 = b_1a_2$, we see that $0 = b_1a_2$.
        However, this case has $b_1 \neq 0$, and $a_2 \neq 0$ by definition, so
        this is a contradiction.  Thus, this case is impossible.
    \end{case}
    \begin{case} $a_1 \neq 0$, $b_1 \neq 0$.
        We must prove that $a_2b_1 = b_2a_1$.  This is essentially the same as
        what we started with, $a_1b_2 = b_1a_2$.
    \end{case}
\end{proof}

\begin{definition}
    Define multiplicative inverses as the operation produced by Theorem
    \ref{unary_op_ex} and the previous lemma.
\end{definition}

\begin{instance}
    The reciprocal in $\Frac(\U)$ is a multiplicative inverse.
\end{instance}
\begin{proof}
    Let $a \neq 0$.  This means that $a_1 \neq 0$.  Then
    \[
        a a^{-1} = \frac{a_1}{a_2} \, \frac{a_2}{a_1}
        = \frac{a_1a_2}{a_2a_1}
        = \frac{1}{1}
        = 1.
    \]
\end{proof}

\begin{instance}
    $\iota_\U$ is multiplicative.
\end{instance}
\begin{proof}
    \[
        \iota_\U(ab)
        = \frac{ab}{1}
        = \frac{ab}{(1)(1)}
        = \frac{a}{1} \, \frac{b}{1}
        = \iota_\U(a) \iota_\U(b).
    \]
\end{proof}

\begin{instance}
    $\iota_\U$ preserves the multiplicative identity.
\end{instance}
\begin{proof}
    This was simply the definition of $1$ in $\Frac(\U)$.
\end{proof}

We can now show that the notation $\frac{a}{b}$ is consistent with our usual use
of division:

\begin{theorem}
    For all $a$ and $b$, $[(a, b)] = \iota_\U(a) \iota_\U(b)^{-1}$
\end{theorem}
\begin{proof}
    By assumption we have $b \neq 0$.  Thus,
    \[
        \iota_\U(a) \iota_\U(b)^{-1}
        = [(a, 1)] [(b, 1)]^{-1}
        = [(a, 1)] [(1, b)]
        = [(a, b)].
    \]
\end{proof}

Note that this theorem also implies that for all $x : \Frac(\U)$, there exist
values $a$ and $b$ in $\U$ such that $x = \frac{a}{b}$.

\section{Order}

In this section, assume that $\U$ is an ordered integral domain, not just an
integral domain.

\begin{theorem} \label{frac_pos_ex} \label{frac_pos_ex_div}
    For all $x : \Frac(\U)$, there exist $a$ and $b$ in $\U$ such that $0 < b$
    and $x = \frac{a}{b}$.
\end{theorem}
\begin{proof}
    If $0 < x_2$, then we simply have $x = \frac{x_1}{x_2}$.  If $x_2 < 0$, then
    $x = \frac{-x_1}{-x_2}$.
\end{proof}

To define an order on $\U$, we will use a cone.

\begin{definition}
    Say that a pair $(a, b) : \U \times \U^*$ is positive if $0 \leq ab$.
\end{definition}

\begin{lemma}
    Being positive is well-defined.
\end{lemma}
\begin{proof}
    By propositional extensionality, it suffices to prove that if $a_1b_2 =
    b_1a_2$ and $0 \leq a_1a_2$, then $0 \leq b_1b_2$.  Because all squares are
    positive, we have $0 \leq b_2^2$, and by multiplicativity we have $0 \leq
    b_2^2 a_1a_2 = b_2b_1a_2^2$, and cancelling $a_2^2$ (which is nonzero by
    Theorem \ref{mult_nz}) we get $0 \leq b_1b_2$.
\end{proof}

\begin{definition}
    Define a set $C : \set{\Frac(\U)}$ given by the result of Theorem
    \ref{unary_op_ex} with the previous lemma.
\end{definition}

\begin{lemma} \label{frac_pos_simpl}
    For all $a$ and $b$ with $b > 0$, $\frac{a}{b} \in C$ if and only if $0 \leq
    a$.
\end{lemma}
\begin{proof}
    $\frac{a}{b} \in C$ means that $0 \leq ab$.  Because $b > 0$, we can either
    multiply or cancel it to get to and from $0 \leq a$.
\end{proof}

\begin{instance}
    The set $C$ is a cone.
\end{instance}
\begin{proof}
    \textit{Addition.} Let $a \in C$ and $b \in C$.  By Theorem
    \ref{frac_pos_ex}, we can assume that $a_2$ and $b_2$ are greater than zero.
    Then $a \in C$ is equivalent to $0 \leq a_1$ and $b \in C$ is equivalent to
    $0 \leq b_1$, and $a + b \in C$ is equivalent to $0 \leq a_1b_2 + b_1a_2$.
    Every quantity in this is positive, so $a + b \in C$.

    \textit{Multiplication.} Let $a \in C$ and $b \in C$.  By Theorem
    \ref{frac_pos_ex}, we can assume that $a_2$ and $b_2$ are greater than zero.
    Then $a \in C$ is equivalent to $0 \leq a_1$ and $b \in C$ is equivalent to
    $0 \leq b_1$, and $ab \in C$ is equivalent to $0 \leq a_1b_1$.  This follows
    directly by multiplicativity of $\leq$.

    \textit{Squares.} Given an $a : \Frac(\U)$, by Theorem \ref{frac_pos_ex}, we
    can assume that $a_2 > 0$.  Then $a^2 \in C$ is equivalent to $0 \leq
    a_1^2$, which is true because all squares are positive.

    \textit{Negative One.} That $-1 \in p$ is equivalent to saying that $0 \leq
    -1$.  To say that that's not true is to say that $-1 < 0$.  This follows
    from one being positive.

    \textit{Totality.} Given an $a : \Frac(\U)$, by Theorem \ref{frac_pos_ex} we
    can assume that $a_2 > 0$.  Then $a \in C$ is equivalent to $0 \leq a_1$,
    and $-a \in C$ is equivalent to $0 \leq -a_1$.  This is true by the
    connexivity of $\leq$ in $\U$.
\end{proof}

\begin{theorem} \label{frac_le}
    If $0 < a_2$ and $0 < b_2$, then
    \[
        \frac{a_1}{a_2} \leq \frac{b_1}{b_2}
    \]
    if and only if
    \[
        a_1b_2 \leq b_1a_2.
    \]
\end{theorem}
\begin{proof}
    The first inequality is equivalent to checking if
    \[
        \frac{b_1}{b_2} - \frac{a_1a_2}
        = \frac{b_1a_2 - a_1b_2}{b_2a_2}
    \]
    is in $C$.  By Lemma \ref{frac_pos_simpl}, this is equivalent to
    \[
        0 \leq b_1a_2 - a_1b_2,
    \]
    and rearranging we get
    \[
        a_1b_2 \leq b_1a_2.
    \]
\end{proof}

\begin{theorem} \label{frac_lt}
    If $0 < a_2$ and $0 < b_2$, then
    \[
        \frac{a_1}{a_2} < \frac{b_1}{b_2}
    \]
    if and only if
    \[
        a_1b_2 < b_1a_2.
    \]
\end{theorem}
\begin{proof}
    The $\leq$ part is the previous theorem, and the $\neq$ part is identical
    after unfolding the definition of equality in $\Frac(\U)$.
\end{proof}

\begin{instance}
    $\iota_\U$ preserves inequalities in one direction.
\end{instance}
\begin{proof}
    Assume that $a \leq b$.  Then $a1 \leq b1$, and by Theorem \ref{frac_le},
    this is the same as $\frac{a}{1} \leq \frac{b}{1}$.
\end{proof}

\begin{lemma} \label{to_frac_nat}
    For all $n : \N$, $n = \iota_\U(n)$.
\end{lemma}
\begin{proof}
    The proof will be by induction on $n$.  When $n = 0$, $0 = \iota_\U(0)$, so
    the base case is true.  Now assume that $n = \iota_\U(n)$.  Then
    \[
        S(n) = 1 + n = 1 + \iota_\U(n) = \iota_\U(1 + n) = \iota_\U(S(n)),
    \]
    so the theorem is true by induction.
\end{proof}

\begin{instance}
    If $\U$ is Archimedean, then $\Frac(\U)$ is as well.
\end{instance}
\begin{proof}
    We will prove that for all $x : \Frac(\U)$ such that $x > 0$, there exists
    an $n : \N$ such that $x < n$.  Let $x = \frac{a}{b}$ with $b > 0$.  Then by
    Lemma \ref{frac_pos_simpl}, from $x > 0$, we have $a > 0$.  Because $\U$ is
    archimedean and $a$ and $b$ are both positive, there exists an $n : \N$ such
    that $a < nb$.  This means that $a1 < nb$, so by Theorem \ref{frac_lt}, we
    have $\frac{a}{b} < \frac{n}{1}$, and $\frac{n}{1} = n$ by Lemma
    \ref{to_frac_nat}.  Thus, we have an $n > \frac{a}{b}$ as required.
\end{proof}

\section{Universal Property}

\begin{theorem}
    Let $F$ be the forgetful functor from fields to integral domains.  Then for
    any given integral domain $\U$, the pair $(\Frac(\U), \iota_\U)$ is initial
    in the comma category $(\U \downarrow F)$
\end{theorem}
\begin{proof}
    Simplified, we must prove that for any field $\F$ and domain homomorphism $f
    : \U \to F(\F)$, there exists a unique field homomorphism $g : \Frac(\U) \to
    \F$ such that for all $x : \U$, $g(\iota_\U(x)) = f(x)$.

    For existence, first define the function $g' : \U \times \U^* \to \F$ given
    by
    \[
        g'(a, b) = \frac{f(a)}{f(b)}.
    \]
    The denominator is nonzero because $f$ is injective.  We will now prove that
    $g'$ is well-defined.  Let $a$ and $b$ be in $\U \times \U^*$ such that $a
    \sim b$.  Then $a_1b_2 = b_1a_2$.  Then
    \begin{align*}
        a_1b_2 &= b_1a_2 \\
        f(a_1b_2) &= f(b_1a_2) \\
        f(a_1)f(b_2) &= f(b_1)f(a_2) \\
        \frac{f(a_1)}{f(a_2)} &= \frac{f(b_1)}{f(b_2)} \\
        g'(a) &= g'(b),
    \end{align*}
    showing that $g'$ is well-defined.  Let $g$ be the function given by Theorem
    \ref{unary_op_ex}.  We will prove that this function is the required
    homomorphism.  We need to prove that it is in fact a homomorphism, and that
    it satisfies $g(\iota_\U(x)) = f(x)$.

    \textit{Additivity.}
    \begin{align*}
        g(a + b)
        &= \frac{f(a_1b_2 + b_1a_2)}{f(a_2b_2)} \\
        &= \frac{f(a_1)f(b_2) + f(b_1)f(a_2)}{f(a_2)f(b_2)} \\
        &= \frac{f(a_1)f(b_2)}{f(a_2)f(b_2)}
            + \frac{f(b_1)f(a_2)}{f(a_2)f(b_2)} \\
        &= \frac{f(a_1)}{f(a_2)} + \frac{f(b_1)}{f(b_2)} \\
        &= g(a) + g(b).
    \end{align*}

    \textit{Multiplicativity.}
    \begin{align*}
        g(ab)
        &= \frac{f(a_1b_1)}{f(a_2b_2)} \\
        &= \frac{f(a_1)f(b_1)}{f(a_2)f(b_2)} \\
        &= \frac{f(a_1)}{f(a_2)} \, \frac{f(b_1)}{f(b_2)} \\
        &= g(a) g(b).
    \end{align*}

    \textit{Multiplicative Identity.}
    \begin{align*}
        g(1) = \frac{f(1)}{f(1)} = \frac{1}{1} = 1.
    \end{align*}

    \textit{Extension of $f$.}
    \begin{align*}
        g(\iota_\U(a)) = \frac{f(a)}{f(1)} = \frac{f(a)}{1} = f(a).
    \end{align*}

    Now for uniqueness, let $h$ be another homomorphism with $h(\iota_\U(x)) =
    f(x)$.  Then for all $a : \Frac(\U)$,
    \begin{align*}
        \frac{f(a_1)}{f(a_2)} &= \frac{f(a_1)}{f(a_2)} \\
        \frac{g(\iota_\U(a_1))}{g(\iota_\U(a_2))} &=
            \frac{h(\iota_\U(a_1))}{h(\iota_\U(a_2))} \\
        g \left( \frac{\iota_\U(a_1)}{\iota_\U(a_2)} \right) &=
            h \left( \frac{\iota_\U(a_1)}{\iota_\U(a_2)} \right) \\
        g(a) = h(a),
    \end{align*}
    showing that $g = h$, so $g$ is unique.
\end{proof}

By the results of section \ref{cat_comma_free}, we see that $\Frac$ is a
functor, and we can now do extensions.

A similar universal property holds for ordered integral domains:

\begin{theorem}
    Let $F$ be the forgetful functor from ordered fields to ordered domains.
    Then for any given ordered domain $\U$, the pair $(\Frac(\U), \iota_\U)$ is
    initial in the comma category $(\U \downarrow F)$
\end{theorem}
\begin{proof}
    All of the parts of the previous proof works exactly the same here, so all
    we need to prove is that $g$ preserves inequalities in one direction.  Let
    $a$ and $b$ be values in $\Frac(\U)$.  Then by Theorem \ref{frac_pos_ex}, we
    can assume that $a_2 > 0$ and $b_2 > 0$.  Then
    \begin{align*}
        a &\leq b \\
        a_1b_2 &\leq b_1a_2 \\
        f(a_1b_2) &\leq f(b_1a_2) \\
        f(a_1)f(b_2) &\leq f(b_1)f(a_2) \\
        \frac{f(a_1)}{f(a_2)} &\leq \frac{f(b_1)}{f(b_2)} \\
        g(a) \leq g(b).
    \end{align*}
\end{proof}

We will call the functor from ordered domains to ordered fields $\OFrac$ rather
than $\Frac$.

\section{The Rationals}

We will define the rationals $\Q$ as the ordered field $\OFrac(\Z)$.

\begin{theorem}
    For all $n : \Z$, $\iota_\Z$ is equal to the injection $\iota$ defined
    above.
\end{theorem}
\begin{proof}
    The result immediately follows by Theorem \ref{from_int_unique}.
\end{proof}

\begin{definition}
    For any field $\F$ of characteristic zero, we define an inclusion $\iQ : \Q
    \to \F$ as the extension of $\iZ$.  If the field is ordered we use the
    extension given by $\OFrac$.  After this chapter, the function $\iQ$ will
    not be written and will be assumed to be used any time a rational appears in
    an expression involving other types, unless the notation is necessary to
    avoid ambiguity.
\end{definition}

\begin{theorem}
    For all $n : \N$, $\iQ(n) = n$.
\end{theorem}
\begin{proof}
    The proof will be by induction on $n$.  When $n = 0$, $\iQ(0) = 0$, so the
    base case is true.  Now assume that $\iQ(n) = n$.  Then
    \[
        \iQ(S(n)) = \iQ(1 + n) = 1 + \iQ(n) = 1 + n = S(n),
    \]
    so the theorem is true by induction.
\end{proof}

\begin{theorem}
    For all $n : \Z$, $\iQ(n) = n$.
\end{theorem}
\begin{proof}
    This is a direct consequence of Theorem \ref{from_int_unique}.
\end{proof}

\begin{theorem} \label{rat_initial}
    The rationals are initial in \vtt{OField}.
\end{theorem}
\begin{proof}
    $\iQ$ is a morphism from the rationals to every ordered field, so all we
    need to prove is that $\iQ$ is unique.  Let $f$ be any morphism from $\Q$ to
    an ordered field $\F$.  Then $f(n) = n$ for all $n : \Z$ by Theorem
    \ref{from_int_unique}.  Now for any $x : \Q$,
    \begin{align*}
        f(x)
        &= f \left( \frac{x_1}{x_2} \right) \\
        &= \frac{f(x_1)}{f(x_2)} \\
        &= \frac{x_1}{x_2} \\
        &= \frac{\iQ(x_1)}{\iQ(x_2)} \\
        &= \iQ \left( \frac{x_1}{x_2} \right) \\
        &= \iQ(x),
    \end{align*}
    showing that $f = \iQ$, so $\iQ$ is unique.
\end{proof}

\begin{theorem}
    The rationals are dense in every Archimedean ordered field.  To be precise,
    in any Archimedean ordered field $\U$ and values $a$ and $b$ in $\U$ such
    that $a < b$, there exists an $r : \Q$ such that $a < r < b$.
\end{theorem}
\begin{proof}
    We will first prove the result in the case that $0 \leq a$.  Because $\U$ is
    Archimedean and $b - a > 0$, there exists an $n$ such that $\frac{1}{n} <
    b - a$, so $1 + an < bn$.  Again, because $\U$ is Archimedean, there exists
    a natural number $m$ such that $an < m$.  Let $m$ be the least such number.
    We will prove that $\frac{m}{n}$ is the required rational number.  First, by
    $an < m$, we already have $a < \frac{m}{n}$.  To prove that $\frac{m}{n} <
    b$, it suffices to prove that $m < bn$.  If $m = 0$, the result follows from
    $n \neq 0$ and $0 < b$.  For the case $m \neq 0$, because $1 + an < bn$, by
    transitivity it suffices to prove that $m \leq 1 + an$.  If this was false,
    we would have $an < m - 1$, contradicting the minimality of $m$.  Thus,
    $a < \frac{m}{n} < b$.

    Now for the general case, we have three cases: When $a > 0$, the previous
    paragraph works.  When $a < 0$ and $b > 0$, $0$ itself works.  When $a < 0$
    and $b \leq 0$, then $0 \leq -b$ and $-b < -a$, by the previous paragraph
    there is an $r : \Q$ such that $-b < r < -a$.  Then $a < -r < b$.  Thus, the
    rationals are dense in $\U$.
\end{proof}

\end{document}
