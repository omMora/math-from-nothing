\documentclass[../../math.tex]{subfiles}
\externaldocument{../../math.tex}

\begin{document}

\setcounter{chapter}{7}

\chapter{The Rationals}

The rationals can be constructed from the integers by considering pairs of
integers $(a, b)$, thought of as the fraction $\frac{a}{b}$.  However, this
construction is more general than just for constructing the rationals from the
integers and applies to any integral domain.  Thus, in this chapter, most of the
construction will be using an arbitrary integral domain, and the rationals and
their unique properties won't be explored until the end.

\section{Basic Construction}

Let $\U$ be an integral domain, and let $\U^*$ be the type of all nonzero values
in $\U$.

\begin{definition}
    Define a relation $\sim$ on $\U \times \U^*$ where $(a_1, a_2) \sim (b_1,
    b_2)$ is defined to mean $a_1b_2 = b_1a_2$.
\end{definition}

\begin{lemma}
    The relation $\sim$ is an equivalence relation.
\end{lemma}
\begin{proof}
    \textit{Reflexivitity.}  We must check $a_1a_2 = a_1a_2$, which is a
    reflexive equality.

    \textit{Symmetry.}  We must prove that if $a_1b_2 = b_1a_2$, then $b_1a_2 =
    a_1b_2$.  This is true by the symmetry of equality.

    \textit{Transitivity.}  We must prove that if $a_1b_2 = b_1a_2$ and
    $b_1c_2 = c_1b_2$, then $a_1c_2 = c_1a_2$.
    \begin{align*}
        b_1a_2c_2 &= b_1c_2a_2 \\
        a_1b_2c_2 &= c_1b_2a_2 \\
        a_1c_2 &= c_1a_2,
    \end{align*}
    where the cancelling is valid because $b_2 \neq 0$ by definition.
\end{proof}

\begin{definition}
    Define the type $\Frac(\U)$ to be the type $(\U \times \U^*)/\sim$.
\end{definition}

\begin{definition}
    Define the inclusion $\iota_\U : \U \to \frac(\U)$ that takes an $x : \U$
    and bring it to $[(x, 1)]$.
\end{definition}

\begin{instance}
    $\iota_\U$ is injective.
\end{instance}
\begin{proof}
    If $\iota_\U(a) = \iota_\U(b)$, this means that $a1 = b1$, meaning that $a =
    b$.
\end{proof}

\begin{instance}
    $\Frac(\U)$ is not trivial.
\end{instance}
\begin{proof}
    Because $0 \neq 1$ in $\U$, by $\iota_\U$ being injective we have
    $\iota_\U(0) \neq \iota_\U(1)$.
\end{proof}

\section{Addition}

\section{Multiplication}

\section{Order}

\section{The Frac functor}

\section{The rationals}

\end{document}
