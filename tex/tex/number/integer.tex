\documentclass[../../math.tex]{subfiles}
\externaldocument{../../math.tex}

\begin{document}

\setcounter{chapter}{6}

\chapter{The Integers} \label{chap_integer}

\begin{definition}
    Define a relation $\sim$ on $\N \times \N$ where $(a_1, a_2) \sim (b_1,
    b_2)$ is defined to mean $a_1 + b_2 = b_1 + a_2$.
\end{definition}

\begin{lemma}
    The relation $\sim$ is an equivalence relation on $\N \times \N$.
\end{lemma}
\begin{proof}
    \textit{Reflexivitity.}  We must check $a_1 + a_2 = a_1 + a_2$, which is a
    reflexive equality.

    \textit{Symmetry.}  We must prove that if $a_1 + b_2 = b_1 + a_2$, then $b_1
    + a_2 = a_1 + b_2$.  This is true by the symmetry of equality.

    \textit{Transitivity.}  We must prove that if $a_1 + b_2 = b_1 + a_2$ and
    $b_1 + c_2 = c_1 + b_2$, then $a_1 + c_2 = c_1 + a_2$.  We can add the first
    two equalities to get
    \[
        a_1 + b_2 + b_1 + c_2 = c_1 + b_2 + b_1 + a_2.
    \]
    We can cancel $b_2$ and $b_1$ to get the result.
\end{proof}

\begin{definition}
    Define the type of integers $\Z$ to be the type $(\N \times \N)/\usim$.
\end{definition}

\end{document}
