\documentclass[../../math.tex]{subfiles}
\externaldocument{../../math.tex}
\externaldocument{../basics/set}

\begin{document}

\setcounter{chapter}{6}

\chapter{The Integers} \label{chap_integer}

\begin{definition}
    Define a relation $\sim$ on $\N \times \N$ where $(a_1, a_2) \sim (b_1,
    b_2)$ is defined to mean $a_1 + b_2 = b_1 + a_2$.
\end{definition}

\begin{lemma}
    The relation $\sim$ is an equivalence relation on $\N \times \N$.
\end{lemma}
\begin{proof}
    \textit{Reflexivitity.}  We must check $a_1 + a_2 = a_1 + a_2$, which is a
    reflexive equality.

    \textit{Symmetry.}  We must prove that if $a_1 + b_2 = b_1 + a_2$, then $b_1
    + a_2 = a_1 + b_2$.  This is true by the symmetry of equality.

    \textit{Transitivity.}  We must prove that if $a_1 + b_2 = b_1 + a_2$ and
    $b_1 + c_2 = c_1 + b_2$, then $a_1 + c_2 = c_1 + a_2$.  We can add the first
    two equalities to get
    \[
        a_1 + b_2 + b_1 + c_2 = c_1 + b_2 + b_1 + a_2.
    \]
    We can cancel $b_2$ and $b_1$ to get the result.
\end{proof}

\begin{definition}
    Define the type of integers $\Z$ to be the type $(\N \times \N)/\usim$.
\end{definition}

\section{Addition}

\begin{definition}
    Define an operation $\oplus : \N \times \N \to \N \times \N \to \N \times
    \N$ given by
    \[
        (a_1, a_2) \oplus (b_1, b_2) = (a_1 + b_1, a_2 + b_2).
    \]
\end{definition}

\begin{lemma}
    The operation $\oplus$ is well-defined under the equivalence relation
    $\sim$.
\end{lemma}
\begin{proof}
    We must prove that for all $a$, $b$, $c$, and $d$, if $a \sim b$ and $c \sim
    d$, we have $a \oplus c \sim b \oplus d$.  Thus, we have
    \[
        a_1 + b_2 = b_1 + a_2
    \]
    and
    \[
        c_1 + d_2 = d_1 + c_2
    \]
    and must prove that
    \[
        a_1 + c_1 + b_2 + d_2 = b_1 + d_1 + a_2 + c_2.
    \]
    This follows directly by rearranging and applying the previous two
    equalities.
\end{proof}

\begin{instance}
    Definition addition in the integers as the binary operation given by Theorem
    \ref{binary_op_ex} and the previous lemma.
\end{instance}

\begin{instance}
    Addition of integers is commutative.
\end{instance}
\begin{proof}
    We must prove that
    \[
        (a_1, a_2) \oplus (b_1, b_2) \sim (b_1, b_2) \oplus (a_1, a_2).
    \]
    This reduces to
    \[
        a_1 + b_1 + b_2 + a_2 = b_1 + a_1 + a_2 + b_2,
    \]
    which follows by the commutativity of natural number addition.
\end{proof}

\begin{instance}
    Addition of integers is associative.
\end{instance}
\begin{proof}
    We must prove that
    \[
        (a_1, a_2) \oplus ((b_1, b_2) \oplus (c_1, c_2)) \sim
        ((a_1, a_2) \oplus (b_1, b_2) \oplus (c_1, c_2)).
    \]
    This reduces to
    \[
        (a_1 + (b_1 + c_1)) + ((a_2 + b_2) + c_2) =
        ((a_1 + b_1) + c_1) + (a_2 + (b_2 + c_2)),
    \]
    which follows by the associativity of natural number addition.
\end{proof}

\begin{instance}
    Define zero in the integers to be $[(0, 0)]$.
\end{instance}

\begin{instance}
    Zero is an additive identity in the integers.
\end{instance}
\begin{proof}
    We must prove that
    \[
        (0, 0) \oplus (a_1, a_2) \sim (a_1, a_2).
    \]
    This reduces to
    \[
        0 + a_1 + a_2 = a_1 + 0 + a_2,
    \]
    which follows by 0 being an identity in the natural numbers.
\end{proof}

\begin{definition}
    Given $(a_1, a_2) : \N$, define
    \[
        \ominus (a_1, a_2) = (a_2, a_1).
    \]
\end{definition}

\begin{lemma}
    $\ominus$ is well-defined under $\sim$.
\end{lemma}
\begin{proof}
    We must prove that if $(a_1, a_2) \sim (b_1, b_2)$, then $\ominus (a_1, a_2)
    \sim \ominus (b_1, b_2)$.  Simplified, we have $a_1 + b_2 = b_1 + a_2$ and
    must prove that $a_2 + b_1 = b_2 + a_1$, which follows from commutativity.
\end{proof}

\begin{instance}
    Definition additive inverses in the integers as the unary operation given by
    Theorem \ref{unary_op_ex} and the previous lemma.
\end{instance}

\begin{instance}
    Negation is a left inverse in the integers.
\end{instance}
\begin{proof}
    We must prove that
    \[
        \ominus (a_1, a_2) \oplus (a_1, a_2) \sim (0, 0).
    \]
    This reduces to
    \[
        a_2 + a_1 + 0 = 0 + a_1 + a_2,
    \]
    which follows by the properties of natural number addition.
\end{proof}

\section{Multiplication}

\section{Order}

\section{The Relation Between Integers And Other Types}

\end{document}
