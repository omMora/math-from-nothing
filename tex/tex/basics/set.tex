\documentclass[../../math.tex]{subfiles}
\externaldocument{../../math.tex}
\externaldocument{foundations}
\externaldocument{elementary_algebra}
\externaldocument{natural}

\begin{document}

\setcounter{chapter}{3}

\chapter{Basic Set Theory} \label{chap_set}

As mentioned in Chapter \ref{chap_foundations}, sets are thought of as
predicates.  To be precise, given a type $\U$, sets are thought of as functions
$S : \U \rightarrow \vtt{Prop}$, where $a \in S$ is defined to be $S(a)$.  Note
that this is very different from the usual way of thinking about sets.  While in
set theories like ZFC you can have any elements you want in a set, here sets
must only contain values from a single type.  While this won't be too big of an
issue in this chapter and in most applications, it will cause radical changes in
our understanding of the cardinals and the ordinals.

\section{Basic Set Operations}

In this section, let $\U$ be a type.  All sets in this section will be sets of
values in $\U$ unless said otherwise.  There are many interesting ideas related
to sets that we can define:

\begin{definition}
    Given two sets $S$ and $T$, we say that $S$ is a subset of $T$ if for all $x
    \in S$, $x \in T$ as well.  We write this relationship as $S \subseteq T$.
    Given that this is a relation defined on sets, we can also define a strict
    relation on sets $S \subseteq T \wedge S \neq T$, which we denote with $S
    \subset T$.
\end{definition}
\noindent Note that unlike in traditional set theory, the relation $S \subseteq
T$ is only defined when $S$ and $T$ are both sets on the same type.

\begin{definition}
    We define the empty set $\varnothing$ on $\U$ as the set $\varnothing = \{x
    : \U \mid \vtt{False}\}$.
\end{definition}
\noindent This is again a break from traditional set theory.  While there is
usually only one empty set, here we have a different empty set for every type.

\begin{definition}
    We define the universal set $\bm U$ on $\U$ as the set $U = \{x : \U \mid
    \vtt{True}\}$.
\end{definition}
\noindent This is another break from traditional set theory since universal sets
can't even be defined in traditional set theory.

\begin{definition}
    Given values $x_1, x_2, \cdots, x_n : \U$, we define the set
    \[
        \{x_1, x_2, \cdots, x_n\} =
        \{y : \U \mid x_1 = y \vee x_2 = y \vee \cdots \vee x_n = y\}.
    \]
\end{definition}

\begin{definition}
    Given two sets $S$ and $T$, we define their union $S \cup T = \{x : \U \mid
    x \in S \vee x \in T\}$.
\end{definition}

\begin{definition}
    Given two sets $S$ and $T$, we define their intersection $S \cap T = \{x :
    \U \mid x \in S \wedge x \in T\}$.
\end{definition}

\begin{definition}
    Given a set $S$, we define its complement $S^c = \{x : \U \mid x \notin
    S\}$.
\end{definition}

\begin{definition}
    Given two sets $S$ and $T$, we define their difference $S - T = S \cap T^c$.
\end{definition}

\begin{definition}
    Given two sets $S$ and $T$, we define their symmetric difference $S \ominus
    T = (S - T) \cup (T - S)$.
\end{definition}
\noindent Notice that like with the subset relation, all of these binary
operations on sets are only defined for sets on the same type.

\begin{definition}
    We say that two sets $S$ and $T$ are disjoint if $S \cap T = \varnothing$,
    and we say that they intersect otherwise.
\end{definition}

While most uses of the Cartesian product are better handled using a product type
as defined in Section \ref{sec_types}, there are still a few cases where the
Cartesian product of sets is useful, so we will still define it here.
\begin{definition}
    Given two types $\U$ and $\V$ with sets $S : \U \to \Prop$ and $T : \V \to
    \Prop$, we define the Cartesian product $S \times T : (\U \times \V) \to
    \Prop$ to be the set $\{(x, y) : \U \times \V \mid x \in S \wedge y \in
    T\}$.
\end{definition}

Note that the power set is not defined here.  This is because it is actually
already defined.  We think of sets as being values in the type $\U \rightarrow
\Prop$, which means that the type $\U \rightarrow \Prop$ itself is the power set
of $\U$.

\begin{instance}
    $\subseteq$ is reflexive.
\end{instance}
\begin{proof}
    Let $S$ be a set.  We must prove that for all $x \in S$, $x \in S$.  Thus
    what we must prove is precisely what we started with.
\end{proof}

\begin{instance}
    $\subseteq$ is transitive.
\end{instance}
\begin{proof}
    Let $R$, $S$, and $T$ be sets such that $R \subseteq S$ and $S \subseteq T$.
    Now let $x \in R$.  Because $R \subseteq S$, $x \in S$.  Now because $S
    \subseteq T$, $x \in T$.  Because $x$ was an arbitrary element of $R$, $R
    \subseteq T$ and $\subseteq$ is transitive.
\end{proof}

\begin{instance}
    $\subseteq$ is antisymmetric.
\end{instance}
\begin{proof}
    Let $S$ and $T$ be sets such that $S \subseteq T$ and $T \subseteq S$.  By
    predicate extensionality, we need to prove that for all $x : \U$, $x \in S
    \leftrightarrow x \in T$.  This is equivalent to $(x \in S \rightarrow x \in
    T) \wedge (x \in T \rightarrow x \in S)$, and each individual statement
    follows from $S \subseteq T$ and $T \subseteq S$.  Thus, $\subseteq$ is
    antisymmetric.
\end{proof}

This shows us that $\subseteq$ partially orders sets on a type.

\begin{theorem} \label{strict_subset_ex}
    For all sets $S$ and $T$, if $S \subset T$, then there exists an $x$ such
    that $x \in T$ and $x \notin S$.
\end{theorem}
\begin{proof}
    Assume the opposite, that is, that for all $x$, if $x \in T$, then $x \in
    S$.  This means that $T \subseteq S$.  Because $S \subseteq T$, we have $S =
    T$, contradicting $S \subset T$.
\end{proof}

\begin{theorem}
    For all sets $S$, $\varnothing \subseteq S$.
\end{theorem}
\begin{proof}
    Let $x \in \varnothing$.  This is already a contradiction, so the theorem
    holds vacuously.
\end{proof}

\begin{theorem}
    For all sets $S$, $S \subseteq \bm U$.
\end{theorem}
\begin{proof}
    Let $x \in S$.  $x \in \bm U$ is always true, so $S \subseteq \bm U$.
\end{proof}

\begin{theorem} \label{empty_eq}
    For all sets $S$, $S = \varnothing$ if and only if $\forall x, x \notin S$.
\end{theorem}
\begin{proof}
    If $S = \varnothing$, then we have $\forall x, x \notin \varnothing$, so the
    forward implication is true.  If $\forall x, x \notin S$, we have $S
    \subseteq \varnothing$ vacuously, and because $\varnothing \subseteq S$, we
    have $S = \varnothing$ by antisymmetry, so the reverse implication is true
    as well.
\end{proof}

\begin{theorem} \label{empty_neq}
    For all sets $S$, $S \neq \varnothing$ if and only if $\exists x, x \in S$.
\end{theorem}
\begin{proof}
    \begin{align*}
        S = \varnothing &\leftrightarrow \forall x, x \notin S \\
        S \neq \varnothing &\leftrightarrow \neg(\forall x, x \notin S) \\
        &\leftrightarrow \exists x, x \in S.
    \end{align*}
\end{proof}

\begin{theorem} \label{all_eq}
    For all sets $S$, $S = \bm U$ if and only if $\forall x, x \in S$.
\end{theorem}
\begin{proof}
    If $S = \bm U$, then we have $\forall x, x \in \bm U$, so the forward
    implication is true.  If $\forall x, x \in S$, we have $\bm U \subseteq S$,
    and because $S \subseteq \bm U$, we have $S = \bm U$ by antisymmetry, so the
    reverse implication is true as well.
\end{proof}

\begin{theorem} \label{all_neq}
    For all sets $S$, $S \neq \bm U$ if and only if $\exists x, x \notin S$.
\end{theorem}
\begin{proof}
    \begin{align*}
        S = \bm U &\leftrightarrow \forall x, x \in S \\
        S \neq \bm U &\leftrightarrow \neg(\forall x, x \in S) \\
                 &\leftrightarrow \exists x, x \notin S.
    \end{align*}
\end{proof}

\begin{theorem} \label{union_comm}
    For all sets $S$ and $T$, $S \cup T = T \cup S$.
\end{theorem}
\begin{proof}
    This follows from predicate extensionality and Theorem \ref{or_comm}.
\end{proof}

\begin{theorem} \label{union_assoc}
    For all sets $R$, $S$, and $T$, $R \cup (S \cup T) = (R \cup S) \cup T$.
\end{theorem}
\begin{proof}
    This follows from predicate extensionality and Theorem \ref{or_assoc}.
\end{proof}

\begin{theorem} \label{union_lid}
    For all sets $S$, $\varnothing \cup S = S$.
\end{theorem}
\begin{proof}
    This follows from predicate extensionality and Theorem \ref{or_lfalse}.
\end{proof}

\begin{theorem} \label{union_lanni}
    For all sets $S$, $\bm U \cup S = \bm U$.
\end{theorem}
\begin{proof}
    This follows from predicate extensionality and Theorem \ref{or_ltrue}.
\end{proof}

\begin{theorem} \label{union_lsub}
    For all sets $S$ and $T$, $S \subseteq S \cup T$.
\end{theorem}
\begin{proof}
    Trivial.
\end{proof}

\begin{theorem} \label{union_compl_all}
    For all sets $S$, $S \cup S^c = \bm U$.
\end{theorem}
\begin{proof}
    This follows from Theorem \ref{all_eq} and the law of the excluded middle.
\end{proof}

\begin{theorem} \label{union_idemp}
    For all sets $S$, $S \cup S = S$.
\end{theorem}
\begin{proof}
    This follows from predicate extensionality and Theorem \ref{or_idemp}.
\end{proof}

\begin{theorem} \label{union_minus}
    For all sets $A$ and $B$, if $A \cap B = \varnothing$, then $A \cup B - B f
    A$.
\end{theorem}
\begin{proof}
    The proof will be by antisymmetry.  Let $x \in A \cup B - B$.  Then $x \in A
    \cup B$ and $x \notin B$.  Because $x \in A \cup B$, either $x \in A$ or $x
    \in B$.  $x \in B$ is impossible because $x \notin B$, so we must have $x \in
    A$.  Thus, $A \cup B - B \subseteq A$.

    Now let $x \in A$.  This means that $x \in A \cup B$.  Because $A \cap B =
    \varnothing$, we can't have $x \in B$.  Thus, $x \in A \cup B - B$, showing
    that $A \subseteq A \cup B - B$.

    Because $A \cup B - B \subseteq A$ and $A \subseteq A \cup B - B$, we have
    $A \cup B - B = A$ by antisymmetry.
\end{proof}

\begin{theorem} \label{inter_comm}
    For all sets $S$ and $T$, $S \cap T = T \cap S$.
\end{theorem}
\begin{proof}
    This follows from predicate extensionality and Theorem \ref{and_comm}.
\end{proof}

\begin{theorem} \label{inter_assoc}
    For all sets $R$, $S$, and $T$, $R \cap (S \cap T) = (R \cap S) \cap T$.
\end{theorem}
\begin{proof}
    This follows from predicate extensionality and Theorem \ref{and_assoc}.
\end{proof}

\begin{theorem} \label{inter_lid}
    For all sets $S$, $\bm U \cap S = S$.
\end{theorem}
\begin{proof}
    This follows from predicate extensionality and Theorem \ref{and_ltrue}.
\end{proof}

\begin{theorem} \label{inter_lanni}
    For all sets $S$, $\varnothing \cap S = \varnothing$.
\end{theorem}
\begin{proof}
    This follows from predicate extensionality and Theorem \ref{and_lfalse}.
\end{proof}

\begin{theorem} \label{inter_lsub}
    For all sets $S$ and $T$, $S \cap T \subseteq S$.
\end{theorem}
\begin{proof}
    Trivial.
\end{proof}

\begin{theorem} \label{lsub_inter_equal}
    For all sets $S$ and $T$, if $S \subseteq T$, $S \cap T = S$.
\end{theorem}
\begin{proof}
    We already have $S \cap T \subseteq S$, so we must prove that $S \subseteq S
    \cap T$.  Let $x \in S$.  Then we have $x \in S$, and because $S \subseteq
    T$, we have $x \in T$ as well.  Thus, $x \in S \cap T$, so $S \subseteq S
    \cap T$.
\end{proof}

\begin{theorem} \label{inter_compl_empty}
    For all sets $S$, $S \cap S^c = \varnothing$.
\end{theorem}
\begin{proof}
    $x \in S$ and $x \in S^c = x \notin S$ is a contradiction, so this follows
    from Theorem \ref{empty_eq}.
\end{proof}

\begin{theorem} \label{inter_idemp}
    For all sets $S$, $S \cap S = S$.
\end{theorem}
\begin{proof}
    This follows from predicate extensionality and Theorem \ref{and_idemp}.
\end{proof}

\begin{theorem} \label{union_ldist}
    For all sets $R$, $S$, and $T$, $R \cup (S \cap T) = (R \cup S) \cap (R \cup
    T)$.
\end{theorem}
\begin{proof}
    This follows from predicate extensionality and Theorem \ref{or_and_ldist}.
\end{proof}

\begin{theorem} \label{inter_ldist}
    For all sets $R$, $S$, and $T$, $R \cap (S \cup T) = (R \cap S) \cup (R \cap
    T)$.
\end{theorem}
\begin{proof}
    This follows from predicate extensionality and Theorem \ref{and_or_ldist}.
\end{proof}

\begin{theorem} \label{union_inter_self}
    For all sets $S$ and $T$, $S \cup (S \cap T) = S$.
\end{theorem}
\begin{proof}
    We already have $S \subseteq S \cup (S \cap T)$, so by antisymmetry we only
    need to prove that $S \cup (S \cap T) \subseteq S$.  Choose any $x \in S
    \cup (S \cap T)$.  Then either $x \in S$ or $x \in S$ and $x \in T$.  Either
    way, $x \in S$, so $S \cup (S \cap T) \subseteq S$.
\end{proof}

\begin{theorem} \label{inter_union_self}
    For all sets $S$ and $T$, $S \cap (S \cup T) = S$.
\end{theorem}
\begin{proof}
    We already have $S \cap (S \cup T) \subseteq S$, so by antisymmetry we only
    need to prove that $S \subseteq S \cap (S \cup T)$.  Choose any $x \in S$.
    Then $x \in S \cup T$ as well, so $x \in S \cap (S \cup T)$, showing that $S
    \subseteq S \cap (S \cup T)$.
\end{proof}

\begin{theorem} \label{compl_compl}
    For all sets $S$, $(S^c)^c = S$.
\end{theorem}
\begin{proof}
    This follows from predicate extensionality and Theorem \ref{not_not}.
\end{proof}

\begin{theorem} \label{compl_empty}
    $\varnothing^c = \bm U$.
\end{theorem}
\begin{proof}
    This follows from predicate extensionality and Theorem \ref{not_false}.
\end{proof}

\begin{theorem} \label{compl_all}
    $\bm U^c = \varnothing$.
\end{theorem}
\begin{proof}
    This follows from predicate extensionality and Theorem \ref{not_true}.
\end{proof}

\begin{theorem} \label{union_compl}
    For all sets $A$ and $B$, $(A \cup B)^c = A^c \cap B^c$.
\end{theorem}
\begin{proof}
    This follows from predicate extensionality and Theorem \ref{not_or}.
\end{proof}

\begin{theorem} \label{inter_compl}
    For all sets $A$ and $B$, $(A \cap B)^c = A^c \cup B^c$.
\end{theorem}
\begin{proof}
    This follows from predicate extensionality and Theorem \ref{not_and}.
\end{proof}

\begin{theorem}
    For all sets $S$, $S - \varnothing = S$.
\end{theorem}
\begin{proof}
    \[
        S - \varnothing = S \cap \varnothing^c = S \cap \bm U = S.
    \]
\end{proof}

\begin{theorem}
    For all sets $S$, $\varnothing - S = \varnothing$.
\end{theorem}
\begin{proof}
    \[
        \varnothing - S = \varnothing \cap S^c = \varnothing.
    \]
\end{proof}

\begin{theorem}
    For all sets $S$, $S - S = \varnothing$.
\end{theorem}
\begin{proof}
    \[
        S - S = S \cap S^c = \varnothing.
    \]
\end{proof}

\begin{theorem}
    For all sets $S$ and $T$, $(S - T) - T = S - T$.
\end{theorem}
\begin{proof}
    \[
        (S - T) - T = (S \cap T^c) \cap T^c = S \cap (T^c \cap T^c) = S \cap T^c
        = S - T.
    \]
\end{proof}

\begin{theorem}
    For all sets $S$ and $T$, $S \ominus T = (S \cup T) - (S \cap T)$.
\end{theorem}
\begin{proof}
    \begin{align*}
        S \ominus T
        &= (S - T) \cup (T - S) \\
        &= (S \cap T^c) \cup (T \cap S^c) \\
        &= (S \cup T) \cap (S \cup S^c) \cap (T^c \cup T) \cap (T^c \cup S^c) \\
        &= (S \cup T) \cap \bm U \cap \bm U \cap (T^c \cup S^c) \\
        &= (S \cup T) \cap (T^c \cup S^c) \\
        &= (S \cup T) \cap (T \cap S)^c \\
        &= (S \cup T) \cap (S \cap T)^c \\
        &= (S \cup T) - (S \cap T).
    \end{align*}
\end{proof}

\begin{theorem}
    For all sets $S$ and $T$, $S \ominus T = T \ominus S$.
\end{theorem}
\begin{proof}
    This follows directly from the definition of $\ominus$ with Theorem
    \ref{union_comm}.
\end{proof}

\begin{theorem}
    For all sets $R$, $S$, and $T$, $R \ominus (S \ominus T) = (R \ominus S)
    \ominus T$.
\end{theorem}
\begin{proof}
    \begin{align*}
        &\phantom{=}\ \ R \ominus (S \ominus T) \\
        &= R \cup (S \ominus T) - R \cap (S \ominus T) \\
        &= R \cup (S \cup T - S \cap T) - R \cap ((S - T) \cup (T - S)) \\
        &= (R \cup ((S \cup T) \cap (S \cap T)^c)) \cap
            (R \cap ((S \cap T^c) \cup (T \cap S^c)))^c \\
        &= (R \cup ((S \cup T) \cap (S^c \cup T^c))) \cap
            (R^c \cup ((S^c \cup T) \cap (T^c \cup S))) \\
        &= (R \cup S \cup T) \cap (R \cup S^c \cup T^c) \cap
            (R^c \cup S^c \cup T) \cap (R^c \cup T^c \cup S) \\
        &= (T \cup S \cup R) \cap (T^c \cup S^c \cup R) \cap
            (T \cup S^c \cup R^c) \cap (T^c \cup R^c \cup S) \\
        &= (T \cup S \cup R) \cap (T \cup S^c \cup R^c) \cap
            (T^c \cup S^c \cup R) \cap (T^c \cup R^c \cup S) \\
        &= (T \cup ((S \cup R) \cap (S^c \cup R^c))) \cap
            (T^c \cup ((S^c \cup R) \cap (R^c \cup S))) \\
        &= (T \cup ((S \cup R) \cap (S \cap R)^c)) \cap
            (T \cap ((S \cap R^c) \cup (R \cap S^c)))^c \\
        &= T \cup (S \cup R - S \cap R) - T \cap ((S - R) \cup (R - S)) \\
        &= T \cup (S \ominus R) - T \cap (S \ominus R) \\
        &= T \ominus (S \ominus R) \\
        &= (S \ominus R) \ominus T \\
        &= (R \ominus S) \ominus T.
    \end{align*}
\end{proof}

\begin{theorem}
    For all sets $S$, $\varnothing \ominus S = S$.
\end{theorem}
\begin{proof}
    \[
        \varnothing \ominus S = (\varnothing - S) \cup (S - \varnothing)
        = \varnothing \cup S = S.
    \]
\end{proof}

\begin{theorem}
    For all sets $S$, $S \ominus S = \varnothing$.
\end{theorem}
\begin{proof}
    \[
        (S - S) \cup (S - S) = \varnothing \cup \varnothing = \varnothing.
    \]
\end{proof}

\begin{theorem}
    Let $\U$ and $\V$ be types with sets $A$ and $B$ on $\U$ and $C$ and $D$ on
    $\V$ such that $A \subseteq B$ and $C \subseteq D$.  Then $A \times C
    \subseteq B \times D$.
\end{theorem}
\begin{proof}
    Let $(x, y)$ be a value in $\U \times \V$ such that $(x, y) \in A \times C$.
    This means that $x \in A$ and $y \in C$.  Because $A \subseteq B$ and $C
    \subseteq D$, we have $x \in B$ and $y \in D$.  Thus, $(x, y) \in B \times
    D$, so $A \times C \subseteq B \times D$.
\end{proof}

\begin{theorem}
    Let $\U$ and $\V$ be types with sets $A$ and $B$ on $\U$ and $C$ and $D$ on
    $\V$.  Then
    \[
        (A \cap B) \times (C \cap D) = (A \times C) \cap (B \times D).
    \]
\end{theorem}
\begin{proof}
    For a value $(x, y)$ to be in $(A \cap B) \times (C \cap D)$, we must have
    $x \in A$, $x \in B$, $y \in C$, and $y \in D$.  For $(x, y)$ to be in $(A
    \times C) \cap (B \times D)$, we must have $x \in A$, $y \in C$, $x \in B$,
    and $y \in D$.  Thus, the conditions for being in the two sets are
    equivalent.
\end{proof}

\section{Converting Sets to Types}

Given a set $S$, it can be useful to treat it as a type, not a set.  This can be
done by creating a new dependent type.

\begin{definition}
    Given a set $S$, define the type $\T(S)$ as a dependent inductive type with
    a single constructor $\forall x : \U, x \in S \rightarrow \T(S)$.  Given an
    $x : \U$ with $H : x \in S$, we write this constructor as $\st{x}{H}$.
    Given a value $X = \st{x}{H} : \T(S)$, we define $\stv{X} = x$ and $\stp{X}
    = H$.
\end{definition}

Note that this is technically the same thing as strong existence.  However,
because the two concepts are fundamentally distinct, we will distinguish between
the two types.

\begin{theorem} \label{set_type_simpl}
    For all sets $S$, $X : \T(S)$, and $H : \stv{X} \in S$, $\st{\stv{X}}{H} =
    x$.
\end{theorem}
\begin{proof}
    Let $x = \stv{X}$ and $H' = \stp{X}$.  Then we need to show that $\st{x}{H}
    = \st{x}{H'}$.  This follows from proof irrelevance.
\end{proof}

\begin{theorem} \label{set_type_eq}
    For all sets $S$ and values $a$ and $b$ in $\T(S)$, $a = b$ if and only if
    $\stv{a} = \stv{b}$.
\end{theorem}
\begin{proof}
    When $a = b$, $\stv{a} = \stv{b}$ is trivial, so the forward implication is
    true.  Now assume that $\stv a = \stv b$.  Let $a' = \stv a$, $H_a = \stp
    a$, $b' = \stv b$, and $H_b = \stp b$.  We must prove that $\st{a'}{H_a} =
    \st{b'}{H_b}$.  We already have $a' = b'$, so all that's left is to prove
    that $H_a = H_b$.  Because $a' = b'$, we see that $H_a$ and $H_b$ are the
    same type, so they are equal by proof irrelevance.
\end{proof}

\begin{theorem} \label{set_type_inj}
    The function $x \to \stv[x]$ is injective.
\end{theorem}
\begin{proof}
    This follows directly from the previous theorem.
\end{proof}

In the future, because it matches up with traditional mathematics more, I will
often not mention that we are actually using $\T$.  This may cause a bit of
confusion at times if you're not following along with the proofs carefully.  The
exact uses of $\T$ can be found in the Coq code.

\subsection{Converting sets to/from $\T(X)$}

\begin{definition}
    Let $X$ and $S$ be sets on $\U$.  Then we define a new set $S_X : \T(X)
    \rightarrow \Prop$ with $S_X = \{x : \T(X) \mid \stv{x} \in S\}$.
\end{definition}

\begin{definition}
    Let $X$ and be a set on $\U$, and $S$ a set on $\T(X)$.  Then we define a
    new set $S_\U : \U \rightarrow \Prop$ with $S_\U = \{x : \U \mid x \in X
    \curlywedge \lambda H, \st{x}{H} \in S\}$.
\end{definition}

In the rest of this subsection, let $X$ be a set on $\U$.

\begin{theorem} \label{to_set_type_in}
    For all sets $A$ with $A \subseteq X$, if $x \in A$, then $\st{x}{H} \in
    A_X$, where $H$ is the proof that $x \in X$ using $A \subseteq X$ and $x \in
    A$.
\end{theorem}
\begin{proof}
    We must prove that $x \in A$, which is true by hypothesis.
\end{proof}

\begin{theorem} \label{from_set_type_in}
    For all sets $A$ on $\T(X)$, for all $x \in A$, $\stv x \in A_\U$.
\end{theorem}
\begin{proof}
    Let $x \in A$.  We know that $x : \T(X)$, so $\stv x \in X$.  Call that
    proof $H$.  Because $x \in A$, we know that $\st{\stv x}{H} \in A$, so we
    see that $\stv x \in A_\U$.
\end{proof}

\begin{theorem} \label{to_from_set_type}
    For all sets $A$ on $\T(X)$, $(A_\U)_X = A$.
\end{theorem}
\begin{proof}
    The proof will be by antisymmetry.  Let $x \in (A_\U)_X$.  This means $H :
    \stv x \in X$, and that $\st{\stv x}{H} \in A$.  By Theorem
    \ref{set_type_simpl} we have $\st{\stv x}{H} = x$, so $x \in A$.  Thus,
    $(A_\U)_X \subseteq A$.

    Now let $x \in A$.  To prove that $x \in (A_\U)_X$, it suffices to show that
    $\stv x \in A_\U$.  Because $x \in A$, this follows from Theorem
    \ref{from_set_type_in}.  Thus, $A \subseteq (A_\U)_X$.
\end{proof}

\begin{theorem} \label{from_to_set_type}
    For all sets $A$ on $\U$, if $A \subseteq X$, then $(A_X)_\U = A$.
\end{theorem}
\begin{proof}
    The proof will be by antisymmetry.  Let $x \in (A_X)_\U$.  Then $H : x \in
    X$ and $\st{x}{H} \in A_X$, so $\stv{\st{x}{H}} \in A$, which means that $x
    \in A$.  Thus, $(A_X)_\U \subseteq A$.

    Now let $x \in A$.  Because $A \subseteq X$, $H : x \in X$ as well.  To
    prove that $x \in (A_X)_\U$, we just need to show that $\st{x}{H} \in A_X$.
    We just need to prove that $\stv{\st{x}{H}} \in A$, which is true because $x
    \in A$.  Thus, $A \subseteq (A_X)_\U$.
\end{proof}

\begin{theorem} \label{to_set_type_inter}
    For all sets $A$ on $\U$, $A_X = (A \cap X)_X$.
\end{theorem}
\begin{proof}
    The proof will be by antisymmetry.  Let $x \in A_X$.  Then $\stv x \in X$
    and $\stv x \in A$, so $\stv x \in (A \cap X)$.  Because $\stv x \in (A \cap
    X)$ and $\stv x \in X$, $x \in (A \cap X)_X$, so $A_X \subseteq (A \cap
    X)_X$.

    Now let $x \in (A \cap X)_X$.  Then $\stv x \in A \cap X \subseteq A$ and
    $\stv x \in X$, so $x \in A_X$.
\end{proof}

\begin{theorem} \label{to_set_type_sub}
    For all sets $A$ and $B$ on $\U$, if $A \subseteq B$, then $A_X \subseteq
    B_X$.
\end{theorem}
\begin{proof}
    Let $x \in A_X$.  Then $\stv x \in A$.  Because $A \subseteq B$, we have
    $\stv x \in B$.  Thus, $x \in B_X$, showing that $A_X \subseteq B_X$.
\end{proof}

\begin{theorem} \label{to_from_set_type_sub}
    For all sets $A$ on $\U$ and $B$ on $\T(X)$, if $A \subseteq X$ and $A_X
    \subseteq B$, then $A \subseteq B_\U$.
\end{theorem}
\begin{proof}
    Let $x \in A$.  Then because $A \subseteq X$, we have $x \in X$.  Call this
    proof $H$.  Then $\st{x}{H} \in A_X$.  Because $A_X \subseteq B$, we have
    $\st{x}{H} \in B$ as well.  Because $x \in X$ and $\st{x}{H} \in B$, we have
    $x \in B_\U$, so $A \subseteq B_\U$.
\end{proof}

\begin{theorem} \label{from_set_type_sub_X}
    For all sets $A$ on $\T(X)$, $A_\U \subseteq X$.
\end{theorem}
\begin{proof}
    Let $x \in A_\U$.  Then $H : x \in X$ and $\st{x}{H} \in A$, so $x \in X$.
    Thus, $A_\U \subseteq X$.
\end{proof}

\begin{theorem} \label{from_set_type_union}
    For all sets $A$ and $B$ on $\T(X)$, if $A \cup B = \bm U\footnote{Note that
    this is the universal set \textit{in $\T(X)$}, not in $\U$}$, then $A_\U
    \cup B_\U = X$.
\end{theorem}
\begin{proof}
    The proof will be by antisymmetry.  If $x \in A_\U \cup B_\U$, there will be
    two cases.  However, both cases instantly get $x \in X$, so $A_\U \cup B_\U
    \subseteq X$.  Now let $x$ be a value such that we have a proof $H : x \in
    X$.  Now $\st{x}{H} \in \bm U$, and because $A \cup B = \bm U$, we have
    $\st{x}{H} \in A \cup B$.  Now when $\st{x}{H} \in A$, because $x \in X$ we
    have $x \in A_\U$.  Similarly, when $\st{x}{H} \in B$, $x \in B_\U$.  Either
    way, $x \in A_\U \cup B_\U$.  Thus, $X \subseteq A_\U \cup B_\U$.
\end{proof}

\subsection{Ordering $\T(S)$}

If $\U$ is ordered, it is possible to order $\T(S)$ for all sets $S$ as follows:

\begin{definition}
    Let $a$ and $b$ be values in $\T(S)$.  Then define $a \leq b$ to mean that
    $\stv a \leq \stv b$.
\end{definition}

From this definition, we instantly see that the order in $\T(S)$ will inherit
reflexivity, antisymmetry, transitivity, and connexivity from $\U$.  The only
thing that doesn't immediately follow is if $\T(S)$ is well-ordered whenever
$\U$ is.

\begin{theorem}
    If $\U$ is well-ordered, then $\T(S)$ is as well.
\end{theorem}
\begin{proof}
    Let $T$ be a nonempty set on $\T(S)$.  Then $T_\U$ is nonempty as well, so
    it has a least element $x \in T_\U$.  Because $x \in T_\U$, we have $H : x
    \in S$ and $\st{x}{H} \in T$.  We will prove that $\st{x}{H}$ is the minimum
    value in $T$.  For all $y \in T$, we have $\stv y \in T_\U$, so because $x$
    is the minimum element of $T_\U$, we have $x \leq \stv y$.  This means that
    $\stv{\st{x}{H}} \leq \stv y$, which is the definition of $\st{x}{H} \leq
    y$.  Thus, $\st{x}{H}$ is the minimum element of $T$, so $\T(S)$ is
    well-ordered.
\end{proof}

\section{Functions To/From Sets}

Throughout this section, let $\A$ and $\B$ be types and let $f$ be a function
from $\A$ to $\B$.

\begin{definition}
    We define the image of $f$ as the set $\{y : \B \mid \exists x : \A, y =
    f(x)\}$.
\end{definition}

\begin{definition}
    Given a set $S : \A \rightarrow \Prop$, we define the image $f(S)$ as the
    set $f(S) = \{y : \B \mid \exists x : \A, x \in S \wedge y = f(x)\}$.
\end{definition}

\begin{definition}
    Given a set $T : \B \rightarrow \Prop$, we define the inverse image
    $f^{-1}(T)$ as the set $f^{-1}(T) = \{x : \A \mid f(x) \in T\}$.
\end{definition}

\begin{theorem} \label{image_under_in}
    For all sets $S : \A \rightarrow \Prop$ and $x \in S$, $f(x) \in f(S)$.
\end{theorem}
\begin{proof}
    We know that $x \in S$ and $f(x) = f(x)$, so $f(x) \in f(S)$.
\end{proof}

\begin{theorem} \label{image_inverse_sub}
    For all sets $T : \B \rightarrow \Prop$, $f(f^{-1}(T)) \subseteq T$.
\end{theorem}
\begin{proof}
    Let $y \in f(f^{-1}(T))$.  This means that there is some $x \in f^{-1}(T)$
    such that $y = f(x)$.  Because $x \in f^{-1}(T)$, we have $f(x) = y \in T$.
    Thus, $f(f^{-1}(T)) \subseteq T$.
\end{proof}

\begin{theorem} \label{image_sub}
    If $S$ and $T$ are sets on $\A$ such that $S \subseteq T$, then $f(S)
    \subseteq f(T)$.
\end{theorem}
\begin{proof}
    Let $y \in f(S)$.  This means there exists an $x \in S$ such that $y =
    f(x)$.  Because $S \subseteq T$, we have $x \in T$, so we now have an $x \in
    T$ such that $y = f(x)$.  This means that $y \in f(T)$, so $f(S) \subseteq
    f(T)$.
\end{proof}

\begin{theorem} \label{inverse_complement}
    For all sets $T : \B \rightarrow \Prop$, $f^{-1}(T^c) = f^{-1}(T)^c$.
\end{theorem}
\begin{proof}
    The set $f^{-1}(T^c)$ is all elements $x : \U$ such that $f(x) \notin T$,
    while the set $f^{-1}(T)^c$ is all elements $x : \U$ such that $x \notin
    f^{-1}(T)$, which is the same as saying that $f(x) \notin T$.  Thus, the
    definition of the two sets is equivalent.
\end{proof}

\begin{theorem} \label{inverse_image_bij_inv}
    If $f$ is bijective with an inverse $g : \B \rightarrow \A$, then for all
    sets $S : \U \rightarrow \Prop$, $g^{-1}(S) = f(S)$.
\end{theorem}
\begin{proof}
    The proof will be by antisymmetry.  Let $y \in g^{-1}(S)$.  This means that
    $g(y) \in S$.  Because $g$ is the inverse of $f$, we have $y = f(g(y))$.
    Because $g(y) \in S$ and $y = f(g(y))$, we have $y \in f(S)$ as required.

    Now let $y \in f(S)$.  This means that there is an $x \in S$ such that $y =
    f(x)$.  Because $g(f(x)) = x$, we have $g(f(x)) = g(y) \in S$, which means
    that $y \in g^{-1}(S)$.
\end{proof}

\begin{theorem} \label{bij_inverse_image}
    If $f$ is bijective, then for all sets $T : \B \rightarrow \Prop$,
    $f(f^{-1}(T)) = T$.
\end{theorem}
\begin{proof}
    The proof will be by antisymmetry.  $f(f^{-1}(T)) \subseteq T$ was already
    proved in Theorem \ref{image_inverse_sub}, so we only need to prove that
    $T \subseteq f(f^{-1}(T))$.  Let $y \in T$.  Because $f$ is bijective, it
    has an inverse $g : \B \rightarrow \A$.  Then $g(y) \in f^{-1}(T)$ because
    $f(g(y)) = y \in T$.  Because $g(y) \in f^{-1}(T)$ and $y = f(g(y))$, we
    have $y \in f(f^{-1}(T))$ as required.
\end{proof}

\begin{theorem} \label{inj_inverse_image}
    If $f$ is injective, then for all sets $S : \A \rightarrow \Prop$,
    $f^{-1}(f(S)) = S$.
\end{theorem}
\begin{proof}
    The proof will be by antisymmetry.  Let $x \in f^{-1}(f(S))$.  This means
    that $f(x) \in f(S)$, so there exists a $y \in S$ such that $f(x) = f(y)$.
    Because $f$ is injective, $x = y$, so $x \in S$ as required.

    Now let $x \in S$.  We need to prove that $x \in f^{-1}(f(S))$ which is
    equivalent to showing that $f(x) \in f(S)$.  This follows from Theorem
    \ref{image_under_in}.
\end{proof}

\section{Sets of Sets}

A set of sets is simply a value in a type $(\U \rightarrow \Prop) \rightarrow
\Prop$.  We will call a set of sets a collection.  There are several basic
definitions using collections that we can make.

\begin{definition}
    Given a collection $\mathcal S : (\U \rightarrow \Prop) \rightarrow \Prop$,
    we define its union $\bigcup \mathcal S$ as the set
    \[
        \bigcup \mathcal S =
            \{x : \U \mid \exists A, A \in \mathcal S \wedge x \in A\}.
    \]
\end{definition}

\begin{definition}
    Given a collection $\mathcal S : (\U \rightarrow \Prop) \rightarrow \Prop$,
    we define its intersection $\bigcap \mathcal S$ as the set
    \[
        \bigcap \mathcal S =
            \{x : \U \mid \forall A, A \in \mathcal S \rightarrow x \in A\}.
    \]
\end{definition}

\begin{definition}
    Given a collection $\mathcal S : (\A \rightarrow \Prop) \rightarrow \Prop$
    and a function $f : \A \rightarrow \B$, we define the image $f(\mathcal S)$
    as the set
    \[
        f(\mathcal S) =
            \{Y : \B \rightarrow \Prop \mid
            \exists X, X \in \mathcal S \wedge Y = f(X)\}.
    \]
\end{definition}

\begin{definition}
    Given a collection $\mathcal T : (\B \rightarrow \Prop) \rightarrow \Prop$
    and a function $f : \A \rightarrow \B$, the define the inverse image
    $f^{-1}(\mathcal T)$ as the set
    \[
        f^{-1}(\mathcal T) =
            \{X : \A \rightarrow \Prop \mid
            \exists Y, Y \in \mathcal T \wedge X = f^{-1}(Y)\}.
    \]
\end{definition}

Of course, for finite collections we want $\bigcup\{A_1, A_2, \cdots, A_n\} =
A_1 \cup A_2 \cup \cdots \cup A_n$, and likewise for intersections.  This can be
done through the next several theorems.

\begin{theorem}
    $\bigcup \Empty = \Empty$.
\end{theorem}
\begin{proof}
    If $x \in \bigcup \Empty$, that would mean that there exists an $A \in
    \Empty$ such that $x \in A$.  But $A \in \Empty$ is impossible, so there
    cannot be any $x \in \bigcup \Empty$.  Thus $\bigcup \Empty = \Empty$.
\end{proof}

\begin{theorem}
    $\bigcap \Empty = \bm U$.
\end{theorem}
\begin{proof}
    We need to prove that $x \in \bigcap \Empty$ for all $x$.  To do so, we need
    to prove that for all $A \in \Empty$, $x \in A$.  Because we can't have $A
    \in \Empty$, this is vacuously true.
\end{proof}

\begin{theorem}
    For all sets $A : \U \rightarrow \Prop$ and collections $\mathcal S : (\U
    \rightarrow \Prop) \rightarrow \Prop$,
    \[
        \bigcup (\{A\} \cup \mathcal S) = A \cup \bigcup \mathcal S.
    \]
\end{theorem}
\begin{proof}
    The proof will be by antisymmetry.  Let $x \in \bigcup (\{A\} \cup \mathcal
    S)$.  This means that there exists a $B \in \{A\} \cup \mathcal S$ such that
    $x \in B$.  So we now have two cases: when $B = A$, and when $B \in \mathcal
    S$.  When $B = A$, we have $x \in A \subseteq A \cup \bigcup \mathcal S$.
    When $B \in \mathcal S$, because $x \in B$ we have $x \in \bigcup \mathcal S
    \subseteq A \cup \bigcup \mathcal S$.  Either way, $x \in A \cup \bigcup
    \mathcal S$, so $\bigcup (\{A\} \cup \mathcal S) \subseteq A \cup \bigcup
    \mathcal S$.

    Now let $x \in A \cup \bigcup \mathcal S$.  We have two cases: when $x \in
    A$, and when $x \in \bigcup \mathcal S$.  When $x \in A$, we also have $A
    \in \{A\} \subseteq \{A\} \cup \mathcal S$, so $x \in \bigcup(\{A\} \cup
    \mathcal S)$.  When $x \in \bigcup \mathcal S$, we know that there exists
    some $B \in \mathcal S$ such that $x \in B$.  Because $B \in \mathcal S$, we
    have $B \in \{A\} \cup \mathcal S$, so $x \in \bigcup(\{A\} \cup \mathcal
    S)$.  Either way, $x \in \bigcup(\{A\} \cup \mathcal S)$, so $A \cup \bigcup
    \mathcal S \subseteq \bigcup(\{A\} \cup \mathcal S)$.
\end{proof}

\begin{theorem}
    For all sets $A : \U \rightarrow \Prop$ and collections $\mathcal S : (\U
    \rightarrow \Prop) \rightarrow \Prop$,
    \[
        \bigcap (\{A\} \cup \mathcal S) = A \cap \bigcap \mathcal S.
    \]
\end{theorem}
\begin{proof}
    The proof will be by antisymmetry.  Let $x \in \bigcap (\{A\} \cup \mathcal
    S)$.  This means that for any $B \in \{A\} \cup \mathcal S$, $x \in B$.
    Because $A \in \{A\} \cup \mathcal S$, we have $x \in A$.  Also, every $B
    \in \mathcal S$ is also in $\{A\} \cup \mathcal S$, so we have $x \in
    \bigcap \mathcal S$ as well.  Together, this means that $x \in A \cap
    \bigcap \mathcal S$, so $\bigcap (\{A\} \cup \mathcal S) \subseteq A \cap
    \bigcap \mathcal S$.

    Now let $x \in A \cap \bigcap \mathcal S$.  This means that $x \in A$ and $x
    \in \bigcap \mathcal S$.  Because $x \in \bigcap \mathcal S$, $x \in B$ for
    all $B \in \mathcal S$.  Now for any $B \in \{A\} \cup \mathcal S$, either
    $B = A$ or $B \in \mathcal S$.  If $B = A$, then we have $x \in B$.  If $B
    \in \mathcal S$, then we also have $x \in B$.  Either way, $x \in B$, which
    means that $x \in \bigcap (\{A\} \cup \mathcal S)$, so $A \cap \bigcap
    \mathcal S \subseteq \bigcap (\{A\} \cup \mathcal S)$.
\end{proof}

The previous four theorems provide an induction scheme that allows us to easily
prove that
\[
    \bigcup \{A_1, A_2, \cdots, A_n\} = A_1 \cup A_2 \cup \cdots \cup A_n
\]
and
\[
    \bigcap \{A_1, A_2, \cdots, A_n\} = A_1 \cap A_2 \cap \cdots \cap A_n.
\]

\begin{theorem}
    For all collections $\mathcal S : \col \U$, defining the set
    \[
        \mathcal S' = \{A : \set U \mid A^c \in \mathcal S\},
    \]
    we have
    \[
        \left( \bigcup \mathcal S \right)^c = \bigcap \mathcal S'.
    \]
\end{theorem}
\begin{proof}
    The proof will be by antisymmetry.  Let $x \in (\bigcup \mathcal S)^c$.  To
    prove that $x \in \bigcap \mathcal S'$, we must prove that for all $A$ such
    that $A^c \in \mathcal S$, we have $x \in A$.  To do this, assume that $x
    \notin A$.  This means that $x \in A^c$.  We now have $x \in A^c$ and $A^c
    \in \mathcal S$, so $x \in \bigcup \mathcal S$.  However, we know that $x
    \in (\bigcup \mathcal S)^c$, so this is a contradiction.  Thus, $x \in A$,
    so $x \in \bigcap \mathcal S'$, meaning that $(\bigcup \mathcal S)^c
    \subseteq \bigcap \mathcal S'$.

    Now let $x \in \bigcap \mathcal S'$.  If $x$ is in fact in $\bigcup \mathcal
    S$, there would exist some $A \in \mathcal S$ such that $x \in A$.  Because
    $A \in \mathcal S$, we have $A^c \in \mathcal S'$, and because $x \in
    \bigcap \mathcal S'$, we have $x \in A^c$.  We now have both $x \in A$ and
    $x \in A^c$, which is a contradiction.  Thus, $x \notin \bigcup \mathcal S$,
    so we have $\bigcap \mathcal S' \subseteq (\bigcup \mathcal S)^c$.
\end{proof}

\begin{theorem}
    For all collections $\mathcal S : \col \U$, defining the set
    \[
        \mathcal S' = \{A : \set U \mid A^c \in \mathcal S\},
    \]
    we have
    \[
        \left( \bigcap \mathcal S \right)^c = \bigcup \mathcal S'.
    \]
\end{theorem}
\begin{proof}
    The proof will be by antisymmetry.  Let $x \in (\bigcap \mathcal S)^c$.
    This means that there exists some $A \in \mathcal S$ such that $x \notin A$.
    We thus have $A^c \in \mathcal S'$ and $x \in A^c$, which shows that $x \in
    \bigcup \mathcal S'$.  Thus, $(\bigcap \mathcal S)^c \subseteq \bigcup
    \mathcal S'$.

    Now let $x \in \bigcup \mathcal S'$.  This means that there exists an $A$
    such that $A \in S'$ and $x \in A$.  From $A \in S'$ we have $A^c \in S$.
    Now if $x \in \bigcap \mathcal S$, because $A^c \in S$, we would have $x \in
    A^c$, which contradicts $x \in A$.  Thus $x \notin \bigcap \mathcal S$, so
    $x \in (\bigcap \mathcal S)^c$.  Thus, $\bigcup \mathcal S' \subseteq
    (\bigcap \mathcal S)^c$.
\end{proof}

\begin{theorem} \label{inverse_union}
    For all functions $f : \A \rightarrow \B$ and collections $\mathcal T : \col
    \B$,
    \[
        f^{-1} \left( \bigcup \mathcal T \right) = \bigcup f^{-1}(\mathcal T).
    \]
\end{theorem}
\begin{proof}
    The proof will be by antisymmetry.  Let $x \in f^{-1}(\bigcup \mathcal T)$.
    This means that $f(x) \in \bigcup \mathcal T$, so there exists an $A \in
    \mathcal T$ such that $f(x) \in A$.  Because $A \in \mathcal T$, we have
    $f^{-1}(A) \in f^{-1}(\mathcal T)$.  Because $f(x) \in A$, we have $x \in
    f^{-1}(A)$.  Because $f^{-1}(A) \in f^{-1}(\mathcal T)$ and $x \in
    f^{-1}(A)$, we have $x \in \bigcup(f^{-1}\mathcal T)$, so $f^{-1}(\bigcup
    \mathcal T) \subseteq \bigcup f^{-1}(\mathcal T)$.

    Now let $x \in \bigcup f^{-1}(\mathcal T)$.  This means that there exists an
    $A \in f^{-1}(\mathcal T)$ such that $x \in A$.  Because $A \in
    f^{-1}(\mathcal T)$, there exists some $Y \in \mathcal T$ such that $A =
    f^{-1}(Y)$.  We thus have $x \in f^{-1}(Y)$, so $f(x) \in Y$.  Because $f(x)
    \in Y$ and $Y \in \mathcal T$, we have $f(x) \in \bigcup \mathcal T$, so $x
    \in f^{-1}(\bigcup \mathcal T)$.  Thus, $\bigcup f^{-1}(\mathcal T)
    \subseteq f^{-1}(\bigcup \mathcal T)$.
\end{proof}

\section{Quotients}

\begin{definition}
    A relation $\sim$ is an equivalence relation if it is reflexive, symmetric,
    and transitive.
\end{definition}

Throughout the rest of this section, let $\A$ be a type and $\sim$ an
equivalence relation on $A$.

\begin{definition}
    For some $a : \A$, we define the equivalence class set of $a$ as the set
    $[[a]] = \{x : \A \mid a \sim x\}$.
\end{definition}

\begin{definition}
    Define a set of sets $\A_\sim = \{S : \A \to \Prop \mid \exists a : \A,
    [[a]] = S\}$.
\end{definition}

\begin{definition}
    Define $\A/\usim = \T(\A_\sim)$.  The elements of $\A/{\sim}$ are called
    equivalence classes.
\end{definition}
Note that unlike in usual theories, what we call equivalence classes are not
quite sets.

\begin{definition}
    For any $a : \A$, $[ [a]]$ is trivially in $\A_\sim$.  Call this proof $H$.
    Then we define $[a] = \st{[ [a]]}{H}$.
\end{definition}

\begin{theorem} \label{equiv_eq}
    For all $a, b : \A$, $[a] = [b] \leftrightarrow a \sim b$.
\end{theorem}
\begin{proof}
    By Theorem \ref{set_type_eq}, we only need to show that $[ [a]] = [ [b]]
    \leftrightarrow a \sim b$.

    For the forward implication, we must prove that $[ [a]] = [ [b]] \to a \sim
    b$.  Because $[ [a]] = [ [b]]$, we know that for all $x : \A$, $(a \sim x) =
    (b \sim x)$.  Using $x = b$, we see that $(a \sim b) = (b \sim b)$.  $b \sim
    b$ because equivalence relations are reflexive, so $a \sim b$.

    For the reverse implication, we must prove that $a \sim b \to [ [a]] = [
    [b]]$.  When $a \sim b$, by the symmetry of equivalence relations we also
    have $b \sim a$.  By predicate extensionality, we need to show that for all
    $x : \U$, $a \sim x \leftrightarrow b \sim x$.  When $a \sim x$, because $b
    \sim a$, by the transitivity of equivalence relations we have $b \sim x$ as
    required.  When $b \sim x$, because $a \sim b$, by the transitivity of
    equivalence relations we again have $a \sim x$ as required.  Thus, when $a
    \sim b$, $a \sim x \leftrightarrow b \sim x$, so $a \sim b \to [ [a]] = [
    [b]]$.
\end{proof}

\begin{theorem}
    For all $X : \A/\usim$, there exists an $x : \A$ such that $[x] = X$.
\end{theorem}
\begin{proof}
    By $[|X]$, there exists an $x : \A$ such that $[ [x]] = [X|]$.  To prove
    $[x] = X$, by Theorem \ref{set_type_eq}, it suffices to prove $\stv{[x]} =
    \stv{X}$.  We have $\stv{[x]} = [ [x]]$ by definition, and we already know
    that $[ [x]] = [X|]$.
\end{proof}

\begin{definition}
    Given an $X : \A/\usim$, define $\{X\}$ to be a value in $\A$ as given by
    the previous theorem.
\end{definition}

\begin{theorem} \label{from_equiv_eq}
    For all $X : \A/\usim$, $[\{X\}] = X$.
\end{theorem}
\begin{proof}
    This is true by definition.
\end{proof}

\begin{theorem} \label{unary_op_ex}
    For all types $\B$ and functions $f : \A \to \B$, if $a \sim b \to f(a) =
    f(b)$ for all $a$ and $b$ in $\A$, then there exists a function $f' :
    \A/\usim \to \B$ such that for all $x : \A$, we have $f'([x]) = f(x)$.
\end{theorem}
\begin{proof}
    Define the function $f' : \A/\usim \to \B$ given by $f'(X) = f(\{X\})$.  We
    must prove that for all $x : \A$, $f(\{[x]\}) = f(x)$.  Because $f$ is
    well-defined, it suffices to prove that $\{[x]\} \sim x$.  By Theorem
    \ref{equiv_eq} this is equivalent to $[\{[x]\}] = [x]$, which is true by
    Theorem \ref{from_equiv_eq}.
\end{proof}

At this point we can forget about the construction of $\A/\usim$ and just use
the previous few theorems.  More complicated functions can be created through
clever applications of Theorem \ref{unary_op_ex}.  Binary functions are slightly
less trivial:

\begin{theorem} \label{binary_op_ex}
    For all types $\B$ and functions $f : \A \to \A \to \B$, if $a \sim b \to c
    \sim d \to f(a, c) = f(b, d)$ for all $a$, $b$, $c$, and $d$ in $\A$, then
    there exists a function $f' : \A/\usim \to \A/\usim \to \B$ such that for
    all $x$ and $y$ in $\A$, we have $f'([x], [y]) = f(x, y)$.
\end{theorem}
\begin{proof}
    First fix a value $a : \A$.  Now consider the function $f_a : \A \to \B$
    given by $f_a(b) = f(a, b)$.  Because $f$ is well-defined, $f_a$ is
    well-defined as an unary function, so by Theorem \ref{unary_op_ex} there
    exists a function $f_a' : \A/\usim \to \B$ such that for all $x : \A$,
    $f_a'([x]) = f_a(x) = f(a, x)$.

    Now let $f_1 : \A \to (\A/\usim \to \B)$ be the function given by $f_1(a) =
    f_a'$.  We want to prove that this is well-defined.  Given $a$ and $b$ such
    that $a \sim b$, we must prove that $f_1(a) = f_1(b)$.  By functional
    extensionality, we must prove that for all $C : \A/\usim$, we have
    $f_1(a)(C) = f_1(b)(C)$.  Choose a representative $c$ such that $[c] = C$.
    Then
    \[
        f_1(a)(C) = f_a'([c]) = f(a, c)
    \]
    and
    \[
        f_1(b)(C) = f_b'([c]) = f(b, c),
    \]
    so we must prove that $f(a, c) = f(b, c)$.  This follows from $a \sim b$ and
    the fact that $f$ is well-defined.

    Because $f_1$ is well-defined, there exists a function $f' : \A/\usim \to
    (\A/\usim \to \B)$ such that $f'([a]) = f_1(a)$.  Thinking of it as a
    function $\A/\usim \to \A/\usim \to \B$, this is the function required:
    Consider values $x$ and $y$ in $\A$.  Then
    \[
        f'([x])([y]) = f_1(x)([y]) = f_x([y]) = f(x, y).
    \]
\end{proof}

To convert a function $f : \A \to \A$ to a function $f' : \A/\usim \to
\A/\usim$, just apply Theorem \ref{unary_op_ex} to the function $f'' : \A \to
\A/\usim$ given by $f''(x) = [f(x)]$.  The condition for being well-defined
simplifies to checking that $a \sim b \to f(a) \sim f(b)$.  A similar process
works for binary functions as well.

To convert a function $f : \A \to \B \to \C$ to a function $f' : \A/\usim \to \B
\to \C$, one can fix a $b : \B$ and apply Theorem \ref{unary_op_ex} to the
function $f_b : \A \to \C$ given by $f_b(a) = f(a, b)$.

\section{Sets and Order}

In all of the following definitions, let $\leq$ be an arbitrary relation and let
$S$ be a set on $\U$.

\begin{definition}
    A value $x$ is the least element of $S$ if $x \in S$ and for all $y \in S$,
    $x \leq y$.
\end{definition}

\begin{definition}
    A value $x$ is the greatest element of $S$ if $x \in S$ and for all $y \in
    S$, $y \leq x$.
\end{definition}

\begin{definition}
    A value $x$ is a minimal element of $S$ if $x \in S$ and for all $y \in S$,
    $y \nless x$.
\end{definition}

\begin{definition}
    A value $x$ is a maximal element of $S$ if $x \in S$ and for all $y \in S$,
    $x \nless y$.
\end{definition}

\begin{definition}
    Given values $a$ and $b$, we define several intervals:
    \begin{itemize}
        \item $(a, b) = \{x : \U \mid a < x \wedge x < b\}$
        \item $[a, b) = \{x : \U \mid a \leq x \wedge x < b\}$
        \item $(a, b] = \{x : \U \mid a < x \wedge x \leq b\}$
        \item $[a, b] = \{x : \U \mid a \leq x \wedge x \leq b\}$
        \item $(a, \infty) = \{x : \U \mid a < x\}$
        \item $[a, \infty) = \{x : \U \mid a \leq x\}$
        \item $(-\infty, b) = \{x : \U \mid x < b\}$
        \item $(-\infty, b] = \{x : \U \mid x \leq b\}$
    \end{itemize}
\end{definition}

\begin{definition}
    In a partially ordered type, a set $S$ is a chain if for all $a$ and $b$ in
    $S$, $a \leq b$ or $b \leq a$.
\end{definition}

\begin{definition}
    In a partially ordered type, a set $S$ is well-ordered if for all nonempty
    subsets $A \subseteq S$, there exists an $a \in A$ such that for all $b \in
    A$, $a \leq b$.
\end{definition}

\begin{definition}
    In an ordered type $\U$, the initial segment $\U_x$ of a value $x : \U$ is
    the set $\{a : \U \mid a < x\}$.
\end{definition}

\begin{definition}
    Given a natural number $n$, we define the type $\T(n)$ as $\T(\N_n)$.
\end{definition}

\begin{theorem}
    In a partially ordered type, a subset of a chain is a chain.
\end{theorem}
\begin{proof}
    Let $S$ be a chain, and $T$ a subset of $S$.  For any two elements $a$ and
    $b$ in $T$, because $T \subseteq S$, we know that $a$ and $b$ are in $S$ as
    well.  Because $S$ is a chain, $a$ and $b$ are comparable, so $T$ is a
    chain.
\end{proof}

\begin{theorem}
    In a partially ordered type, a subset of a well-ordered set is well-ordered.
\end{theorem}
\begin{proof}
    Let $S$ be a well-ordered set, and $T$ a subset of $S$.  Let $A$ be a
    nonempty subset of $T$.  Because $A \subseteq T$ and $T \subseteq S$, we
    have $A \subseteq S$ by transitivity.  Because $A$ is a nonempty subset of
    $S$, it has a least element.  Thus, $T$ is well-ordered.
\end{proof}

\begin{theorem}
    In a partially ordered type, a well-ordered set is a chain.
\end{theorem}
\begin{proof}
    Let $S$ be a well-ordered set, and let $a$ and $b$ be two elements of $S$.
    Because $a \in S$ and $b \in S$, $\{a, b\} \subseteq S$.  Furthermore,
    because $a \in \{a, b\}$, $\{a, b\}$ is nonempty.  Thus, because $S$ is
    well-ordered, $\{a, b\}$ has a least element.  If it is $a$, then $a \leq
    b$, and if it is $b$, then $b \leq a$.
\end{proof}

\section{Finite Types}

While a full discussion of infinite and finite types requires developing the
theory of the cardinals, some facts about finite types can be proved directly
here.  In the Coq code, this also has the practical benefit that using the ideas
developed here allows parts of the code to not depend on the cardinals, allowing
for more parallelization when compiling it.  Thus, as in the Coq code, we will
distinguish between the concept of finiteness developed here and the concept of
finiteness in the cardinals, even though they are equivalent.

\begin{definition}
    Let $\U$ be a type.  Then we say that $\U$ is simply finite if there exists
    a natural number $n$ and an injective function $f : \U \to \T(n)$.
\end{definition}

\begin{theorem} \label{simple_finite_trans}
    For all types $\U$ and $\V$, if $\V$ is simply finite and there exists an
    injective function from $\U$ to $\V$, then $\U$ is simply finite.
\end{theorem}
\begin{proof}
    Let $f : \U \to \V$ be injective.  Because $\V$ is simply finite, there
    exists a natural number $n$ and an injective function $g : \V \to \T(n)$.
    Then by Theorem \ref{inj_comp}, $g \circ f$ is an injective function from
    $\U$ to $\T(n)$, so $\U$ is simply finite.
\end{proof}

\begin{theorem} \label{simple_finite_nat}
    For all $n : \N$, $\T(n)$ is simply finite.
\end{theorem}
\begin{proof}
    The identity function is injective.
\end{proof}

\begin{theorem} \label{simple_finite_sum}
    If $\U$ and $\V$ are simply finite types, then $\U + \V$ is as well.
\end{theorem}
\begin{proof}
    Let $m$ and $n$ be natural numbers such that there are injective functions
    $f : \U \to \T(m)$ and $g : \V \to \T(n)$.  Now because $f(x) < m$ for all
    $x : \U$, we also have $f(x) < m + n$ for all $x$.  Furthermore, because
    $g(x) < n$ for all $x : \V$, we have $m + g(x) < m + n$.  With this we can
    define a function $h$ from $\U + \V$ to $\T(m + n)$:
    \[
        h(x) =
        \begin{cases}
            f(x) \quad &\text{if $x : \A$} \\
            m + g(x) \quad &\text{if $x : \B$.}
        \end{cases}
    \]
    It remains to prove that $h$ is injective.  Let $a$ and $b$ be values in $\U
    + \V$ such that $h(a) = h(b)$.  We must prove that $a = b$.  There are four
    cases, depending on when $a$ and $b$ are in $\U$ or $\V$.
    \begin{case} $a : \U$ and $b : \U$. \\
        Here we have $f(a) = f(b)$.  Because $f$ is injective, we have $a = b$
        as required.
    \end{case}
    \begin{case} $a : \U$ and $b : \V$. \\
        Here we have $f(a) = m + g(b)$.  But $f(a) < m$, meaning that $m + g(b)
        < m$.  Thus, this case is impossible.
    \end{case}
    \begin{case} $a : \V$ and $b : \U$. \\
        Here we have $m + g(a) = f(b)$.  But $f(b) < m$, meaning that $m + g(a)
        < m$.  Thus, this case is impossible.
    \end{case}
    \begin{case} $a : \U$ and $b : \U$. \\
        Here we have $m + g(a) = m + g(b)$.  By canceling $m$ we get $g(a) =
        g(b)$.  Because $g$ is injective, we have $a = b$ as required.
    \end{case}
\end{proof}

\begin{theorem} \label{simple_finite_prod}
    If $\U$ and $\V$ are simply finite types, then $\U \times \V$ is as well.
\end{theorem}
\begin{proof}
    Let $m$ and $n$ be natural numbers such that there are injective functions
    $f : \U \to \T(m)$ and $g : \V \to \T(n)$.  We know that for all $a : \U$
    and $b : \V$, $f(a) < m$ and $g(b) < n$.  By Theorem \ref{nat_le_suc_lt},
    $S(f(a)) \leq m$.  We can multiply this inequality by $n$ to get $nS(f(a)) =
    nf(a) + n \leq mn$.  Because $g(b) < n$, we have $nf(a) + g(b) < mn$.  Thus,
    we can define a function $h : \U \times \V \to \T(mn)$ with $h(a, b) = nf(a)
    + g(b)$.  It remains to prove that $h$ is injective.

    Let $(a_1, a_2)$ and $(b_1, b_2)$ be such that $h(a_1, a_2) = h(b_1, b_2)$.
    We must prove that $a_1 = b_1$ and $a_2 = b_2$.  Because $h(a_1, a_2) =
    h(b_1, b_2)$ we have
    \[
        nf(a_1) + g(a_2) = nf(b_1) + g(b_2).
    \]
    For the sake of a contradiction, assume that $f(a_1) \neq f(b_1)$.  Without
    loss of generality we can assume that $f(a_1) < f(b_1)$.  By Theorem
    \ref{nat_lt_ex}, there exists some $c$ such that $f(a_1) + S(c) = f(b_1)$.
    Then we have
    \[
        nf(a_1) + g(a_2) = nf(a_1) + nS(c) + g(b_2).
    \]
    Canceling $nf(a_1)$, we get
    \[
        g(a_2) = nS(c) + g(b_2).
    \]
    But $g(a_2) < n$, so
    \[
        nS(c) + g(b_2) = n + nc + g(b_2) < n,
    \]
    which is a contradiction.  Thus, $f(a_1) = f(b_1)$.

    From here the rest of the proof is relatively straightforward.  Because
    $f(a_1) = f(b_1)$ and $f$ is injective, $a_1 = b_1$.  Furthermore, we now
    have
    \[
        nf(a_1) + g(a_2) = nf(a_1) + g(b_2),
    \]
    so by canceling we get
    \[
        g(a_2) = g(b_2).
    \]
    Again, $g$ is injective, so $a_2 = b_2$.  We have shown that $a_1 = b_1$ and
    $a_2 = b_2$, so $h$ is injective.
\end{proof}

\begin{theorem} \label{simple_finite_bij}
    If $\U$ is a simply finite type, then there exists an $n : \N$ and a
    bijective function $f : \U \to \T(n)$.
\end{theorem}
\begin{proof}
    Let $n$ be a natural number and $f$ an injective function from $\U$ to
    $\T(n)$.  The proof will be by induction on $n$.  When $n = 0$, because
    $\T(0)$ is empty, $f$ is necessarily the empty function, which is bijective.
    Thus the base case holds.  Now assume that for all types $\U'$ and natural
    numbers $n$ with an injective function $f' : \U' \to \T(n)$, there exists a
    natural number $m$ and a bijective function $g' : \U' \to \T(m)$.  We must
    prove that for all $n$ and injective functions $f : \U \to \T(S(n))$, there
    exists a bijective function $g : \U \to \T(S(n))$.  Now if $f(a) \neq n$ for
    all $a : \U$, then $f$ itself is a function from $\U$ to $\T(n)$, so the
    result follows directly from the inductive hypothesis.  Thus, we now only
    need to consider the case of when there exists an $a : \U$ such that $f(a) =
    n$.

    Define a new type $\V = \T(\{x : \U \mid x \neq a\})$.  We will now prove
    that for all $x : \V$, $f(x) < n$.  First, by definition of $f$, we have
    $f(x) < S(n)$, which by Theorem \ref{nat_lt_suc_le} means that $f(x) \leq
    n$.  Assume for a contradiction that $f(x) = n$.  Then because $f(a) = n$,
    we have $f(x) = f(a)$, and by injectivity, $x = a$, which is a
    contradiction.  Thus, $f(x) \neq n$.  Because $f(x) \leq n$ and $f(x) \neq
    n$, we have $f(x) < n$.

    Thus, restricting $f$ to $\V$ produces an injective function $\V \to \T(n)$,
    so by the inductive hypothesis there exists a natural number $m$ and a
    bijective function $g' : \V \to \T(m)$.  We can define a new function $g :
    \U \to \T(S(m))$ by
    \[
        g(x) = \begin{cases}
            m \quad &\text{if $x = a$} \\
            g'(x) \quad &\text{if $x \neq a$.}
        \end{cases}
    \]
    We will now prove that $g$ is bijective.

    To prove that $g$ is injective, let $x$ and $y$ be elements of $\U$ with
    $g(x) = g(y)$.  We must prove that $x = y$.  If $x = a$ and $y = a$, this is
    trivial.  If $x = a$ and $y \neq a$, $g(x) = g(y)$ implies that $m = g'(y)$.
    But $g'(y) < m$, so this is impossible.  The case $x \neq a$ and $y = a$ is
    similar.  If $x \neq a$  and $y \neq a$, then $g(x) = g(y)$ implies that
    $g'(x) = g'(y)$, and because $g'$ is injective, $x = y$ as required.

    To prove that $g$ is surjective, let $y$ be a natural number less than
    $S(m)$.  If $y = m$, then $g(a) = m = y$ as required.  If $y \neq m$, then
    $y < m$, and because $g'$ is surjective, there exists an $x$ such that
    $g'(x) = y$, and thus $g(x) = y$ as required.  Thus, $g$ is surjective.

    Because $g$ is injective and surjective, it is bijective, so $g$ is a
    bijective function from $\U$ to $\T(S(m))$.
\end{proof}

\begin{theorem} \label{nat_not_finite}
    The natural numbers are not simply finite.
\end{theorem}
\begin{proof}
    Assume that there exists a natural number $n$ and an injective function $f :
    \N \to \T(n)$.  We will derive a contradiction using induction on $n$.  When
    $n = 0$, then $f$ is a function from a nonempty type to the empty type
    $\T(0)$, which is impossible.  Thus the base case is true.  Now assume that
    it's impossible to have an injective function from $\N$ to $\T(n)$.  We must
    derive a contradiction from the existence of an injective function $f : \N
    \to \T(S(n))$.  Now if $f(a) \neq n$ for all $a : \N$, then $f$ itself is a
    function from $\N$ to $\T(n)$, which is impossible by the inductive
    hypothesis.  Thus, we now only need to consider the case of when there
    exists an $a : \N$ such that $f(a) = n$.

    Now define a new function $g : \N \to \N$ given by
    \[
        g(x) = \begin{cases}
            f(x) \quad &\text{if $x < a$} \\
            f(S(x)) \quad &\text{otherwise.}
        \end{cases}
    \]
    We will now prove that for all $x : \N$, $g(x) < n$.  First, because $g$ is
    defined to always be some output of $f$, $g(x) < S(n)$, so $g(x) \leq n$.
    So we only need to prove that $g(x) \neq n$.  If $g(x) = n$, because $f(a) =
    n$, we have $g(x) = f(a)$.  Now when $x < a$, we have $g(x) = f(x) = f(a)$,
    and by injectivity we have $x = a$, contradicting $x < a$.  Now when $\neg(x
    < a)$, we have $g(x) = f(S(x)) = f(a)$, and by injectivity we have $S(x) =
    a$, meaning that $\neg(x < S(x))$, contradicting Theorem \ref{nat_lt_suc}.
    Thus, both cases have a contradiction, so we have $g(x) \neq n$, meaning
    that $g(x) < n$ for all $x : \N$.  This means that $g$ is a function from
    $\N \to \T(n)$.

    We will now prove that it is injective.  Let $x$ and $y$ be natural numbers
    such that $g(x) = g(y)$.  We must prove that $x = y$.  If $x < a$ and $y <
    a$, we have $f(x) = f(y)$, and by injectivity of $f$ we get $x = y$.  If $x
    < a$ and $a \leq y$, we get $f(x) = f(S(y))$, and by injectivity of $f$ we
    get $x = S(y)$.  But this implies $S(y) < a \leq y$, contradicting Theorem
    \ref{nat_lt_suc}.  The case $a \leq x$ and $y < a$ is similar.  If $a \leq
    x$ and $a \leq y$, then $f(S(x)) = f(S(y))$, and by injectivity of $f$ we
    get $S(x) = S(y)$, so $x = y$.  By exhausting all cases, we see that $g$ is
    injective.

    We now see that $g$ is an injective function from $\N \to \T(n)$, which is
    impossible by the inductive hypothesis.  Thus, we see by induction that
    there is no injective function $\N \to \T(n)$ for any $n : \N$, showing that
    $\N$ is not simply finite.
\end{proof}

\begin{theorem} \label{simple_finite_union}
    For all sets $S$ and $T$, if $\T(S)$ and $\T(T)$ are simply finite, then
    $\T(S \cup T)$ is simply finite.
\end{theorem}
\begin{proof}
    Let $m$ and $n$ be natural numbers such that there are injective functions
    $f : \T(S) \to \T(m)$ and $g : \T(S) \to \T(n)$.  Now because $f(x) < m$ for
    all $x : \T(S)$, we also have $f(x) < m + n$ for all $x$.  Furthermore,
    because $g(x) < n$ for all $x : \T(T)$, we have $m + g(x) < m + n$.  With
    this we can define a function $h$ from $\T(S \cup T)$ to $\T(m + n)$:
    \[
        h(x) =
        \begin{cases}
            f(x) \quad &\text{if $x \in S$} \\
            m + g(x) \quad &\text{if $x \in T$.}
        \end{cases}
    \]
    It remains to prove that $h$ is injective.  Let $a$ and $b$ be values in
    $\T(S \cup T)$ such that $h(a) = h(b)$.  We must prove that $a = b$.  There
    are four cases, depending on when $a$ and $b$ are in $S$ or $T$.
    \begin{case} $a \in S$ and $b \in S$. \\
        Here we have $f(a) = f(b)$.  Because $f$ is injective, we have $a = b$
        as required.
    \end{case}
    \begin{case} $a \in S$ and $b \in T$. \\
        Here we have $f(a) = m + g(b)$.  But $f(a) < m$, meaning that $m + g(b)
        < m$.  Thus, this case is impossible.
    \end{case}
    \begin{case} $a \in T$ and $b \in S$. \\
        Here we have $m + g(a) = f(b)$.  But $f(b) < m$, meaning that $m + g(a)
        < m$.  Thus, this case is impossible.
    \end{case}
    \begin{case} $a \in T$ and $b \in T$. \\
        Here we have $m + g(a) = m + g(b)$.  By canceling $m$ we get $g(a) =
        g(b)$.  Because $g$ is injective, we have $a = b$ as required.
    \end{case}
\end{proof}

\begin{theorem} \label{simple_finite_min}
    If $\U$ is a totally-ordered simply finite set, then $\U$ has a minimum
    element.
\end{theorem}
\begin{proof}
    Assume that $\U$ has no minimum element.  This means that for all $m : \U$,
    there exists an $a < m$.  Let $f$ be a function from $\N$ to $\U$ defined
    recursively such that $f(S(x))$ is such an element less than $f(x)$.  We
    will prove that $f$ is injective.  Let $a$ and $b$ be natural numbers such
    that $f(a) = f(b)$.  We must prove that $a = b$.

    Assume that $a \neq b$.  Without loss of generality we can assume that $a <
    b$.  From this we can prove that $f(b) < f(a)$.  Because $a < b$, we know
    that there exists a $c$ such that $a + S(c) = b$.  We will now prove that
    $f(a + S(c)) < f(a)$ by induction on $c$.  When $c = 0$, $f(S(a)) < f(a)$ by
    the definition of $f$, so the base case is true.  When $f(a + S(c)) < f(a)$,
    then $f(a + S(S(c))) < f(a + S(c))$ by the definition of $f$, so by
    transitivity we have $f(a + S(S(c))) < f(a)$, proving the inductive case.
    Thus, by induction, $f(b) < f(a)$.  But this contradicts $f(a) = f(b)$,
    meaning that our original assumption of $a \neq b$ is false.  Thus, $a = b$,
    and $f$ is injective.

    Because $\U$ is simply finite, there exists a natural number $n$ and an
    injective function $g : \U \to \T(n)$.  The function $g \circ f : \N \to
    \T(n)$ is thus an injective function, proving that $\N$ is simply finite,
    contradicting Theorem \ref{nat_not_finite}.  Thus, $\U$ has a minimum
    element.
\end{proof}

\begin{theorem} \label{simple_finite_max}
    If $\U$ is a totally-ordered simply finite set, then $\U$ has a maximum
    element.
\end{theorem}
\begin{proof}
    This follows from the previous theorem by using the dual relation to $\leq$.
\end{proof}

\section{Transfinite Induction and Recursion}

In a well-ordered type, there are two useful theorems that we can use:
transfinite induction and transfinite recursion.  In this section, let $\U$ be a
well-ordered type.

\begin{theorem}[Transfinite Induction] \label{transfinite_induction}
    For all sets $S : \set{\U}$, if for all $\alpha : \U$, $\beta \in S$ for all
    $\beta < \alpha$ implies that $\alpha \in S$, then $\alpha \in S$ for all
    $\alpha : \U$.
\end{theorem}
\begin{proof}
    Assume that there is some $\alpha \notin S$.  Then there exists some element
    in set $S^c$, and because $\U$ is well-ordered, there exists a least element
    $\beta \in S^c$.  Because $\beta$ is the least element in $S^c$, every
    element less than $\beta$ is in $S$, so by hypothesis $\beta \in S$ as well,
    which contradicts $\beta \in S^c$.  Thus, there is no $\alpha \notin S$, so
    $\alpha \in S$ for all $\alpha : \U$.
\end{proof}

Transfinite recursion is a bit trickier.  The proof employed here has a few
lemmas that are repeated for the whole type and for initial segments.  I feel
like this repetition is unneeded because we can consider an initial segment to
be its own type, but I have not managed to figure out how to do this in the Coq
code, so the extra lemmas will be presented here.  Because the proof of the
lemmas for initial segments is identical to the proof for the whole type, they
will not be presented here.

Throughout the rest of this section, in addition to $\U$ being a well-ordered
type, let $\A$ be any type, and let $f$ be a binary function from values $p$ in
$\U$ and functions $\T(\U_p) \to \A$ to $\A$.  More explicitly, $f$ is of the
type $\forall p : \U, (\T(\U_p) \to \A) \to \A$.  Given a function $g : \U \to
\A$ and a value $p : \U$, we define $g \uparrow p$ to be the function from
$\T(\U_p)$ to $\A$ given by $(g \uparrow p)(x) = g(\stv x)$.

\begin{theorem}[Uniqueness of Transfinite Recursion]
    For all functions $g$ and $h$ from $\U$ to $\A$, if $g(n) = f(n, g \uparrow
    n)$ and $h(n) = f(n, h \uparrow n)$ for all $n : \U$, then $g = h$.
\end{theorem}
\begin{proof}
    Let $x : \U$.  We must prove that $g(x) = h(x)$.  By transfinite induction,
    we can assume that for all $y < x$, $g(y) = h(y)$.  By hypothesis, we can
    replace $g(x)$ and $h(x)$ with $f(x, g \uparrow x)$ and $f(x, h \uparrow x$,
    so we must prove that $f(x, g \uparrow x) = f(x, h \uparrow x)$.  It
    suffices to prove that $g \uparrow x = h \uparrow x$.  Let $y$ be such that
    $y < x$.  We must prove that $(g \uparrow x)(y) = (h \uparrow x)(y)$, that
    is, that $g(y) = h(y)$.  This is true by the inductive hypothesis.  Thus, by
    transfinite induction, the theorem is true.
\end{proof}

\begin{lemma}[Uniqueness of Transfinite Recursion for Initial Segments]
    For all $\alpha : \U$ and functions $g$ and $h$ from $\T(\U_\alpha)$ to
    $\A$, if $g(n) = f(\stv n, g \uparrow n)\footnote{Note that using $g
    \uparrow n$ in this context is a bit of an abuse of notation.  $g \uparrow
    n$ is technically a function from $\T(\T(\U_\alpha)_n)$ to $\A$, but we are
    considering it a function from $\T(\U_n) \to \A$, which we can do by
    transitivity.  This abuse of notation will be used in the rest of this
    section as well.}$ and $h(n) = f(\stv n, h \uparrow n)$ for all $n :
    \T(\U_\alpha)$, then $g = h$.
\end{lemma}
\begin{proof}
    This proof is similar in form to the previous theorem.
\end{proof}

\begin{lemma} \label{transfinite_recursion_part}
    For all $g : \forall n, \T(\U_n) \to \A$, if for all $n : \U$ and $a :
    \T(\U_n)$ we have $g(n, a) = f(\stv a, g(n) \uparrow a)$, then for all $n :
    \U$ and $a : \T(U_n)$, we have $g(n, a) = f(\stv a, g(\stv a))$.
\end{lemma}
\begin{proof}
    By hypothesis, $g(n, a) = f(\stv a, g(n) \uparrow a)$, so we need to prove
    that $f(\stv a, g(n) \uparrow a) = f(\stv a, g(\stv a))$.  It suffices to
    prove that $g(n) \uparrow a = g(\stv a)$, which follows from the uniqueness
    of transfinite recursion.
\end{proof}

\begin{lemma} \label{transfinite_recursion_part_initial}
    For all $\alpha : \U$ and $g : \forall n : \T(\U_\alpha), \T(\U_{\stv n})
    \to \A$, if for all $n : \T(\U_\alpha)$ and $a : \T(\U_{\stv n})$ we have
    $g(n, a) = f(\stv a, g(n) \uparrow a)$, then for all $n : \T(\U_\alpha)$ and
    $a : \T(U_{\stv n})$, we have $g(n, a) = f(\stv a, g(\stv a))$.
\end{lemma}
\begin{proof}
    This proof is similar in form to the previous theorem.
\end{proof}

\begin{lemma} \label{transfinite_recursion_initial}
    For all $\alpha : \U$, there exists a function $g : \T(\U_\alpha) \to \A$
    such that for all $n$, $g(n) = f(\stv n, g \uparrow n)$.
\end{lemma}
\begin{proof}
    By transfinite induction on $\alpha$, we can assume that for all $\beta <
    \alpha$, there exists a function $h_\beta : \T(\U_\beta) \to \A$ such that
    for all $n : \T(\U_\beta)$, $h_\beta(n) = f(\stv n, h_\beta \uparrow n)$.
    We can now define the function $g(n)$ as $g(n) = f(\stv n, h_n)$.  Then for
    all $n < \alpha$, we must prove that $f(\stv n, h_n) = f(\stv n, f(\stv n,
    h_n) \uparrow n)$.  It suffices to prove that $h_n = f(\stv n, h_n) \uparrow
    n$.  This follows from Lemma \ref{transfinite_recursion_part_initial}.
\end{proof}

\begin{theorem}[Transfinite Recursion]
    There exists a function $g : \U \to \A$ such that for all $n$, $g(n) = f(n,
    g \uparrow n)$.
\end{theorem}
\begin{proof}
    By Lemma \ref{transfinite_recursion_initial}, for all $\alpha$, there exists
    a function $h_\alpha : \T(\U_\alpha) \to \A$ such that for all $n :
    \T(\U_\alpha)$, $h_\alpha(n) = f(\stv n, h_\alpha \uparrow n)$.  We can now
    define the function $g(n)$ as $g(n) = f(n, h_n)$.  Then for all $n$, we must
    prove that $f(n, h_n) = f(n, f(n, h_n) \uparrow n)$.  It suffices to prove
    that $h_n = f(n, h_n) \uparrow n$.  This follows from Lemma
    \ref{transfinite_recursion_part}.
\end{proof}

\section{Zorn's Lemma}

A common alternative to transfinite recursion is Zorn's Lemma.  Throughout this
section, let $\U$ be a partially-ordered set.  All lemmas and definitions made
in this section up until Theorem \ref{zorn} will not be referenced outside of
this section.  We say that a set $S$ has a strict upper bound if there exists an
$x$ such that for all $a \in S$, $a < x$.

\begin{definition}
    Define a collection of sets $\mathcal F$ where $S \in \mathcal F$ if and
    only if $S$ well-orders $\U$ and $S$ has a strict upper bound.
\end{definition}

\begin{definition}
    Define a function $f : \mathcal F \rightarrow \U$ that takes in a set in
    $\mathcal F$ and produces a strict upper bound of that set.
\end{definition}

\begin{definition}
    Given a set $C$ and a value $x$, define $P(C, x) = \{y \mid y \in C \wedge y
    < x\}$.
\end{definition}

\begin{lemma}
    For all sets $A$ and values $x$, if $A$ well-orders $\U$, then $P(A, x) \in
    \mathcal F$.
\end{lemma}
\begin{proof}
    Because $A$ well-orders $\U$, we know that $P(A, x)$ well-orders $\U$ as
    well.  Because $y < x$ for all $y \in P(A, x)$, $x$ is a strict upper bound.
    Thus, $P(A, x) \in \mathcal F$.
\end{proof}

\begin{definition}
    If a set $A$ well-orders $\U$ and if for all $x \in A$, $x = f(P(A, x))$,
    then we say that $A$ is conforming.  Call the collection of conforming sets
    $\mathcal C$.
\end{definition}

\begin{lemma} \label{conforming_add_wo}
    If $A \in \mathcal F$ is a conforming set, then $A \cup \{f(A)\}$
    well-orders $\U$.
\end{lemma}
\begin{proof}
    Let $S$ be a nonempty subset of $A \cup \{f(A)\}$.  There are two cases:
    when there exists something in $S$ that is not equal to $f(A)$, or when no
    such element exists.

    When there exists a $y \in S$ such that $f(A) \neq y$, we must have $y \in
    A$.  Because $A$ is conforming, it is well-ordered, and because $y \in A
    \cap S$, there exists a least element $a \in A \cap S$.  We will prove that
    $a$ is the least element of $S$.  Let $z \in S$.  Because $S \subseteq A
    \cup \{f(A)\}$, either $z \in A$ or $z \in \{f(A)\}$.  When $z \in A$, we
    have $a \leq z$ because $a$ is the minimum element of $A \cap S$.  When $z
    \in \{f(A)\}$, we have $z = f(A)$, and $a < z$ by the definition of $f(A)$.
    Either way, $a \leq z$, so $a$ is the minimum element of $S$.

    Now when there is nothing in $S$ that is not equal to $f(A)$, because $S$ is
    nonempty, that means that everything in $S$ is equal to $f(A)$, that is, $S
    = \{f(A)\}$.  In this case, $f(A)$ is trivially the least element of $S$.
\end{proof}

\begin{lemma} \label{conforming_add_conforming}
    If $A \in \mathcal F$ is a conforming set, then $A \cup \{f(A)\}$ is
    conforming.
\end{lemma}
\begin{proof}
    We already know that $A \cup \{f(A)\}$ well-orders $\U$ by the previous
    theorem, so we only need to prove that for all $y \in A \cup \{f(A)\}$, we
    have $y = f(P(A \cup \{f(A)\}, y))$.  There are two cases: when $y
    \in A$, and when $y \in \{f(A)\}$.

    When $y \in A$, because $A$ is conforming, $y = f(P(A, y))$.  Thus, we must
    prove that $f(P(A, y)) = f(P(A \cup \{f(A)\}, y))$.  It suffices to prove
    that $P(A, y) = P(A \cup \{f(A)\}, y)$, which we will do by antisymmetry.
    Let $a \in P(A, y)$.  Then $a \in A \subseteq A \cup \{f(A)\}$ and $a < y$,
    so $a \in P(A \cup \{f(A)\}, y)$, showing that $P(A, y) \subseteq P(A \cup
    \{f(A)\}, y)$.  Now let $a \in P(A \cup \{f(A)\}, y)$.  We already have $a <
    y$, and when $a \in A$, we easily have $a \in P(A, y)$.
    When $a \in \{f(A)\}$, we have $f(A) < y$, contradicting the definition of
    $f(A)$.  So we see that $P(A \cup \{f(A)\}, y) \subseteq P(A, y)$.  Thus, by
    antisymmetry, $P(A, y) = P(A \cup \{f(A)\}, y)$, showing that this case is
    true.

    Now when $y \in \{f(A)\}$, we must prove that $f(A) = f(P(A \cup \{f(A)\},
    f(A)))$.  It suffices to prove that $A = P(A \cup \{f(A)\}, f(A))$, which we
    will do by antisymmetry.  Let $a \in A$.  Then $a \in A \cup \{f(A)\}$ as
    well, and $a < f(A)$ by the definition of $f(A)$.  Thus, $a \in P(A \cup
    \{f(A)\}, f(A))$, so $A \subseteq P(A \cup \{f(A)\}, f(A))$.  Now let $a \in
    P(A \cup \{f(A)\}, f(A))$.  Now $a \notin \{f(A)\}$ because $a < f(A)$, so
    we must have $a \in A$, showing that $P(A \cup \{f(A)\}, f(A)) \subseteq A$.
    By antisymmetry, we have seen that $A = P(A \cup \{f(A)\}, f(A))$, showing
    that this case is true as well.
\end{proof}

\begin{lemma} \label{zorn_initial1}
    For all conforming sets $A$ and $B$ and values $a$ and $b$, if $b \in A$, $a
    \in B$, and $b \notin B$, then $a < b$.
\end{lemma}
\begin{proof}
    Because $b \in A - B$ and $A$ well-orders $\U$, there exists a least element
    $x \in A - B$.  By transitivity we only need to prove that $a < x$.  For the
    sake of a contradiction, assume that $a \nless x$.  Then $a \in B - P(A,
    x)$, so there exists a least element $y \in B - P(A, x)$.  We know that $x
    \in A - P(B, y)$, so there exists a least element $z \in A - P(B, y)$.  By
    definition we have $z \leq x$.

    We will now prove that $P(A, z) = P(B, y)$.  To prove that $P(A, z)
    \subseteq P(B, y)$, let $c \in P(A, z)$.  If $c \notin P(B, y)$, then we
    would have both $c \in A - P(B, y)$ and $c < z$, contradicting the
    minimality of $z$, so $c \in P(B, y)$.  Thus, $P(A, z) \subseteq P(B, y)$.

    To prove that $P(B, y) \subseteq P(A, z)$, let $u \in P(B, y)$.  Now $u$
    must be in $P(A, x)$, because if it was not, we would have $u \in B - P(A,
    x)$ and $u < y$, contradicting the minimality of $y$.  Thus, we also have $u
    \in A$, so to prove that $u \in P(A, z)$ all we must prove is that $u < z$.
    Now $u = z$ is impossible because $u \in P(B, y)$ and $z \notin P(B, y)$.
    Because $u \in A$ and $z \in A$, they are comparable.  If $u \leq z$,
    because $u \neq z$, we have $u < z$ as required.  We will show that the $z
    \leq u$ case is impossible by showing that it implies $z \in P(B, y)$,
    contradicting $z \notin P(B, y)$.  If $z \leq u$, then because $u < y$, we
    have $z < y$, so all that remains is showing that $z \in B$.  If $z \notin
    B$, then $z \in A - B$.  Because $u < x$ (because $u \in P(A, x)$), we also
    have $z < x$.  This contradicts the minimality of $x$, showing that the case
    $z \leq u$ is impossible.  Thus, $u \in A$ and $u < z$, showing that $u \in
    P(A, z)$.  This shows that $P(B, y) \subseteq P(A, z)$, so $P(A, z) = P(B,
    y)$.

    Because $P(B, y) = P(A, z)$, we have $f(P(B, y)) = f(P(A, z))$.  Because $A$
    and $B$ are conforming, this implies $y = z$.  Because $z \in A$, we now
    have $y \in A$.  Because $z \leq x$, we also have $y \leq x$.  Furthermore,
    $y \neq x$ because $y \in B$ and $x \notin B$, so $y < x$.  Thus, because $y
    \in A$ and $y < x$, we have $y \in P(A, x)$, contradicting the definition of
    $y$.  This shows that our (much) earlier assumption that $a \nless x$ is
    false, proving the lemma.
\end{proof}

\begin{lemma} \label{zorn_initial2}
    For all conforming sets $A$ and $B$ and values $a$ and $b$, if $a \leq b$,
    $a \in A$, and $b \in B$, then $a \in B$.
\end{lemma}
\begin{proof}
    Assume that $a \notin B$.  Then by the previous lemma we would have $b < a$,
    contradicting $a \leq b$.
\end{proof}

\begin{lemma}
    $\bigcup \mathcal C$ is conforming.
\end{lemma}
\begin{proof}
    To prove that $\bigcup \mathcal C$ well-orders $\U$, let $S$ be a nonempty
    subset of $\bigcup \mathcal C$, so that there exists an $x \in S$.  Because
    $x \in S \subseteq \bigcup \mathcal C$, there exists some conforming set $A$
    such that $x \in A$.  Because $A$ well-orders $\U$, because $A \cap S
    \subseteq A$ and $x \in A \cap S$, we know that there exists a minimum
    element $a \in A \cap S$.  We will prove that $a$ is the minimum element of
    $S$ as well.  We know that $a \in S$ already, so all we need to prove is
    that for all $y \in S$, we have $a \leq y$.  Because $y \in S \subseteq
    \bigcup \mathcal C$, there exists a conforming set $B$ such that $y \in B$.
    Now if $y \in A$, then $a \leq y$ because $a$ is the minimum element of $A
    \cap S$.  If $y \notin A$, then $a < y$ by Lemma \ref{zorn_initial1}.
    Either way, $a \leq y$, so $a$ is the minimum element of $S$, showing that
    $\bigcup \mathcal C$ well-orders $\U$.

    Now we must show that for all $x \in \bigcup \mathcal C$, we have $x =
    f(P(\bigcup \mathcal C, x))$.  Because $x \in \bigcup \mathcal C$, there
    exists some conforming set $A$ such that $x \in A$.  Because $A$ is
    conforming, we know that $x = f(P(A, x))$, so we must prove that $f(P(A, x))
    = f(P(\bigcup \mathcal C, x))$.  It suffices to prove that $P(A, x) =
    P(\bigcup \mathcal C, x)$, which we will now prove by antisymmetry.  Let $c
    \in P(A, x)$.  Since $c \in A$ and $A$ is conforming, we know that $c \in
    \bigcup \mathcal C$, and because $c < x$, we have $c \in P(\bigcup \mathcal
    C, x)$.  Thus, $P(A, x) \subseteq P(\bigcup \mathcal C, x)$.  Now let $c \in
    P(\bigcup \mathcal C, x)$.  Because $c \in \bigcup \mathcal C$, there exists
    a conforming set $B$ with $c \in B$.  By Lemma \ref{zorn_initial2} we know
    that $c \in A$, and because $c < x$, we have $c \in P(A, x)$.  Thus,
    $P(\bigcup \mathcal C, x) \subseteq P(A, x)$, so by antisymmetry we have
    $P(A, x) = P(\bigcup \mathcal C, x)$.
\end{proof}

\begin{lemma} \label{union_no_sub}
    $\bigcup \mathcal C$ has no strict upper bound.
\end{lemma}
\begin{proof}
    Assume that it does have a strict upper bound.  Then $\bigcup \mathcal C \in
    \mathcal F$.  Define $x = f(\bigcup \mathcal C)$.  By Lemma
    \ref{conforming_add_conforming}, we know that $\bigcup \mathcal C \cup
    \{x\}$ is conforming, so $\bigcup \mathcal C \cup \{x\} \subseteq \bigcup
    \mathcal C$, showing that $x \in \bigcup \mathcal C$.  However, this implies
    that $x < x$, which is a contradiction.  Thus, $\bigcup \mathcal C$ has no
    strict upper bound.
\end{proof}

Notice that up until now we have never used the hypothesis of Zorn's lemma,
showing that in any partially ordered set, there exists a well-ordered subset
that has no strict upper bound.  From this fact, Zorn's lemma itself is trivial.

\begin{theorem}[Zorn's Lemma] \label{zorn}
    If every chain in $\U$ has an upper bound, then $\U$ has a maximal element.
\end{theorem}
\begin{proof}
    $\bigcup \mathcal C$ is a chain, so it has an upper bound $m$.  If there
    existed some element $x > m$, $x$ would be a strict upper bound of $\bigcup
    \mathcal C$, contradicting Lemma \ref{union_no_sub}.  Thus, $m$ is a maximal
    element of $\U$.
\end{proof}

The rest of this section will be several subsections describing particular
special cases of Zorn's lemma that are often useful.

\subsection{Zorn's lemma on preorders}

Let $\U$ be a set with an order that is only reflexive and transitive.  Such an
order is called a preorder.

\begin{definition}
    Define a relation $\sim$ on $\U$ where $a \sim b$ if and only if $a \leq b$
    and $b \leq a$.
\end{definition}

\begin{instance}
    $\sim$ is an equivalence relation.
\end{instance}
\begin{proof}
    $\sim$ is reflexive by the reflexivity of $\leq$.  $\sim$ is symmetric by
    the commutativity of logical conjunction.  Now if $x \sim y$ and $y \sim z$,
    then $x \leq z$ and $z \leq x$ by transitivity of $\leq$, so $\sim$ is
    transitive.
\end{proof}

\begin{lemma}
    $\leq$ is well-defined under the equivalence relation $\sim$.
\end{lemma}
\begin{proof}
    By symmetry all we need to prove is that if $a \sim b$, $c \sim d$, and $a
    \leq c$, then $b \leq d$.  From $a \sim b$ we know that $b \leq a$, and from
    $c \sim d$ we know that $c \leq d$.  Thus, by transitivity with $b \leq a$,
    $a \leq c$, and $c \leq d$, we have $b \leq d$ as required.
\end{proof}

\begin{instance}
    In $\U/\usim$, $\leq$ is a partial order.
\end{instance}
\begin{proof}
    Reflexivity and transitivity follow directly from the reflexivity and
    transitivity of $\leq$ in $\U$.  For antisymmetry, $a \leq b$ and $b \leq a$
    is the definition of $a = b$.
\end{proof}

Using $\sim$, we can prove a version of Zorn's lemma that is valid on preorders:

\begin{theorem}[Zorn's lemma on preorders] \label{preorder_zorn}
    If every chain in $\U$ has an upper bound, then there exists an element $a :
    \U$ such that for all $x \geq a$, we also have $a \geq x$.
\end{theorem}
\begin{proof}
    Let $F'$ be a chain in $\U/\usim$.  Now define a set $F : \U \to \Prop$
    given by $F = \{x : \U, [x] \in F'\}$.  We will prove that $F$ is a chain.
    Let $x$ and $y$ be elements of $F$.  Then $[x] \in F'$ and $[y] \in F'$, and
    because $F'$ is a chain, $[x]$ and $[y]$ are comparable.  Thus, $x$ and $y$
    are comparable as well, proving that $F$ is a chain.  Thus, by hypothesis,
    $F$ has an upper bound $a$.  Let $[x]$ be an element of $F'$.  Then $x \in
    F$, so $x \leq a$.  This means that $[x] \leq [a]$ as well, showing that
    $[a]$ is an upper bound of $F'$.  Thus, every chain in $\U/\usim$ has an
    upper bound.

    Because of this, we can apply Zorn's lemma on $\U/\usim$ to get a maximal
    element $[a]$ of $\U/\usim$.  We will show that $a$ is the element we are
    looking for.  Let $x \geq a$.  Then $[x] \geq [a]$, and because $[a]$ is
    maximal, we must have $[x] = [a]$.  This implies that $a \geq x$, as
    required.
\end{proof}

\subsection{Zern's lemma on subsets}

\begin{theorem} \label{set_zorn}
    Let $\U$ be a type and $\mathcal S$ be a collection of sets on $\U$.  If
    for every chain $\mathcal F$ made up of sets in $\mathcal S$, we have
    $\bigcup \mathcal F \in \mathcal S$ as well, then $\mathcal S$ has a maximal
    element with respect to set inclusion.
\end{theorem}
\begin{proof}
    $\mathcal S$ is partially ordered by inclusion, so if we can prove that
    every chain has an upper bound, Zorn's lemma provides the maximal element.
    Given a chain $\mathcal F$ and a set $A \in \mathcal F$, we know that $A
    \subseteq \bigcup \mathcal F$.  Because $\bigcup \mathcal F \in \mathcal S$
    by hypothesis, this means that $\bigcup \mathcal F$ is an upper bound of
    $\mathcal F$.
\end{proof}

\section{The Well-Ordering Theorem}

\begin{theorem}
    Every type can be well-ordered.
\end{theorem}
\begin{proof}
    Let $\U$ be a type.  Consider the set $\mathcal F : ((\U \to \Prop) \times
    (\U \to \U \to \Prop)) \to \Prop$, where $(D, \leq) \in \mathcal F$ if and
    only if:
    \begin{enumerate}
        \item For all $a$ and $b$ in $D$, if $a \leq b$ and $b \leq a$, then $a
        = b$.
        \item $D$ well-orders $\U$ under $\leq$.
        \item If $a \notin D$, then for all $b$, $a \leq b$ and $b \leq a$.
    \end{enumerate}
    Note that this is slightly different from the usual proofs given for the
    well-ordering theorem: they usually consider well-orderings of subsets of
    $\U$, whereas here we allow the relation to be defined on all of $\U$, but
    only a well-ordering on a subset.  This is because working on relations on
    sets of a type involves having to deal with dependent types, which can be
    annoying to work with in Coq and their use is preferably avoided unless
    needed.

    For any $(D, \leq) \in \mathcal F$, because $D$ well-orders $\U$, we also
    know that $D$ is a chain, and that $\leq$ is reflexive (in fact, $\leq$ is
    reflexive outside of $D$ as well, because we have defined $\mathcal F$ such
    that all elements outside $D$ are related to every element).

    To prove that $(A, \leq_A) = (B, \leq_B)$, it suffices to prove that $A = B$
    and that $\leq_A$ and $\leq_B$ agree on $A$.  Both $\leq_A$ and $\leq_B$ are
    defined such that any elements not in $A$ are related to every other
    element, so if they agree on $A$, they agree on all arguments and are equal.

    We will now define an order on $\mathcal F$ where we define $(A, \leq_A)
    \leq (B, \leq_B)$ if and only if:
    \begin{enumerate}
        \item (Subset) $A \subseteq B$.
        \item (Equality) For all $a \in A$ and $b \in B$, if $a \leq_A b$, then
        $a \leq_B b$.
        \item (Extension) For all $a \in A$ and $b \notin A$, then $a \leq_B b$.
    \end{enumerate}
    This is a partial order, as we will now prove.

    Let $(D, \leq_D)$ be in $\mathcal F$.  We have $D \subseteq D$ and $a \leq_D
    b \to a \leq_D b$ trivially.  The extension condition for $(D, \leq_D) \leq
    (D, \leq_D)$ follows from the extension condition of $\mathcal F$.  Thus,
    the order on $\mathcal F$ is reflexive.

    Let $(A, \leq_A)$ and $(B, \leq_B)$ be elements of $\mathcal F$ such that
    $(A, \leq_A) \leq (B, \leq_B)$ and $(B, \leq_B) \leq (A, \leq_B)$.  We have
    $A \subseteq B$ and $B \subseteq A$, so $A = B$.  To prove that $\leq_A =
    \leq_B$, let $a$ and $b$ be elements of $A$.  We need to prove that $a
    \leq_A b \leftrightarrow a \leq_B b$.  This follows from the equality
    condition of $\leq$.  Thus, the order on $\mathcal F$ is antisymmetric.

    Let $(A, \leq_A)$, $(B, \leq_B)$, and $(C, \leq_C)$ be elements of $\mathcal
    F$ such that $(A, \leq_A) \leq (B, \leq_B)$ and $(B, \leq_B) \leq (C,
    \leq_C)$.  Then we have $A \subseteq B$ and $B \subseteq C$, so $A \subseteq
    C$ by transitivity.  Now let $a \in A$ and $b \in A$ and assume that $a
    \leq_A b$.  Then from the equality condition of $(A, \leq_A) \leq (B,
    \leq_B)$, we have $a \leq_B b$, and from the equality condition of $(B,
    \leq_B) \leq (C, \leq_C)$, we have $a \leq_C b$ as required for the equality
    condition of $\leq$.  Finally, we need to prove that for all $a \in A$ and
    $b \notin A$, we have $a \leq_C b$.  From the extension condition for $(A,
    \leq_A) \leq (B, \leq_B)$, we have $a \leq_B b$.  Now if $b \in B$, we would
    have $a \leq_C b$ from the equality condition of $(B, \leq_B) \leq (C,
    \leq_C)$.  If $b \notin B$, we would have $a \leq_C b$ from the extension
    condition of $(B, \leq_B) \leq (C, \leq_C)$.  Either way, $a \leq_C b$,
    showing that the extension condition holds.  Thus, all three conditions
    hold, so $(A, \leq_A) \leq (C, \leq_C)$, and the order on $\mathcal F$ is
    transitive.

    We will now show that every chain in $\mathcal F$ has an upper bound.  Let
    $\mathcal C$ be a chain in $\mathcal F$.  Consider the set $F$ that
    is the union of all of the domains in $\mathcal C$.  We will also define a
    relation $\leq_F$ by saying that for $a \leq_F b$ if and only if either one
    of these conditions hold:
    \begin{enumerate}
        \item There exists an $(A, \leq_A) \in \mathcal C$ such that $a \in A$,
        $b \in A$, and $a \leq_A b$.
        \item There exists no $(A, \leq_A) \in \mathcal C$ such that $a \in A$
        and $b \in A$.
    \end{enumerate}
    We will now prove that $(F, \leq_F) \in \mathcal F$.

    For antisymmetry, let $a$ and $b$ be in $F$, and let $a \leq_F b$ and $b
    \leq_F a$.  We have four cases, depending on the two cases created by
    $a \leq_F b$ and $b \leq_F a$.
    \begin{case} There exists an $(A, \leq_A) \in \mathcal C$ such that $a \in
    A$, $b \in A$, and $a \leq_A b$, and there exists a $(B, \leq B) \in
    \mathcal C$ such that $a \in B$, $b \in B$, and $b \leq_B a$. \\
        \indent In this case, because $(A, \leq_A)$ and $(B, \leq_B)$ are both
        in $\mathcal C$, they are comparable, so that we either have $a \leq_A
        b$ and $b \leq_A a$, or $a \leq_B b$ and $b \leq_B a$.  In either case,
        the result follows from antisymmetry.
    \end{case}
    \begin{case} There exists an $(A, \leq_A) \in \mathcal C$ such that $a \in
    A$, $b \in A$, and $a \leq_A b$, but there exists no $(B, \leq_B) \in
    \mathcal C$ such that $a \in B$ and $b \in B$. \\
        \indent This case is impossible.
    \end{case}
    \begin{case} There exists no $(A, \leq_A) \in \mathcal C$ such that $a \in
    A$ and $b \in A$, but there exists a $(B, \leq_B) \in \mathcal C$ such that
    $a \in B$ and $b \in B$, and $a \leq_B b$. \\
        \indent This case is impossible.
    \end{case}
    \begin{case} There exists no $(A, \leq_A) \in \mathcal C$ such that $a \in
    A$ and $b \in A$, and there exists no $(B, \leq_B) \in \mathcal C$ such that
    $a \in B$ and $b \in B$. \\
        \indent Because $a \in F$ we have an $(A, \leq_A) \in \mathcal C$ such
        that $a \in A$.  Likewise, because $b \in F$, we have a $(B, \leq_B) \in
        \mathcal C$ such that $b \in B$.  Because $F$ is a chain, $(A, \leq_A)$
        and $(B, \leq_B)$ are comparable.  Assume without loss of generality
        that $(A, \leq_A) \leq (B, \leq_B)$.  Then we have both $a \in B$ and $b
        \in B$, contradicting the premise of this case.  Thus, this case is also
        impossible.
    \end{case}
    \noindent All cases have been checked, so $\leq_F$ is antisymmetric in $F$.

    To prove that $F$ well-orders $\U$ under $\leq_F$, let $S$ be a subset of
    $F$ and $x$ be an element of $S$.  This means that $x \in F$, so there
    exists an $(A, \leq_A) \in \mathcal C$ such that $x \in A$.  Then $A \cap S$
    is a nonempty subset of $S$, so it has a least element $a$.  We will prove
    that $a$ is the least element of $A$.  Let $b \in S$.  If $b \in A$, then $a
    \leq_A b$ by $a$ being the least element of $A \cap F$, so $a \leq_F b$.  If
    $b \notin A$, because $b \in S$, there exists a $(B, \leq_B) \in \mathcal C$
    such that $b \in B$.  Because they are in $\mathcal C$, $(A, \leq_A)$ and
    $(B, \leq_B)$ are comparable.  If $(B, \leq_B) \leq (A, \leq_A)$, then $b
    \in A$, so $a \leq_A b$ by $a$ being the least element of $A \cap F$, so $a
    \leq_F b$.  If $(A, \leq_A) \subseteq (B, \leq_B)$, by the extension
    condition, we have $a \leq_B b$, so $a \leq_F b$.  In all cases, $a \leq_F
    b$, so $a$ is the minimum element of $S$.

    For the last condition for $(F, \leq_F) \in \mathcal C$, let $a \notin F$
    and let $b$ be arbitrary.  Because $a \notin F$, there exists no $(A,
    \leq_A) \in \mathcal C$ such that $a \in A$ and $b \in A$.  This means that
    the second condition of $\leq_F$ is true for both $a \leq_F b$ and $b \leq_F
    a$.

    We have now shown that $(F, \leq_F) \in \mathcal F$.  We will now show that
    it is an upper bound of $\mathcal C$.  Let $(A, \leq_A)$ be an element of
    $\mathcal C$.  $A \subseteq F$ is trivial by the definition of $F$.  To
    prove the equality condition, note that when we have an $a \in A$ and a $b
    \in A$ with $a \leq_A b$, we have one of the possible conditions for $a
    \leq_F b$.  To prove the extension condition, let $a \in A$ and $b \notin
    A$.  If there did not exist a $(B, \leq_B) \in \mathcal C$ such that $a \in
    B$ and $b \in B$, we would have $a \leq_F b$ because the second condition is
    satisfied.  So assume there is a $(B, \leq_B) \in \mathcal C$ such that $a
    \in B$ and $b \in B$.  Because $(A, \leq_A)$ and $(B, \leq_B)$ are both in
    $\mathcal C$, they are comparable.  If $(A, \leq_A) \leq (B, \leq_B)$, then
    by the extension condition we have $a \leq_B b$, proving $a \leq_F b$.  If
    $(B, \leq_B) \leq (A, \leq_A)$, then we would have $b \in A$, which
    contradicts $b \notin A$.  Every case has been accounted for, so $\leq_F$
    satisfies the extension condition.  Because $\leq_F$ satisfies all of the
    conditions, $(A, \leq_A) \leq (F, \leq_F)$.

    Thus, $(F, \leq_F)$ is an upper bound of $\mathcal C$, showing that every
    chain in $\mathcal F$ has an upper bound.  Because every chain has an upper
    bound, by Zorn's lemma there exists a maximal element $(W, \leq_W) \in
    \mathcal F$.  We will now prove that $W = \bm U$.

    Assume that $W \neq \bm U$, that is, that there is some $x \notin W$.  Then
    define the set $X = W \cup \{x\}$ and a new relation $\leq_X$ defined
    piecewise by
    \[
        a \leq_X b =
        \begin{cases}
            a \leq_W b  & \quad \text{if $a \in W$ and $b \in W$} \\
            x \neq a    & \quad \text{if $a \notin W$ and $b \in W$} \\
            \verb|True| & \quad \text{if $b \notin W$.}
        \end{cases}
    \]
    We will prove that $(X, \leq_X) \in \mathcal F$.

    To prove that $\leq_X$ is antisymmetric in $X$, let $a$ and $b$ be elements
    of $X$ with $a \leq_X b$ and $b \leq_X a$.  Now if $a$ and $b$ are in $W$,
    then they are equal by the antisymmetry of $\leq_W$.  If they are both not
    in $W$, then they are both equal to $x$ and are equal.  If $a \in W$ and $b
    \notin W$, this means that $x \neq b$ (from $b \leq_X a$) and $x = b$ (from
    $b \notin W$), so this case is impossible.  The case $a \notin W$ and $b \in
    W$ is similar.  Thus, all cases have been accounted for, so $\leq_X$ is
    antisymmetric in $X$.

    To prove that $X$ well-orders $\U$ under $\leq_X$, let $S$ be a subset of
    $X$ and let $x$ be an element of $S$.  Now we have two cases: either $S =
    \{x\}$, or there exists something in $S$ that is in $W$.  When $S = \{x\}$,
    $\{x\}$ is the least element of $S$ because $x \notin W$ implies that $x
    \leq_X x = \verb|True|$.  When an element $y'$ is in both $S$ and $W$,
    because $W$ well-orders $\leq_W$, there is a least element $a$ in $S \cap
    W$.  Then $a$ is a least element in $S$, because for any other $b \in S$, if
    $b \in W$, then $a \leq b$ by $a$ being the least element in $S \cap W$, and
    if $b \notin W$, then $a \leq_X b$ by definition.  Thus, $X$ well-orders
    $\U$ under $\leq_X$.

    For the final condition for $(X, \leq_X) \in \mathcal F$, let $a \notin X$
    and $b$ be an arbitrary element.  Because $a \notin X$, we know that $a
    \notin W$ and $a \neq x$.  Because $a \notin W$, we easily have $b \leq_X
    a$.  If $b \notin W$, we again would easily have $a \leq_X b$, and if $b \in
    W$, then $a \leq_X b$ because $a \neq x$.  We have seen that $a \leq_X b$
    and $b \leq_X a$, so $(X, \leq_X)$ satisfies the third condition of being in
    $\mathcal F$.

    Thus, $(X, \leq_X) \in \mathcal X$.  We will now prove that $(W, \leq_W)
    \leq (X, \leq_X)$.  $W \subseteq W \cup \{x\} = X$ is trivial.  Equality and
    extension follow directly from the definition of $X$.  However, $(W, \leq_W)
    \neq (X, \leq_X)$.  For if it was, then $x \in W$, contradicting our
    assumption earlier that $x \notin W$.  Thus, $(W, \leq_W) < (X, \leq_X)$.
    But this contradicts the maximality of $(W, \leq_W)$, showing that our
    assumption much earlier that $x \notin W$ is false.  Thus, $x \in W$,
    showing that $W = \bm U$.

    We now have a relation defined on $\U$ that is antisymmetric and
    well-ordered on all of $\U$.  By Instances \ref{wo_connex} and
    \ref{wo_trans}, this is enough to prove that $\U$ is well-ordered.
\end{proof}

Note: While trying to find simpler proofs of the well-ordering theorem (yes, the
above three pages are simpler than the previous iterations were), I came up with
a proof that I don't know if I've seen before so I want to at least mention the
idea here.  The idea is to take the collection $\mathcal C$ of collections
$\mathcal W$ of $\U$ such that the subset relation of all elements in $\mathcal
W$ is well-ordered, and that for all $x$ in the union of a collection, there
exists some set $S$ in the collection with $x \in S$, $S \in \mathcal W$, and
$(S - \{x\}) \in \mathcal W$.  You can then apply Zorn's lemma to this set of
collections ordered by inclusion, and then by maximality every element in $\U$
is in some set in the maximal collection, and then you define $a \leq b$ to mean
that the minimal set containing $a$ is a subset of the minimal set containing
$b$.  This does work, and it produces a proof that is roughly the same length as
the one presented here.  However, it is conceptually much more confusing, so I
opted to go with this proof.

\end{document}
