\documentclass[../../math.tex]{subfiles}
\externaldocument{../../math.tex}
\externaldocument{set}

\begin{document}

\setcounter{chapter}{4}

\chapter{Lists}

\begin{definition}
    Let $\A$ be a type.  Then we can define a new inductive type $\L(\A)$ with
    two constructors: a value constructor $[]$, and a constructor that takes in
    a value $x : \A$ and a value $a : \L(\A)$, denoted $x : a$.  We call a
    value in $\L(\A)$ a list of $\A$, and we call $[]$ the empty list.  We use
    the notation $[a, b, \cdots, n]$ to mean $a : b : \cdots : n : []$.
\end{definition}

By the properties of inductive types, we can perform induction on lists.  Note
that we are not performing induction on the size of the list.  We are performing
induction on the list itself.  This will be an important distinction to make
when we are talking about the size of the list.

Note that by the properties of constructors, we have $x : a \neq []$.  We also
know that if $x_1 : a_1 = x_2 : a_2$, then $x_1 = x_2$ and $a_1 = a_2$.

Throughout this chapter, let $\U$ be a type.  Unless said otherwise, all values
that are not lists will be assumed to be of this type.

\section{Basic Operations}

\begin{instance}
    Let $a$ and $b$ be two lists.  We define their sum $a + b$ recursively on
    $a$:
    \begin{align*}
              [] + b &= b \\
        (x : a') + b &= x : (a' + b).
    \end{align*}
\end{instance}
\noindent Notice that addition is concatenation of lists.

\begin{definition}
    Let $a$ be a list.  Then we define its reverse $a^{-1}$ recursively:
    \begin{align*}
              []^{-1} &= [] \\
        (x : a')^{-1} &= a'^{-1} + [x].
    \end{align*}
\end{definition}

\begin{definition}
    Let $\A$ and $\B$ be types, and $f$ a function from $\A$ to $\B$.  Then for
    a list $a : \L(\A)$, we can define a new list $f(a) : \L(\B)$ recursively:
    \begin{align*}
            f([]) &= [] \\
        f(x : a') &= f(x) : f(a').
    \end{align*}
\end{definition}

\begin{instance}
    Define $0 = []$.  Note that we will still use the notation $[]$ instead of
    $0$.
\end{instance}

\begin{instance}
    $[]$ is a left additive identity.
\end{instance}
\begin{proof}
    This follows directly from the definition of $+$.
\end{proof}

\begin{theorem}
    For all values $x$ and lists $a$,
    \[
        [x] + a = x : a.
    \]
\end{theorem}
\begin{proof}
    \[
        [x] + a = (x : []) + a = x : ([] + a) = x : a.
    \]
\end{proof}

\begin{instance}
    $[]$ is a right additive identity.
\end{instance}
\begin{proof}
    The proof will be by induction on $a$.  When $a = []$, then $[] + [] = []$.
    Now assume that $a + [] = a$.  We must prove that for all $x$, $(x : a) + []
    = x : a$.  This is true because
    \[
        (x : a) + [] = x : (a + []) = x : a,
    \]
    where the last equality follows from the inductive hypothesis.
\end{proof}

\begin{instance}
    Concatenation of lists is associative.
\end{instance}
\begin{proof}
    Let $a$, $b$, and $c$ be lists.  We will prove that $a + (b + c) =
    (a + b) + c$ by induction on $a$.  First, when $a = []$, then
    \[
        [] + (b + c) = b + c = ([] + b) + c,
    \]
    so the base case is true.  Now assume that $a + (b + c) = (a +
    b) + c$, and let $x$ be any value.  Then
    \begin{align*}
        &(x : a) + (b + c) \\
        ={}& x : (a + (b + c) \\
        ={}& x : ((a + b) + c) \\
        ={}& (x : (a + b)) + c \\
        ={}& ((x : a) + b) + c,
    \end{align*}
    showing that the inductive case is true.  Thus, by induction, concatenation
    of lists is associative.
\end{proof}

\begin{instance}
    Concatenation of lists is left cancellative.
\end{instance}
\begin{proof}
    Let $a$, $b$, and $c$ be lists such that $c + a = c + b$.  The proof will be
    by induction on $c$.  When $c = []$, we have
    \[
        a = [] + a = [] + b = b,
    \]
    so the base case is true.  Now assume that if $c + a = c + b$, then $a = b$.
    Let $x$ be any value.  Then
    \begin{align*}
        (x : c) + a &= (x : c) + b \\
        x : (c + a) &= x : (c + b) \\
        c + a &= c + b,
    \end{align*}
    so the result follows from the inductive hypothesis.  Thus, by induction,
    concatenation of lists is left cancellative.
\end{proof}

\begin{theorem}
    For all values $x$, $[x]^{-1} = [x]$.
\end{theorem}
\begin{proof}
    \[
        [x]^{-1} = (x : [])^{-1} = [] + [x] = [x].
    \]
\end{proof}

\begin{theorem}
    For all lists $a$ and $b$, $(a + b)^{-1} = b^{-1} + a^{-1}$.
\end{theorem}
\begin{proof}
    The proof will be by induction on $a$.  When $a = []$, we have
    \[
        ([] + b)^{-1} = b^{-1} = b^{-1} + [],
    \]
    so the base case is true.  Now assume that $(a + b)^{-1} = b^{-1} +
    a^{-1}$.  Then for all $x : \U$,
    \begin{align*}
           & ((x : a) + b)^{-1} \\
        ={}& (x : (a + b))^{-1} \\
        ={}& (a + b)^{-1} + [x] \\
        ={}& (b^{-1} + a^{-1}) + [x] \\
        ={}& b^{-1} + (a^{-1} + [x]) \\
        ={}& b^{-1} + (x : a)^{-1},
    \end{align*}
    so the inductive case is true.  Thus, by induction, $(a + b)^{-1} =
    b^{-1} + a^{-1}$ for all lists $a$ and $b$.
\end{proof}

\begin{theorem} \label{list_reverse_reverse}
    For all lists $a$, $(a^{-1})^{-1} = a$.
\end{theorem}
\begin{proof}
    The proof will be by induction on $a$.  When $a = []$, we have
    $([]^{-1})^{-1} = []^{-1} = []$, so the base case is true.  Now assume that
    $(a^{-1})^{-1} = a$ and let $x$ be any value of $\U$.  Then
    \begin{align*}
           & ((x : a)^{-1})^{-1} \\
        ={}& (a^{-1} + [x])^{-1} \\
        ={}& [x]^{-1} + (a^{-1})^{-1} \\
        ={}& [x] + a \\
        ={}& x : a,
    \end{align*}
    se the inductive case is true.  Thus, by induction $(a^{-1})^{-1} = a$ for
    all lists $a$.
\end{proof}

\begin{theorem}
    For all lists $a$ and $b$, $a = b$ if and only if $a^{-1} = b^{-1}$.
\end{theorem}
\begin{proof}
    The forward direction is trivial.  For the reverse direction, assume that
    $a^{-1} = b^{-1}$.  Then we can apply the reverse to both sides to get
    $(a^{-1})^{-1} = (b^{-1})^{-1}$.  By the previous theorem this means that $a
    = b$, as required.
\end{proof}

\begin{instance}
    Concatenation of lists is right cancellative.
\end{instance}
\begin{proof}
    Let $a$, $b$, and $c$ be lists such that $a + c = b + c$.  Then
    \begin{align*}
        a + c &= b + c \\
        (a + c)^{-1} &= (b + c)^{-1} \\
        c^{-1} + a^{-1} &= c^{-1} + b^{-1} \\
        a^{-1} &= b^{-1} \\
        a &= b.
    \end{align*}
\end{proof}

\begin{theorem}
    For all lists $a$, if $a^{-1} = []$, then $a = []$ as well.
\end{theorem}
\begin{proof}
    Because $[]^{-1} = []$, we have $a^{-1} = []^{-1}$, and the result follows
    from the previous theorem.
\end{proof}

\begin{theorem}
    For all $x : \A$, $f([x]) = [f(x)]$.
\end{theorem}
\begin{proof}
    \[
        f([x]) = f(x : []) = f(x) : [] = f(x).
    \]
\end{proof}

\begin{theorem}
    For all lists $a$ and $b$, $f(a + b) = f(a) + f(b)$.
\end{theorem}
\begin{proof}
    The proof will be by induction on $a$.  When $a = []$,
    \[
        f([] + b) = f(b) = [] + f(b) = f([]) + f(b),
    \]
    so the base case is true.  Now assume that $f(a + b) = f(a) + f(b)$.
    Then for all $x : \A$,
    \begin{align*}
           & f((x : a) + b) \\
        ={}& f(x : (a + b)) \\
        ={}& f(x) : f(a + b) \\
        ={}& f(x) : f(a) + f(b) \\
        ={}& (f(x) : f(a)) + f(b) \\
        ={}& f(x : a) + f(b),
    \end{align*}
    so the inductive hypothesis is true.  Thus, $f(a + b) = f(a) + f(b)$
    by induction.
\end{proof}

\begin{theorem} \label{list_image_comp}
    Let $\A$, $\B$, and $\C$ be types, and $f$ a function from $\A$ to $\B$ and
    $g$ a function from $\B$ to $\C$.  Then for all lists $a : \L(\A)$,
    \[
        g(f(a)) = (g \circ f)(a).
    \]
\end{theorem}
\begin{proof}
    The proof will be by induction on $a$.  When $a = []$,
    \[
        g(f([])) = g([]) = [] = (g \circ f)([]),
    \]
    so the base case is true.  Now assume that $g(f(a)) = (g \circ f)(a)$.  Then
    for all $x : \A$,
    \begin{align*}
           & g(f(x : a)) \\
        ={}& g(f(x) : f(a)) \\
        ={}& g(f(x)) : g(f(a)) \\
        ={}& (g \circ f)(x) : (g \circ f)(a) \\
        ={}& (g \circ f)(x : a),
    \end{align*}
    so the inductive hypothesis is true.  Thus, $g(f(a)) = (g \circ f)(a)$ by
    induction.
\end{proof}

\section{Values in Lists}

\begin{definition}
    Let $a$ be a list and $x$ be a value.  Then we define the $x \in a$
    recursively on $a$:
    \begin{align*}
              x \in [] &= \vtt{False} \\
        x \in (y : a') &= (x = y \vee x \in a').
    \end{align*}
\end{definition}

\begin{definition}
    Let $a$ be a list.  Then we define $a$ having unique elements recursively:
    \begin{align*}
            \text{$[]$ has unique elements} &= \vtt{True} \\
        \text{$x : a'$ has unique elements} &= x \notin a' \wedge \text{$a'$ has
        unique elements.}
    \end{align*}
\end{definition}
\noindent Notice that because $(a : l)$ having unique elements implies that $l$
has unique elements, when doing induction on a list, it having unique elements
does not need to be considered as a part of the inductive hypothesis.

\begin{theorem}
    For all elements $x$ and $y$, $x \in [y] \leftrightarrow x = y$.
\end{theorem}
\begin{proof}
    \[
        x \in [y] \leftrightarrow x = y \vee x \in []
        \leftrightarrow x = y \vee \vtt{False}
        \leftrightarrow x = y.
    \]
\end{proof}

\begin{theorem} \label{in_list_conc}
    For all lists $a$ and $b$ and for all values $x$, if $x \in (a + b)$, then
    either $x \in a$ or $x \in b$.
\end{theorem}
\begin{proof}
    Assume that $x \notin b$.  Then the proof will be by induction on $a$.  When
    $a = []$, then we have $x \in b$, contradicting $x \notin b$, so the base
    case is true.  Now assume that if $x \in (a + b)$, then $x \in a$, and
    let $y$ be a value such that $x \in (y : a + b)$.  We must prove that $x \in
    (y : a)$.  If $x = y$, then we have $x \in (y : a)$.  If $x \neq y$, then we
    have $x \in (a + b)$, so by the inductive hypothesis we have $x \in a$.
    Thus, $x \in (y : a)$.  Either way, $x \in (y : a)$, showing that the
    theorem is true by induction.
\end{proof}

\begin{theorem} \label{in_list_rconc}
    For all lists $a$ and $b$ and values $x$, if $x \in b$, then $x \in (a +
    b)$.
\end{theorem}
\begin{proof}
    The proof will be by induction on $a$.  When $a = []$, then $x \in b$ by
    hypothesis, so the base case is true.  Now assume that $x \in (a + b)$.  We
    must prove that $x \in (y : a + b)$ for any value $y$.  But $x \in (a + b)$
    is precisely one of the ways that we can have $x \in (y : a + b)$, so the
    result is trivial.  Thus, the theorem is true by induction.
\end{proof}

\begin{theorem} \label{in_list_lconc}
    For all lists $a$ and $b$ and values $x$, if $x \in a$, then $x \in (a +
    b)$.
\end{theorem}
\begin{proof}
    The proof will be by induction on $a$.  When $a = []$, we $x \in []$, a
    contradiction.  Thus the base case is true vacuously.  Now assume that if $x
    \in a$, then $x \in (a + b)$.  We must prove that for some value $y$, if $x
    \in (y : a)$, then $x \in (y : a + b)$.  Now if $x = y$, then we have $x \in
    (y : a + b)$.  If $x \neq y$, then we have $x \in a$, so by the inductive
    hypothesis we have $x \in (a + b)$, showing that $x \in (y : a + b)$.  Thus,
    the theorem is true by induction.
\end{proof}

\begin{theorem} \label{in_list_comm}
    For all lists $a$ and $b$ and values $x$, if $x \in (a + b)$, then $x \in (b
    + a)$.
\end{theorem}
\begin{proof}
    By Theorem \ref{in_list_conc}, we have either $x \in a$ or $x \in b$.  If $x
    \in a$, the result follows from Theorem \ref{in_list_lconc}, and if $x \in
    b$, the result follows from Theorem \ref{in_list_rconc}.
\end{proof}

\begin{theorem} \label{in_list_split}
    For all lists $a$ and values $x$, in $x \in a$, then there exist lists $b$
    and $c$ such that $a = b + x : c$.
\end{theorem}
\begin{proof}
    The proof will be by induction on $a$.  $x \in []$ is a contradiction, so
    the base case is vacuously true.  Now assume that if $x \in a$, then there
    exist lists $b$ and $c$ such that $a = b + x : c$, and assume that $x \in
    (y : a)$ for some $y$.  We must prove that there exist lists $b$ and $c$
    such that $y : a = b + x : c$.  If $x = y$, then we have $y : a = [] + x :
    a$ as required.  If $x \neq y$, then $x \in a$, so by the inductive
    hypothesis we have lists $b$ and $c$ such that $a = b + x : c$, so $y : a
    = y : b + x : c$.  Either way, the inductive step is true, so the theorem is
    true by induction.
\end{proof}

\begin{theorem} \label{in_list_image}
    For all lists $a$, values $x$, and functions $f : \U \to \V$, if $x \in a$,
    then $f(x) \in f(a)$.
\end{theorem}
\begin{proof}
    The proof will be by induction on $a$.  $x \in []$ is a contradiction, so
    the base case is vacuously true.  Now assume that if $x \in a$, then $f(x)
    \in f(a)$, and that $x \in y : a$ for some value $y$.  We must prove that
    $f(x) \in f(y : a) = f(y) : f(a)$.  If $x = y$, then $f(x) \in f(y) : f(a)$.
    If $x \neq y$, then $x \in a$, so by the inductive hypothesis, $f(x) \in
    f(a)$, so $f(x) \in f(y) : f(a)$.  Either way, the inductive step is true,
    so the theorem is true by induction.
\end{proof}

\begin{theorem} \label{image_in_list}
    For all lists $a$, values $y$, and functions $f : \U \to \V$, if $y \in
    f(a)$, then there exists an $x \in a$ such that $f(x) = y$.
\end{theorem}
\begin{proof}
    The proof will be by induction on $a$.  When $a = []$, $y \in f([]) = []$ is
    a contradiction, so the base case is vacuously true.  Now assume that if $y
    \in f(a)$, then there exists an $x \in a$ such that $f(x) = y$, and assume
    that there is some value $z$ such that $y \in f(z : a)$.  This means that
    either $f(z) = y$ or $y \in f(a)$.  If $f(z) = y$, then $z$ itself is an
    element $x$ such that $f(x) = y$ and $x \in (z : a)$.  If $y \in f(a)$, then
    by the inductive hypothesis, there is some $x$ such that $f(x) = y$ and $x
    \in l$, so $x \in (z : a)$ as well.  Either way, the inductive step is true,
    so the theorem is true by induction.
\end{proof}

\begin{theorem}
    For all values $a$, $[a]$ has unique elements.
\end{theorem}
\begin{proof}
    $[a]$ having unique elements is the same as $a \notin []$ and $[]$ having
    unique elements.  Both of these are trivial.
\end{proof}

\begin{lemma} \label{list_unique_comm_add}
    For all lists $a$ and values $x$, if $x : a$ has unique elements, then $a +
    [x]$ has unique elements.
\end{lemma}
\begin{proof}
    Because $x : a$ has unique elements, $x \notin a$ and $a$ has unique
    elements.  The proof will be by induction on $a$.  When $a = []$, then $[] +
    [x] = [x]$, which has unique elements, so the base case is true.  Now assume
    that if $x \notin a$, then $(a + [x])$ has unique elements, and assume that
    there is a $y$ such that $y \notin (x : a)$.  This means that $y \neq x$ and
    $y \notin a$.  We must prove that $y : a + [x]$ has unique elements, that
    is, that $y \notin (a + [x])$ and that $(a + [x])$ has unique elements.  The
    latter follows from the inductive hypothesis.  To prove that $y \notin (a +
    [x])$, assume that $y \in (a + [x])$.  Then by Theorem \ref{in_list_conc},
    either $y \in a$ or $y \in [x]$.  If $y \in a$, that contradicts $y \notin
    a$ from before.  If $y \in [x]$, then $y = x$ contradicting $y \neq x$ from
    before.  Both cases have a contradiction, so our assumption that $y \in (a +
    [x])$ was false, meaning that $y \notin (a + [x])$.  Thus, the inductive
    step is true, so by induction, the theorem is true.
\end{proof}

\begin{theorem} \label{list_unique_comm}
    For all lists $a$ and $b$, $a + b$ has unique elements if and only if $b +
    a$ has unique elements.
\end{theorem}
\begin{proof}
    By symmetry we only need to prove the forward implication.  We will use
    induction on $a$ to prove the statement ``For all lists $b$, if $a + b$ has
    unique elements, then $b + a$ has unique elements.''  First, when $a = []$,
    then when $[] + b = b$ has unique elements, then $b + [] = b$ has unique
    elements, so the base case is true.

    Now assume that for all lists $b$, if $a + b$ has unique elements, then $b +
    a$ has unique elements.  We need to prove that if $x : a + b$ has unique
    elements, then $b + x : a$ has unique elements.  By Lemma
    \ref{list_unique_comm_add}, we know that $a + b + [x] = a + (b + [x])$ has
    unique elements.  Then by the inductive hypothesis, $(b + [x]) + a = b + x :
    a$ has unique elements.  Thus, by induction, the theorem is true.
\end{proof}

\begin{theorem} \label{list_unique_lconc}
    For all lists $a$ and $b$, if $a + b$ has unique elements, then $a$ has
    unique elements.
\end{theorem}
\begin{proof}
    The proof will be by induction on $a$.  $[]$ has unique elements, so the
    base case is true.  Now assume that $a$ has unique elements and let $x$ be
    such that $x : a + b$ has unique elements.  Because $x : a + b$ has unique
    elements, we know that $x \notin a + b$.  By the contrapositive of Theorem
    \ref{in_list_lconc}, we know that $x \notin a$.  Thus, $x \notin a$ and $a$
    has unique elements, showing that $x : a$ has unique elements.  Thus, the
    theorem is true by induction.
\end{proof}

\begin{theorem} \label{list_unique_rconc}
    For all lists $a$ and $b$, if $a + b$ has unique elements, then $b$ has
    unique elements.
\end{theorem}
\begin{proof}
    This is just the combination of Theorems \ref{list_unique_lconc} and
    \ref{list_unique_comm}.
\end{proof}

\begin{theorem} \label{list_unique_conc}
    For all lists $a$ and $b$, if $a + b$ has unique elements, then for every $x
    \in a$, we have $x \notin b$.
\end{theorem}
\begin{proof}
    Assume that there is an $x$ such that $x \in a$ and $x \in b$.  We will
    derive a contradiction by induction on $a$.  $x \in []$ is a contradiction,
    so the base case is true.  Now assume that $x \notin a$, and that there's a
    $y$ such that $x \in y : a$ and $y : a + b$ has unique elements.  Because $y
    : a + b$ has unique elements, we have $y \notin (a + b)$.  Because $x \in y
    : a$, we have two cases: when $x = y$, and when $x \in a$.  When $x = y$, we
    have $x \in b$, and by Theorem \ref{in_list_rconc} we have $x \in (a + b)$,
    contradicting $x \notin (a + b)$.  When $x \in a$, this directly
    contradicts the inductive hypothesis.  Thus, by induction, we have a
    contradiction, meaning that we can't have both $x \in a$ and $x \in b$.
\end{proof}

\begin{theorem} \label{list_image_unique}
    For all lists $a$ and functions $f : \U \to \V$, if $f(a)$ has unique
    elements, then $a$ does as well.
\end{theorem}
\begin{proof}
    The proof will be by induction on $a$.  When $a = []$, it has unique
    elements, so the base case is true.  Now assume that $a$ has unique
    elements.  We must prove that if $f(x : a) = f(x) : f(a)$ has unique
    elements, then $x : a$ has unique elements.  Because $f(x) : f(a)$ has
    unique elements, we know that $f(x) \notin f(a)$ and that $f(a)$ has unique
    elements.  Because $f(x) \notin f(a)$, by the contrapositive of Theorem
    \ref{in_list_image}, we know that $x \notin a$.  By the inductive
    hypothesis, $a$ has unique elements.  Thus, $x : a$ has unique elements, so
    the theorem is true by induction.
\end{proof}

\begin{theorem} \label{list_image_unique_inj}
    For all lists $a$ and injective functions $f : \U \to \V$, if $a$ has unique
    elements, then $f(a)$ does as well.
\end{theorem}
\begin{proof}
    The proof will be by induction on $a$.  When $a = []$, $f([]) = []$ trivial
    has unique elements, so the base case is true.  Now assume that $a$ has
    unique elements, $f(a)$ has unique elements, and $x \notin a$ for some value
    $x$.  We must prove that $f(x : a)$ has unique elements.  Because $f(a)$ has
    unique elements, we just need to prove that $f(x) \notin f(a)$.  If $f(x)
    \in f(a)$, be Theorem \ref{image_in_list} we would have some $y \in a$ such
    that $f(x) = f(y)$.  Because $f$ is injective, we have $x = y$, but because
    $x \notin a$ and $y \in a$ we have a contradiction.  Thus, $f(x) \notin
    f(a)$, showing that the theorem is true by induction.
\end{proof}

\section{Lists and Sets}

\begin{definition}
    Let $a$ be a list, and let $S$ be a set.  We will define the filtered list
    $a^S$ recursively:
    \begin{align*}
             []^S &= [] \\
        (x : a)^S &= \text{If $x \in S$ then $x : a^S$ else $a^S$}.
    \end{align*}
\end{definition}

\begin{definition} \label{list_prop}
    Let $a$ be a list, and let $S$ be a set.  We will define $a \in S$, meaning
    that every element in $a$ is in $S$, recursively:
    \begin{align*}
           [] \in S &= \vtt{True} \\
        x : a \in S &= x \in S \wedge a \in S.
    \end{align*}
\end{definition}
\noindent Notice that like a list having unique elements, because $(a : l) \in
S$ implies that $l \in S$, when doing induction on a list, it being in $S$ does
not need to be considered as a part of the inductive hypothesis.

\begin{definition}
    Given a relation $\usim$, let $x \usim$ represent the set of all elements
    $y$ such that $x \sim y$.  Then for a list $a$, we define $a \in \usim$
    recursively:
    \begin{align*}
           [] \in \usim &= \vtt{True} \\
        x : a \in \usim &= a \in (x \usim) \wedge a \in \usim.
    \end{align*}
    Note that in the last expression, the first $\in$ is being used in the sense
    of definition \ref{list_prop}, while the second $\in$ is the recursive one.
\end{definition}

Note that this definition only checks that elements are related in one
direction, and that elements are not checked against themselves.  To be honest
part of this reason is that this definition is used in a grand total of one
situations (at least currently), in a place where the relation is symmetric and
shouldn't be checked against itself.  However, you can derive other forms that
may be more suitable for other situations from the last two definitions, so not
much is lost by using this definition.

It is important to note that we now have four different usages of $\in$.
Context can distinguish between them all.  Given a value $x$, a list $a$, a set
$S$, and a relation $\usim$, $x \in S$ is traditional set membership, $x \in a$
checks if $x$ occurs in the list $a$, $a \in S$ checks if every elements of $a$
is in $S$, and $a \in \usim$ checks if all elements of $a$ are related to each
other.  This confusion won't be as bad in future chapters because these concepts
will usually just be written in English rather than using this notation (e.g.
explicitly saying that every value in a list satisfies a predicate).

\begin{theorem}
    For all sets $S$ and values $x$, if $x \in S$, then $[x]^S = [x]$.
\end{theorem}
\begin{proof}
    \[
        [x]^S = x : [] = [x].
    \]
\end{proof}

\begin{theorem}
    For all sets $S$ and values $x$, if $x \in S$, then $[x]^S = []$.
\end{theorem}
\begin{proof}
    Trivial.
\end{proof}

\begin{theorem}
    For all sets $S$ and lists $a$ and $b$, $(a + b)^S = a^S + b^S$.
\end{theorem}
\begin{proof}
    The proof will be by induction on $a$.  When $a = []$,
    \[
        ([] + b)^S = b^S = [] + b^S = []^S + b^S,
    \]
    so the base case is true.  Now assume that $(a + b)^S = a^S + b^S$.  We must
    prove that for any $x$, $(x : a + b)^S = (x : a)^S + b^S$.  There are two
    cases: when $x \in S$, and when $x \notin S$.  When $x \in S$,
    \[
        (x : a + b)^S = x : (a + b)^S = x : a^S + b^S = (x : a)^S + b^S.
    \]
    When $x \notin S$,
    \[
        (x : a + b)^S = (a + b)^S = a^S + b^S = (x : a)^S + b^S.
    \]
    Either way, the inductive step is true, so the theorem is true by induction.
\end{proof}

\begin{lemma} \label{list_filter_in_both}
    For all sets $S$, lists $a$, and values $x$, if $x \in a^S$, then $x \in a$
    and $x \in S$.
\end{lemma}
\begin{proof}
    The proof will be by induction on $a$.  When $a = []$, $x \in []^S = []$ is
    a contradiction, so the base case is vacuously true.  Now assume that for
    any $x \in a^S$, we have $x \in a$ and $x \in S$, and that $x \in (y : a)^S$ for some
    value $y$.  We must prove that $x \in y : a$ and that $x \in S$.  We have two cases: when $y
    \in S$, and when $y \notin S$.  When $y \notin S$, we have $x \in a^S$, so
    by the inductive hypothesis we have $x \in S$ and $x \in a$, meaning that $x
    \in (y : a)$.  When $y \in S$, we have $x \in y : a^S$, so we have two more
    cases here: when $x = y$, and when $x \in a^S$.  When $x = y$, we easily
    have $x \in y : a$ and $y \in S$.  When $x \in a^S$, by the inductive
    hypothesis we have $x \in S$ and $x \in a$, so $x \in y : a$.  The inductive
    step is true in all cases, so the theorem is true by induction.
\end{proof}

\begin{theorem} \label{list_filter_in}
    For all sets $S$, lists $a$, and values $x$, if $x \in a^S$, then $x \in a$.
\end{theorem}
\begin{proof}
    This is just a special case of Lemma \ref{list_filter_in_both}.
\end{proof}

\begin{theorem} \label{list_filter_in_set}
    For all sets $S$, lists $a$, and values $x$, if $x \in a^S$, then $x \in S$.
\end{theorem}
\begin{proof}
    This is just a special case of Lemma \ref{list_filter_in_both}.
\end{proof}

\begin{theorem} \label{list_filter_unique}
    For all sets $S$ and lists $a$, if $a$ has unique elements, then $a^S$ has
    unique elements.
\end{theorem}
\begin{proof}
    The proof will be by induction on $a$.  When $a = []$, then $[]^S = []$ has
    unique elements, so the base case is true.  Now assume that $a^S$ has unique
    elements and that $x : a$ has unique elements for some value $x$.  We must
    prove that $(x : a)^S$ has unique elements.  Because $x : a$ has unique
    elements, we know that $x \notin a$ and that $a$ has unique elements.  There
    are two cases: when $x \in S$, and when $x \notin S$.  When $x \in S$, we
    must prove that $x : a^S$ has unique elements.  $x \notin a^S$ follows from
    the contrapositive of Theorem \ref{list_filter_in}, and $a^S$ having unique
    elements is the inductive hypothesis.  When $x \notin S$, we must prove that
    $a^S$ has unique elements, which is the inductive hypothesis.  Either way,
    $(x : a)^S$ has unique elements, so the theorem is true by induction.
\end{proof}

\begin{theorem} \label{list_filter_image_in}
    For all sets $S : \set{\U}$, functions $f : \U \to \V$, lists $a$, and
    values $y : \V$, if $y \in f(a^S)$, then $y \in f(a)$.
\end{theorem}
\begin{proof}
    By Theorem \ref{image_in_list}, there is some $x \in a^S$ such that $f(x) =
    y$.  Because $x \in a^S$, we know that $x \in a$ by Theorem
    \ref{in_list_image}.  Thus, by Theorem \ref{list_filter_in}, $f(x) = y \in
    f(a)$.
\end{proof}

\begin{theorem} \label{list_filter_image_unique}
    For all sets $S : \set{\U}$, function $f : \U \to \V$, and lists $a$, if
    $f(a)$ has unique elements, then $f(a^S)$ has unique elements.
\end{theorem}
\begin{proof}
    The proof will be by induction on $a$.  When $a = []$, then $f([]^S) = f([])
    = []$ has unique elements, so the base case is true.  Now assume that
    $f(a^S)$ has unique elements, and let $x$ be a value such that $f(x : a)$
    has unique elements.  This means that $f(x) \notin f(a)$.  We must prove
    that $f((x : a)^S)$ has unique elements.  There are two cases: when $x \in
    S$, and when $x \notin S$.  When $x \in S$, we have $f((x : a)^S) = f(x :
    a^S) = f(x) : f(a^S)$.  $f(a^S)$ has unique elements by the inductive
    hypothesis, so we just need to prove that $f(x) \notin f(a^S)$, which
    follows from the contrapositive of Theorem \ref{list_filter_image_in}.  When
    $x \notin S$, we have $f((x : a)^S) = f(a^S)$, which has unique elements by
    the inductive hypothesis.  In both cases, $f((x : a)^S)$ has unique
    elements, so the theorem is true by induction.
\end{proof}

\begin{theorem} \label{list_filter_inter}
    For all sets $S$ and $T$ and lists $a$, $(a^S)^T = a^{S \cap T}$.
\end{theorem}
\begin{proof}
    The proof will be by induction on $a$.  When $a = []$, then $([]^S)^T = []^T
    = [] = []^{S \cap T}$, so the base case is true.  Now assume that $(a^S)^T =
    a^{S \cap T}$.  We must prove that for all values $x$, we have $((x :
    a)^S)^T = (x : a)^{S \cap T}$.  There are three cases: when $x \in S$ and $x
    \in T$, when $x \in S$ and $x \notin T$, and when $x \notin S$.  When $x \in
    S$ and $x \in T$, we also have $x \in S \cap T$, so
    \[
        ((x : a)^S)^T = (x : a^S)^T = x : (a^S)^T = x : a^{S \cap T}
        = (x : a)^{S \cap T}.
    \]
    When $x \in S$ and $x \notin T$, we also have $x \notin S \cap T$, so
    \[
        ((x : a)^S)^T = (x : a^S)^T = (a^S)^T = a^{S \cap T}
        = (x : a)^{S \cap T}.
    \]
    When $x \notin S$, we also have $x \notin S \cap T$, so
    \[
        ((x : a)^S)^T = (a^S)^T = a^{S \cap T} = (x : a)^{S \cap T}.
    \]
    All cases work, so the theorem is true by induction.
\end{proof}

\begin{theorem} \label{list_filter_filter}
    For all sets $S$ and lists $a$, $(a^S)^S = a^S$.
\end{theorem}
\begin{proof}
    This is just the combination of Theorem \ref{list_filter_inter} and Theorem
    \ref{inter_idemp}.
\end{proof}

\begin{theorem}
    For all sets $S$ and values $x$, $[x] \in S$ if and only if $x \in S$.
\end{theorem}
\begin{proof}
    \[
        [x] \in S \leftrightarrow x \in S \wedge [] \in S
        \leftrightarrow x \in S \wedge \vtt{True}
        \leftrightarrow x \in S.
    \]
\end{proof}

\begin{theorem} \label{list_prop_conc}
    For all sets $S$ and lists $a$ and $b$, $(a + b) \in S$ if and only if $a
    \in S$ and $b \in S$.
\end{theorem}
\begin{proof}
    The proof will be by induction on $a$.  When $a = []$, then
    \begin{align*}
        &([] + b) \in S \\
        \leftrightarrow{}&b \in S \\
        \leftrightarrow{}&\vtt{True} \wedge b \in S \\
        \leftrightarrow{}&[] \in S \wedge b \in S,
    \end{align*}
    so the base case is true.  Now assume that $(a + b) \in S$ if and only if $a
    \in S$ and $b \in S$.  Then for any value $x$,
    \begin{align*}
        &(x : a + b) \in S \\
        \leftrightarrow{}&x \in S \wedge (a + b) \in S \\
        \leftrightarrow{}&x \in S \wedge a \in S \wedge b \in S \\
        \leftrightarrow{}&(x : a) \in S \wedge b \in S,
    \end{align*}
    showing that the inductive step is true.  Thus, the theorem is true by
    induction.
\end{proof}

\begin{theorem} \label{list_prop_sub}
    For all sets $S$ and $T$ with $S \subseteq T$ and all lists $a$, if $a \in
    S$, then $a \in T$.
\end{theorem}
\begin{proof}
    The proof will be by induction on $a$.  When $a = []$, we trivially have $[]
    \in T$, so the base case is true.  Now assume that $a \in T$.  We must prove
    that for all values $x$ with $(x : a) \in S$, we also have $(x : a) \in T$.
    Because $x \in S$ and $S \subseteq T$, we have $x \in T$, and we have $a \in
    T$ by the inductive hypothesis.  Thus, $(x : a) \in T$.
\end{proof}

\begin{theorem} \label{list_prop_filter}
    For all lists $a$ and sets $S$, $a^S \in S$.
\end{theorem}
\begin{proof}
    The proof will be by induction on $a$.  When $a = []$, $[]^S = [] \in S$, so
    the base case is true.  Now assume that $a^S \in S$.  We must prove that for
    all values $x$, $(x : a)^S \in S$.  There are two cases: when $x \in S$, and
    when $x \notin S$.  When $x \in S$, we have $(x : a)^S = x : a^S$, and
    because $x \in S$ and $a^S \in S$ by the inductive hypothesis, we have $(x :
    a)^S \in S$.  When $x \notin S$, we have $(x : a)^S = a^S$, so by the
    inductive hypothesis we have $(x : a)^S \in S$.  Either way, $(x : a)^S \in
    S$, so the theorem is true by induction.
\end{proof}

\begin{theorem} \label{list_prop_in}
    For all lists $a$ and sets $S$, if $a \in S$, then for all $x \in a$, we
    have $x \in S$.
\end{theorem}
\begin{proof}
    The proof will be by induction on $a$.  $x \in []$ is impossible, so the
    base case is vacuously true.  Now assume that if $x \in a$, then $x \in S$.
    We must prove that for all values $y$ such that $(y : a) \in S$ and $x \in
    (y : a)$, we have $x \in S$.  Because $x \in (y : a)$, we have either $x =
    y$ or $x \in a$.  If $x = y$, then because $(y : a) \in S$, we have $x \in
    S$.  If $x \in a$, then $x \in S$ by the inductive hypothesis.  Either way,
    $x \in S$, so the theorem is true by induction.
\end{proof}

\begin{theorem} \label{list_in_prop}
    For all lists $a$ and sets $S$, if $x \in S$ whenever $x \in a$, then $a \in
    S$.
\end{theorem}
\begin{proof}
    The proof will be by induction on $a$.  When $a = []$, $[] \in S$, so the
    base case is true.  Now assume that if $x \in S$ whenever $x \in a$, we have
    $a \in S$.  We must prove that for some value $y$, if $x \in S$ whenever $x
    \in (y : a)$, we have $(y : a) \in S$.  Because $y \in y : a$, we have $y
    \in S$.  Now for any $x \in a$, we have $x \in (y : a)$, so $x \in S$.
    Thus, the condition for the inductive hypothesis is true, meaning that $a
    \in S$.  Thus, because $y \in S$ and $a \in S$, we have $(y : a) \in S$ as
    required, so the theorem is true by induction.
\end{proof}

\begin{theorem} \label{list_prop_in_sub}
    For all lists $a$ and $b$ and sets $S$, if for all $x \in a$ we have $x \in
    b$, and if $b \in S$, then $a \in S$.
\end{theorem}
\begin{proof}
    The proof will be by induction on $a$.  First $[] \in S$, so the base case
    is true.  Now assume that if for all $x \in a$ we have $x \in b$, then we
    have $a \in S$.  We must prove that for some value $y$ with $x \in (y : a)$
    implying $x \in b$, we have $(y : a) \in S$.  Because $y \in (y : a)$, we
    have $y \in b$, and because $b \in S$, by Theorem \ref{list_prop_in} we have
    $y \in S$.  Also, any $x \in a$ is $(y : a)$ as well, so we have that any $x
    \in a$ is in $b$.  Thus, by the inductive hypothesis, we have $a \in S$.
    Thus, $y \in S$ and $a \in S$, showing that $(y : a) \in S$, showing that
    the theorem is true by induction.
\end{proof}

\begin{theorem} \label{list_prop_other_filter}
    For all lists $a$ and sets $S$ and $T$, if $a \in S$, then $a^T \in S$.
\end{theorem}
\begin{proof}
    By Theorem \ref{list_prop_in_sub}, it suffices to prove that for all $x \in
    a^T$, we have $x \in a$, which follows directly from Theorem
    \ref{list_filter_in}.
\end{proof}

\begin{theorem} \label{list_prop2_single}
    For all relations $\usim$ and values $x$, $[x] \in \usim$.
\end{theorem}
\begin{proof}
    \[
        [x] \in \usim
        \leftrightarrow [] \in (x\usim) \wedge [] \in \usim
        \leftrightarrow \vtt{True} \wedge \vtt{True}
        \leftrightarrow \vtt{True}.
    \]
\end{proof}

\section{Lists and Natural Numbers}

\begin{definition}
    Let $a$ be a list.  Then we define its size $|a|$ (a natural number)
    recursively:
    \begin{align*}
            |[]| &= 0 \\
        |x : a'| &= 1 + |a'|.
    \end{align*}
\end{definition}

\begin{definition}
    Let $a$ be a list, and $x$ a value.  We define the count of $x$ in $a$,
    denoted $|a|_x$, recursively:
    \begin{align*}
           |[]|_x &= 0 \\
        |y : a|_x &= |a|_x + \begin{cases}
            1 \quad \text{If $x = y$} \\
            0 \quad \text{If $x \neq y$}.
        \end{cases}
    \end{align*}
\end{definition}

\begin{theorem}
    For all values $x$, $|[x]| = 1$.
\end{theorem}
\begin{proof}
    $|[x]| = 1 + |[]| = 1 + 0 = 1$.
\end{proof}

\begin{theorem} \label{list_size_conc}
    For all lists $a$ and $b$, $|a + b| = |a| + |b|$.
\end{theorem}
\begin{proof}
    The proof will be by induction on $a$.  When $a = []$,
    \[
        |[] + b| = |b| = 0 + |b| = |[]| + |b|,
    \]
    so the base case is true.  Now assume that $|a + b| = |a| + |b|$.  Then for
    all values $x$,
    \[
        |x : a + b| = 1 + |a + b| = 1 + |a| + |b| = |x : a| + |b|,
    \]
    so the theorem is true by induction.
\end{proof}

\begin{theorem} \label{list_size_comm}
    For all lists $a$ and $b$, $|a + b| = |b + a|$.
\end{theorem}
\begin{proof}
    This is just the combination of Theorem \ref{list_size_conc} and the
    commutativity of the addition of natural numbers.
\end{proof}

\begin{theorem} \label{list_image_size}
    For all lists $a$ and functions $f : \U \to \V$, we have $|f(a)| = |a|$.
\end{theorem}
\begin{proof}
    The proof will be by induction on $a$.  When $a = []$,
    \[
        |f([])| = |[]|,
    \]
    so the base case is true.  Now assume that $|f(a)| = |a|$.  Then for all
    values $x$,
    \[
        |f(x : a)| = |f(x) : f(a)| = 1 + |f(a)| = 1 + |a| = |x : a|,
    \]
    so the theorem is true by induction.
\end{proof}

\begin{theorem}
    For all values $x$, $|[x]|_x = 1$.
\end{theorem}
\begin{proof}
    \[
        |[x]|_x = 1 + |[]|_x = 1 + 0 = 1.
    \]
\end{proof}

\begin{theorem}
    For all values $x$ and $y$ with $x \neq y$, $|[x]|_y = 0$.
\end{theorem}
\begin{proof}
    \[
        |[x]|_y = 0 + |[]|_x = 0 + 0 = 0.
    \]
\end{proof}

\begin{theorem} \label{list_count_conc}
    For all lists $a$ and $b$ and values $x$, $|a + b|_x = |a|_x + |b|_x$.
\end{theorem}
\begin{proof}
    The proof will be by induction on $a$.  When $a = []$,
    \[
        |[] + b|_x = |b|_x = 0 + |b|_x = |[]|_x + |b|_x,
    \]
    so the base case is true.  Now assume that $|a + b|_x = |a|_x + |b|_x$.
    Then for all values $y$,
    \[
        |y : a + b|_x = |[y]|_x + |a + b|_x = |[y]|_x + |a|_x + |b|_x
        = |y : a|_x + |b|_x,
    \]
    so the theorem is true by induction.
\end{proof}

\begin{theorem} \label{list_count_comm}
    For all lists $a$ and $b$ and values $x$, $|a + b|_x = |b + a|_x$.
\end{theorem}
\begin{proof}
    This is just the combination of Theorem \ref{list_count_conc} and the
    commutativity of the addition of natural numbers.
\end{proof}

\begin{theorem} \label{list_count_reverse}
    For all lists $a$ and values $x$, $|a|_x = |a^{-1}|_x$.
\end{theorem}
\begin{proof}
    The proof will be by induction on $a$.  When $a = []$,
    \[
        |[]|_x = |[]^{-1}|_x,
    \]
    so the base case is true.  Now assume that $|a|_x = |a^{-1}|_x$.  Then for
    all values $y$,
    \[
        |y : a|_x = |[y]|_x + |a|_x = |a^{-1}|_x + |[y]|_x = |a^{-1} + [y]|_x
        = |(y : a)^{-1}|_x,
    \]
    so the theorem is true by induction.
\end{proof}

\begin{theorem} \label{list_count_in}
    For all lists $a$ and values $x$, $x \in a$ if and only if $0 \neq |a|_x$.
\end{theorem}
\begin{proof}
    For the forward direction, let $x \in a$.  Then by Theorem
    \ref{in_list_split}, there exist lists $b$ and $c$ such that $a = b + x :
    c$.  Then
    \[
        |a|_x = |b + x : c|_x = |b|_x + |[x]|_x + |c|_x = |b|_x + 1 + |c|_x.
    \]
    Zero is not equal to any natural number plus one, so $0 \neq |a|_x$.

    For the reverse direction, assume that $0 \neq |a|_x$.  The proof will be by
    induction on $a$.  When $a = []$, $0 \neq |[]|_x = 0$ is a contradiction, so
    the base case is vacuously true.  Now assume that if $0 \neq |a|_x$, then $x
    \in a$.  We must prove that if $0 \neq |y : a|_x$ for some value $y$, then
    $x \in y : l$.  If $x = y$, the result is trivial, so assume that $x \neq
    y$.  Then $|y : a|_x = |a|_x$, so we have $0 \neq |a|_x$, so $x \in a$ by
    the inductive hypothesis.  Thus, by induction, the theorem is true.
\end{proof}

\begin{theorem} \label{list_count_nin}
    For all lists $a$ and values $x$, $x \notin a$ if and only if $0 = |a|_x$.
\end{theorem}
\begin{proof}
    This is just the previous theorem stated in a slightly different way.
\end{proof}

\begin{theorem} \label{list_count_unique}
    For all lists $a$, if $a$ has unique elements, then for all values $x$,
    $|a|_x \leq 1$.
\end{theorem}
\begin{proof}
    The proof will be by induction on $a$.  When $a = []$, then $|[]|_x = 0 \leq
    1$, so the base case is true.  Now assume that $|a|_x \leq 1$.  Then we must
    prove that for all values $y$ such that $y : a$ has unique elements, then
    $|y : a|_x \leq 1$.  Because $y : a$ has unique elements, we know that $y
    \notin a$, so by Theorem \ref{list_count_nin} we have $0 = |a|_y$.  There
    are now two cases: when $x = y$, and when $x \neq y$.  When $x = y$, we have
    \[
        |y : a|_x = |y : a|_y = 1 + |a|_y = 1 + 0 \leq 1.
    \]
    When $x \neq y$, we have
    \[
        |y : a|_x = |a|_x,
    \]
    which is less than or equal to one by the inductive hypothesis.  Either way,
    $|y : a|_x \leq 1$, so the theorem is true by induction.
\end{proof}

\begin{theorem} \label{list_count_in_unique}
    For all lists $a$ and values $x$, if $a$ has unique elements and $x \in a$,
    then $|a|_x = 1$.
\end{theorem}
\begin{proof}
    This follows directly from antisymmetry and Theorems \ref{list_count_unique}
    and \ref{list_count_in}.
\end{proof}

\section{Folds}

\begin{definition}
    Let $*$ be a binary operation on $\U$, $e$ a value in $\U$, and $a$ a list.
    Then we define the right fold $*(a, e)$ recursively on $a$:
    \begin{align*}
           *([], e) &= e \\
        *(x : a, e) &= x * (*(a, e)).
    \end{align*}
\end{definition}

\begin{definition}
    Given a list $a$, we define the sum of the list $\sum a$ to be $+(a, 0)$.
\end{definition}

\begin{definition}
    Given a list $a$, we define the product of the list $\prod a$ to be
    $\cdot(a, 1)$, where $\cdot$ represents multiplication.
\end{definition}

\begin{theorem}
    For all elements $x$, $\sum [x] = x$.
\end{theorem}
\begin{proof}
    \[
        \sum [x] = x + \sum [] = x + 0 = x.
    \]
\end{proof}

\begin{theorem} \label{list_sum_conc}
    For all lists $a$ and $b$, $\sum (a + b) = \sum a + \sum b$.
\end{theorem}
\begin{proof}
    The proof will be by induction on $a$.  First, when $a = []$,
    \[
        \sum ([] + b) = \sum b = 0 + \sum b = \sum [] + \sum b,
    \]
    so the base case is true.  Now assume that $\sum (a + b) = \sum a + \sum b.$
    Then for all values $x$,
    \[
        \sum(x : a + b) = x + \sum(a + b) = x + \sum a + \sum b = \sum (x : a)
        \sum b,
    \]
    so the theorem is true by induction.
\end{proof}

\begin{theorem} \label{list_sum_neg}
    For all lists $a$, $\sum (-a) = -\sum a$.
\end{theorem}
\begin{proof}
    The proof will be by induction on $a$.  First, when $a = []$,
    \[
        \sum (-[]) = \sum [] = 0 = -0 = -\sum [],
    \]
    so the base case is true.  Now assume that $\sum (-a) = -\sum a$.  Then for
    all values $x$,
    \[
        \sum (-(x : a)) = \sum(-x : -a) = -x + \sum (-a) = -x - \sum a = -(x +
        \sum a) = -\sum (x : a),
    \]
    so the theorem is true by induction.
\end{proof}

\begin{theorem} \label{list_sum_minus}
    For all lists $a$ and $b$, $\sum (a + (-b)) = \sum a - \sum b$.
\end{theorem}
\begin{proof}
    This is just the combination of the previous two theorems.
\end{proof}

\begin{theorem}
    For all elements $x$, $\prod [x] = x$.
\end{theorem}
\begin{proof}
    \[
        \prod [x] = x \prod [] = x 1 = x.
    \]
\end{proof}

\begin{theorem} \label{list_prod_conc}
    For all lists $a$ and $b$, $\prod (a + b) = (\prod a) (\prod b)$.
\end{theorem}
\begin{proof}
    The proof is identical in form to the proof of Theorem \ref{list_sum_conc}.
\end{proof}

\section{Permutations}

\begin{definition}
    Let $a$ and $b$ be two lists.  Then we say that $a$ is a permutation of $b$,
    written $a \sim b$, if for all $x : \U$, $|a|_x = |b|_x$.
\end{definition}

\begin{instance}
    Permutations of lists is an equivalence relation.
\end{instance}
\begin{proof}
    \textit{Reflexivity.}  Given a list $a$, we must prove that for all $x$,
    $|a|_x = |a|_x$, which is trivial.

    \textit{Symmetry.}  Let $a$ and $b$ be lists such that $a \sim b$.  Then for
    all $x : \U$, we have $|a|_x = |b|_x$, so $|b|_x = |a|_x$, showing that $b
    \sim a$.

    \textit{Transitivity.} Let $a$, $b$, and $c$ be lists such that $a \sim b$
    and $b \sim c$.  Then for all $x : \U$, we have $|a|_x = |b|_x$ and $|b|_x =
    |c|_x$, so $|a|_x = |c|_x$, showing that $a \sim c$.
\end{proof}

\begin{theorem} \label{list_perm_skip}
    For all values $x$ and lists $a$ and $b$, if $a \sim b$, then $x : a \sim x
    : b$.
\end{theorem}
\begin{proof}
    Let $y$ be any value.  Then
    \begin{align*}
        |a|_y &= |b|_y \\
        |[x]|_y + |a|_y &= |[x]|_y + |b|_y \\
        |[x] + a|_y &= |[x] + b|_y \\
        |x : a|_y &= |x : b|_y.
    \end{align*}
\end{proof}

\begin{theorem} \label{list_perm_swap}
    For all values $x$ and $y$ and lists $a$, $x : y : a \sim y : x : a$.
\end{theorem}
\begin{proof}
    Let $z$ be any value.  Then
    \begin{align*}
        |a|_z &= |a|_z \\
        |[x]|_z + |[y]|_z + |a|_z &= |[y]|_z + |[x]|_z + |a|_z \\
        |[x] + [y] + a|_z &= |[y] + [x] + a|_z \\
        |x : y : a|_z &= |y : x : a|_z.
    \end{align*}
\end{proof}

\begin{theorem} \label{list_perm_nil_eq}
    For all lists $a$, if $[] \sim a$, then $[] = a$.
\end{theorem}
\begin{proof}
    Assume that $[] \neq a$.  Then there exists a value $x$ and list $a'$ such
    that $a = x : a'$, so we have $[] \sim x : a'$.  This implies that $|[]|_x =
    |x : a'|_x$, so $0 = S(|a'|_x)$, which is impossible.  Thus, $[] = a$.
\end{proof}

\begin{theorem} \label{list_perm_lpart}
    For all lists $a$, $b$, and $c$, if $a \sim b$, then $a + c \sim b + c$.
\end{theorem}
\begin{proof}
    Let $x$ be any value.  Then
    \begin{align*}
        |a|_x &= |b|_x \\
        |a|_x + |c|_x &= |b|_x + |c|_x \\
        |a + c|_x &= |b + c|_x.
    \end{align*}
\end{proof}

\begin{theorem} \label{list_perm_rpart}
    For all lists $a$, $b$, and $c$, if $b \sim c$, then $a + b \sim a + c$.
\end{theorem}
\begin{proof}
    Let $x$ be any value.  Then
    \begin{align*}
        |b|_x &= |c|_x \\
        |a|_x + |b|_x &= |a|_x + |c|_x \\
        |a + b|_x &= |a + c|_x.
    \end{align*}
\end{proof}

\begin{theorem} \label{list_perm_comm}
    For all lists $a$ and $b$, $a + b \sim b + a$.
\end{theorem}
\begin{proof}
    This follows directly from Theorem \ref{list_count_comm}.
\end{proof}

\begin{theorem} \label{list_perm_split}
    For all lists $a$ and $b$ and values $x$, $a + x : b \sim x : a + b$.
\end{theorem}
\begin{proof}
    By Theorem \ref{list_perm_comm} we have $a + x : b \sim x : b + a$.  By
    Theorem \ref{list_perm_comm} we also have $a + b \sim b + a$, so by Theorem
    \ref{list_perm_skip} we have $x : b + a \sim x : a + b$.  Thus, by
    transitivity, $a + x : b \sim x : a + b$.
\end{proof}

\begin{theorem} \label{list_split_perm}
    For all lists $a$ and values $x$, if $x \in a$, then there exists a list $b$
    such that $a \sim x : b$.
\end{theorem}
\begin{proof}
    By Theorem \ref{in_list_split}, there exists lists $b$ and $c$ such that $a
    = b + x : c$.  Then by Theorem \ref{list_perm_split}, $b + x : c \sim x : b
    + c$, showing that $a \sim x : b + c$.
\end{proof}

\begin{theorem} \label{list_perm_conc_lcancel}
    For all lists $a$, $b$, and $c$, if $a + b \sim a + c$, then $b \sim c$.
\end{theorem}
\begin{proof}
    Let $x$ be any value.  Then
    \begin{align*}
        |a + b|_x &= |a + c|_x \\
        |a|_x + |b|_x &= |a|_x + |c|_x \\
        |b|_x &= |c|_x.
    \end{align*}
\end{proof}

\begin{theorem} \label{list_perm_conc_rcancel}
    For all lists $a$, $b$, and $c$, if $b + a \sim c + a$, then $b \sim c$.
\end{theorem}
\begin{proof}
    Let $x$ be any value.  Then
    \begin{align*}
        |b + a|_x &= |c + a|_x \\
        |b|_x + |a|_x &= |c|_x + |a|_x \\
        |b|_x &= |c|_x.
    \end{align*}
\end{proof}

\begin{theorem} \label{list_perm_add_eq}
    For all values $x$ and lists $a$ and $b$, if $x : a \sim x : b$, then $a
    \sim b$.
\end{theorem}
\begin{proof}
    This follows from Theorem \ref{list_perm_conc_lcancel}.
\end{proof}

\begin{theorem} \label{list_in_unique_perm}
    For all lists $a$ and $b$, if they both have unique elements, and if $z \in
    a \leftrightarrow z \in b$ for all $x : \U$, then $a \sim b$.
\end{theorem}
\begin{proof}
    Let $x$ be any value.  There are two cases: when $x \in a$, and when $x
    \notin a$.  When $x \in a$, then $x \in b$ as well, so by Theorem
    \ref{list_count_in_unique}, $|a|_x = 1 = |b|_x$.  When $x \notin a$, then $x
    \notin b$ as well, so by Theorem \ref{list_count_nin}, $|a|_x = 0 = |b|_x$.
\end{proof}

\begin{theorem} \label{list_perm_in}
    For all lists $a$ and $b$, if $a \sim b$, then for all $x$, $x \in a
    \leftrightarrow x \in b$.
\end{theorem}
\begin{proof}
    By symmetry we only need to prove that if $x \in a$, then $x \in b$.
    Because $x \in a$, by Theorem \ref{list_count_in}, we have $|a|_x \neq 0$.
    Because $a \sim b$, $|b|_x \neq 0$ as well, so $x \in b$ by Theorem
    \ref{list_count_in}.
\end{proof}

\begin{theorem} \label{list_perm_split_eq}
    For all values $x$ and lists $a$ and $b$, if $x : a \sim b$, then there
    exists lists $c$ and $d$ such that $b = c + x : d$ and $a \sim c + d$.
\end{theorem}
\begin{proof}
    We have $x \in x : a$, so by Theorem \ref{list_perm_in} we have $x \in b$.
    By Theorem \ref{in_list_split} we have lists $c$ and $d$ such that $b = c +
    x : d$, so all that we need to prove is that $a \sim c + d$.  We know that
    $x : a \sim c + x : d$, and by Theorem \ref{list_perm_split}, $c + x : d
    \sim x : c + d$.  Thus, by transitivity, $x : a \sim x : c + d$, so by
    Theorem \ref{list_perm_add_eq} we have $a \sim c + d$ as required.
\end{proof}

\begin{theorem} \label{list_perm_single}
    For all values $x$ and lists $a$, if $[x] \sim a$, then $a = [x]$.
\end{theorem}
\begin{proof}
    By Theorem \ref{list_perm_split_eq} we have lists $c$ and $d$ such that $a =
    c + x : d$ and $[] \sim c + d$.  By Theorem \ref{list_perm_nil_eq}, we have
    $[] = c + d$.  This means that $b = []$ and $c = []$,  meaning that $a = []
    + x : [] = [x]$.
\end{proof}

\begin{theorem} \label{list_perm_reverse}
    For all lists $a$, $a \sim a^{-1}$.
\end{theorem}
\begin{proof}
    This follows directly from Theorem \ref{list_count_reverse}.
\end{proof}

\section{Unordered Lists}

Because permutation of lists is an equivalence relation, we can think of
equivalence classes of lists under permutations, which leads us to the following
definition.

\begin{definition}
    Let $\A$ be a type.  Then we define the type $\L_U(\A)$, the type of
    unordered lists of $\A$, to be the type  $\L(\A)/\usim$, where $\usim$ is
    the permutation relation.
\end{definition}

\begin{lemma}
    Adding elements to lists is well-defined under $\usim$.
\end{lemma}
\begin{proof}
    We must prove that if there are two lists $a$ and $b$ such that $a \sim b$,
    then for all values $x$, $x : a \sim x : b$.  This follows directly from
    Theorem \ref{list_perm_skip}.
\end{proof}

\begin{definition}
    Let $x$ be a value in $\U$.  Then we define adding an element to an
    unordered list to be the unary operation given by Theorem \ref{unary_op_ex}
    and the previous lemma, although we will think of it as a binary operation
    $\U \to \L_U(\U) \to \L_U(\U)$.  We will notate it in the same way as for
    lists, where adding an element $x$ to an unordered list $a$ is written $x :
    a$.
\end{definition}

We will write $\llbracket a, b, \cdots, n\rrbracket$ to mean $a : b : \cdots : n
: \llbracket \rrbracket$, where $\llbracket \rrbracket$ is the empty unordered
list, as defined under the canonical inclusion $\L(\U) \to \L_U(\U)$.

\begin{theorem} \label{ulist_end_neq}
    For all values $x$ and unordered lists $a$, $x : a \neq \llbracket
    \rrbracket$.
\end{theorem}
\begin{proof}
    If $x : a = \llbracket \rrbracket$, then by Theorem \ref{list_perm_nil_eq},
    we would have $x : a = []$, which is impossible.
\end{proof}

\begin{theorem} \label{ulist_single_eq}
    For all values $a$ and $b$, if $\llbracket a \rrbracket = \llbracket b
    \rrbracket$, then $a = b$.
\end{theorem}
\begin{proof}
    By Theorem \ref{list_perm_single}, we have $[a] = [b]$, so $a = b$.
\end{proof}

\begin{theorem} \label{ulist_swap}
    For all values $x$ and $y$ and unordered lists $a$, $x : y : a = y : x : a$.
\end{theorem}
\begin{proof}
    This is just a restatement of Theorem \ref{list_perm_swap}.
\end{proof}

\begin{theorem} \label{ulist_add_eq}
    For all values $x$ and unordered lists $a$ and $b$, if $x : a = x : b$, then
    $a = b$.
\end{theorem}
\begin{proof}
    This is just a restatement of Theorem \ref{list_perm_add_eq}.
\end{proof}

Unordered lists may seem harder to use than lists, but the following theorem
shows us that unordered lists act very similarly to normal lists:

\begin{theorem}[Unordered list induction]
    For all sets $S : \set{\L_U(\U)}$, if $\llbracket\rrbracket \in S$ and if
    for all lists $a$ with $a \in S$, we also have $x : a \in S$ for all values
    $x$, then $S = \bm U$.
\end{theorem}
\begin{proof}
    Let $a$ be any list.  We must prove that $[a] \in S$, where $[a]$ is the
    image of $a$ under the canonical inclusion map $\L(\U) \to \L_U(\U)$.  We
    will prove this by induction on $a$.  When $a$ is empty, then $[a] =
    \llbracket \rrbracket$, and $\llbracket \rrbracket \in S$ by assumption, so
    the base case is true.  Now assume that $[a] \in S$, and let $x$ be any
    value.  We must prove that $[x : a] \in S$.  By assumption, because $[a] \in
    S$, we know that $x : [a] \in S$.  $x : [a] = [x : a]$ by definition, so $[x
    : a] \in S$.  Thus, the theorem is true by induction.
\end{proof}

This theorem shows us that we can induct on unordered lists in exactly the same
way as on ordered lists.  However, we still need to define operations on
unordered lists before we can really do things with them.  Thus, we must prove
that any of the operations that we might want to do on unordered lists are
well-defined.

\begin{lemma}
    Concatenation of lists is well-defined under permutation of lists.
\end{lemma}
\begin{proof}
    We mult prove that if $a \sim b$ and $c \sim d$, then $a + c \sim b + d$.
    By Theorem \ref{list_perm_rpart} we have $a + c \sim a + d$, and by Theorem
    \ref{list_perm_lpart} we have $a + d \sim b + d$.  Thus, by transitivity, $a
    + c \sim b + d$.
\end{proof}

\begin{lemma}
    Applying a function to a list is well-defined under permutation of lists.
\end{lemma}
\begin{proof}
    We must prove that for all functions $f : \U \to \V$, that for all lists $a$
    and $b$, if $a \sim b$, then $f(a) \sim f(b)$.  Let $y$ be any value in
    $\V$.  We must prove that $|f(a)|_y = |f(b)|_y$.  We will prove this by
    induction on $a$.  When $a = []$, then because $a \sim b$, we have $b = []$,
    so the base case is trivial.  Now assume that for all lists $b$, if $a \sim
    b$, then $|f(a)|_y = |f(b)|_y$.  Let $x$ be any value such that $x : a \sim
    b$.  We must prove that $|f(x : a)|_y = |f(b)|_y$.  Because $x : a \sim b$,
    by Theorem \ref{list_perm_split_eq} we have lists $c$ and $d$ such that $b =
    c + x : d$ and $a \sim c + d$.  Then by the inductive hypothesis, we have
    $|f(a)|_y = |f(c + d)|_y$, so
    \begin{align*}
        |f(x : a)|_y
        &= |[f(x)]|_y + |f(a)|_y \\
        &= |[f(x)]|_y + |f(c + d)|_y \\
        &= |[f(x)]|_y + |f(c)|_y + |f(d)|_y \\
        &= |f(c)|_y + |[f(x)]|_y + |f(d)|_y \\
        &= |f(c + x : d)|_y \\
        &= |f(b)|_y,
    \end{align*}
    so the theorem is true by induction.
\end{proof}

\begin{lemma}
    Values being in lists is well-defined under permutation of lists.
\end{lemma}
\begin{proof}
    This is a direct consequence of Lemma \ref{list_perm_in}.
\end{proof}

\begin{lemma}
    Lists having unique elements is well-defined under permutation of lists.
\end{lemma}
\begin{proof}
    By symmetry, all we have to do is prove that if $a$ has unique elements and
    $a \sim b$, then $b$ has unique elements.  The proof will be by induction on
    $a$.  If $a = []$, then because $a \sim b$, $b = []$ as well, and $[]$ has
    unique elements, so the base case is true.  Now assume that for all lists
    $b$ such that $a \sim b$, $b$ has unique elements, and assume that $x : a
    \sim b$ for some value $x$ and that $x \notin a$.  We must prove that $b$
    has unique elements.  By Theorem \ref{list_perm_split_eq}, there exist lists
    $c$ and $d$ such that $b = c + x : d$ and $a \sim c + d$.  By the inductive
    hypothesis, $c + d$ has unique elements.  By Theorem \ref{list_unique_comm},
    we know that $d + c$ has unique elements.  Because $x \notin a$ and $a \sim
    c + d \sim d + c$, we know that $x \notin d + c$ by Theorem
    \ref{list_perm_in}.  Because $d + c$ has unique elements and $x \notin d +
    c$, we know that $x : d + c$ has unique elements.  By Theorem
    \ref{list_unique_comm}, this means that $c + x : d = b$ has unique elements,
    so the theorem is true by induction.
\end{proof}

\begin{lemma}
    Filtering lists is well-defined under permutation of lists.
\end{lemma}
\begin{proof}
    We must prove that for all lists $a$ and $b$ and sets $S$, if $a \sim b$,
    then $a^S \sim b^S$.  Let $x$ be any value in $\U$.  We must prove that
    $|a^S|_x = |b^S|_x$.  The proof will be by induction on $a$.  When $a = []$,
    then $b = []$, so $|[]^S|_x = |[]^S|_x$, meaning that the base case is true.
    Now assume that for any list $b$ such that $a \sim b$, we have $|a^S|_x =
    |b^S|_x$, and assume that there is some value $y$ such that $y : a \sim b$.
    By Theorem \ref{list_perm_split_eq} there exist lists $c$ and $d$ such that
    $b = c + y : d$ and that $a \sim c + d$.  By the inductive hypothesis, we
    have
    \[
        |a^S|_x = |(c + d)^S|_x = |c^S + d^S|_x = |c^S|_x + |d^S|_x.
    \]
    We must now prove that
    \[
        |(y : a)^S|_x = |(c + y : d)^S|_x = |c^S + (y : d)^S|_x
        = |c^S|_x + |(y : d)^S|_x.
    \]
    There are two cases: when $y \in S$, and when $y \notin S$.  When $y \in S$,
    we have
    \begin{align*}
        |(y : a)^S|_x
        &= |y : a^S|_x \\
        &= |[y]|_x + |a^S|_x \\
        &= |[y]|_x + |c^S|_x + |d^S|_x \\
        &= |c^S|_x + |[y]|_x + |d^S|_x \\
        &= |c^S|_x + |(y : d)^S|_x.
    \intertext{When $y \notin S$,}
        |(y : a)^S|_x
        &= |a^S|_x \\
        &= |c^S|_x + |d^S|_x \\
        &= |c^S|_x + |(y : d)^S|_x.
    \end{align*}
    Either way, the inductive case is true, so the theorem is true by induction.
\end{proof}

\begin{lemma}
    Every element of a list being in a set is well-defined under permutation of
    lists.
\end{lemma}
\begin{proof}
    We must prove that for all sets $S$ and lists $a$ and $b$ such that $a \sim
    b$, if $a \in S$, then $b \in S$.  The proof will be by induction on $a$.
    When $a = []$, $b = []$ as well, and $[] \in b$ by definition, so the base
    case is true.  Now assume that there is some element $x$ such that $(x : a)
    \in S$, $x : a \sim b$, and that for all lists $b$ such that $a \sim b$, $b
    \in S$.  Because $x : a \sim b$, by Theorem \ref{list_perm_split_eq}, there
    exist lists $c$ and $d$ such that $b = c + x : d$ and $a \sim c + d$.  Then
    by the inductive hypothesis $c + d \in S$.  By Theorem \ref{list_prop_conc},
    we have $c \in S$ and $d \in S$.  Because $x : a \in S$, we have $x \in S$.
    Thus, because $c \in S$, $x \in S$, and $d \in S$, we have $c + x : d \in
    S$, showing that the theorem is true by induction.
\end{proof}

\begin{lemma}
    The size of a list is well-defined under permutation of lists.
\end{lemma}
\begin{proof}
    We must prove that for all lists $a$ and $b$ with $a \sim b$, $|a| = |b|$.
    The proof will be by induction on $a$.  When $a = []$, $b = []$ as well, so
    $|[]| = |[]|$, showing that the base case is true.  Now assume that for all
    lists $b$ such that $a \sim b$, $|a| = |b|$, and that there exists an
    element $x$ such that $x : a \sim b$.  We must prove that $|x : a| = |b|$.
    By Theorem \ref{list_perm_split_eq}, there exist lists $c$ and $d$ such that
    $b = c + x : d$ and $a \sim c + d$.  Then by the inductive hypothesis, $|a|
    = |c + d|$.  Thus,
    \[
        |x : a| = 1 + |a| = 1 + |c| + |d| = |c| + 1 + |d| = |c + x : d|,
    \]
    showing that the theorem is true by induction.
\end{proof}

\begin{lemma}
    The count of elements in a list is well-defined under permutation of lists.
\end{lemma}
\begin{proof}
    We must prove that for all lists $a$ and $b$ with $a \sim b$, for all
    elements $x$, $|a|_x = |b|_x$.  This is precisely the definition of $a \sim
    b$.
\end{proof}

\begin{lemma}
    The sum of a list is well-defined under permutation of lists.
\end{lemma}
\begin{proof}
    We must prove that for all lists $a$ and $b$ with $a \sim b$, we have $\sum
    a = \sum b$.  The proof will be by induction on $a$.  When $a = []$, $b =
    []$ as well, and $\sum [] = \sum []$, so the base case is true.  Now assume
    that for all lists $b$ such that $a \sim b$, $\sum a = \sum b$, and let $x$
    be an element such that $x : a \sim b$.  Then by Theorem
    \ref{list_perm_split}, there exist lists $c$ and $d$ such that $b = c + x :
    d$ and $a \sim c + d$.  Then by the inductive hypothesis, $\sum a = \sum (c
    + d) = \sum c + \sum d$, so
    \[
        \sum (x : a) = x + \sum a = x + \sum c + \sum d = \sum c + x + \sum d
        = \sum (c + x : d),
    \]
    so the theorem is true by induction.
\end{proof}

\begin{lemma}
    The product of a list is well-defined under permutation of lists.
\end{lemma}
\begin{proof}
    The proof is identical in form to the previous proof.  Note that this
    requires multiplication to be commutative.
\end{proof}

Thus, almost all of the operations that we can do on lists, we can do on
unordered lists as well.  The few exceptions are the reverse of a list, checking
if the elements of a list satisfy a relation, and permutations themselves.  The
reverse of an unordered list is equal to itself so the reverse is useless.
Similarly, permutation is equality in unordered lists, so permutations of
unordered lists are also useless.  While there is no strict reason that checking
if all of the elements of a list satisfy a relation can't be defined for
unordered lists, the original definition is asymmetric, so I didn't bother with
it.

At this point, almost all theorems about the usual operations on lists are
trivially true for unordered lists as well.  I will now list all of the theorems
that are true for unordered lists, most of which are trivial.  I will only
provide proofs of statements that don't readily follow from the corresponding
theorem for lists.

\begin{theorem} \label{ulist_conc_add}
    For all values $x$ and unordered lists $a$ and $b$, $(x : a) + b = x : (a +
    b)$.
\end{theorem}

\begin{instance}
    Concatenation of unordered lists is commutative.
\end{instance}
\begin{proof}
    This is just a restatement of Theorem \ref{list_perm_comm}.
\end{proof}

\begin{instance}
    $\llbracket \rrbracket$ is an additive identity.
\end{instance}

\begin{theorem} \label{ulist_conc_single}
    For all values $x$ and unordered lists $a$, $\llbracket x \rrbracket + a = x
    : a$.
\end{theorem}

\begin{instance}
    Concatenation of unordered lists is associative.
\end{instance}

\begin{instance}
    Concatenation of unordered lists is cancellative.
\end{instance}

\begin{theorem}
    For all $x : \A$, $f(\llbracket x\rrbracket) = \llbracket f(x)\rrbracket$.
\end{theorem}

\begin{theorem}
    For all unordered lists $a$ and $b$, $f(a + b) = f(a) + f(b)$.
\end{theorem}

\begin{theorem}
    Let $\A$, $\B$, and $\C$ be types, and $f$ a function from $\A$ to $\B$ and
    $g$ a function from $\B$ to $\C$.  Then for all unordered lists $a :
    \L_U(\A)$,
    \[
        g(f(a)) = (g \circ f)(a).
    \]
\end{theorem}

\begin{theorem}
    For all elements $x$ and $y$, $x \in \llbracket y\rrbracket \leftrightarrow
    x = y$.
\end{theorem}

\begin{theorem} \label{in_ulist_conc}
    For all unordered lists $a$ and $b$ and for all values $x$, if $x \in (a +
    b)$, then either $x \in a$ or $x \in b$.
\end{theorem}

\begin{theorem} \label{in_ulist_lconc}
    For all unordered lists $a$ and $b$ and values $x$, if $x \in a$, then $x
    \in (a + b)$.
\end{theorem}

\begin{theorem} \label{in_ulist_rconc}
    For all unordered lists $a$ and $b$ and values $x$, if $x \in b$, then $x
    \in (a + b)$.
\end{theorem}

\begin{theorem} \label{in_ulist_split}
    For all unordered lists $a$ and values $x$, in $x \in a$, then there exists
    an unordered list $b$ such that $a = x : b$.
\end{theorem}
\begin{proof}
    This is just a restatement of Theorem \ref{list_split_perm}.
\end{proof}

\begin{theorem} \label{in_ulist_image}
    For all unordered lists $a$, values $x$, and functions $f : \U \to \V$, if
    $x \in a$, then $f(x) \in f(a)$.
\end{theorem}

\begin{theorem} \label{image_in_ulist}
    For all unordered lists $a$, values $y$, and functions $f : \U \to \V$, if
    $y \in f(a)$, then there exists an $x \in a$ such that $f(x) = y$.
\end{theorem}

\begin{theorem}
    For all values $a$, $\llbracket a\rrbracket$ has unique elements.
\end{theorem}

\begin{theorem} \label{ulist_in_unique_eq}
    For all unordered lists $a$ and $b$, if they both have unique elements and
    if for all values $x$, $x \in a \leftrightarrow x \in b$, then $a = b$.
\end{theorem}
\begin{proof}
    This is just a restatement of Theorem \ref{list_in_unique_perm}.
\end{proof}

\begin{theorem} \label{ulist_unique_lconc}
    For all unordered lists $a$ and $b$, if $a + b$ has unique elements, then
    $a$ has unique elements.
\end{theorem}

\begin{theorem} \label{ulist_unique_rconc}
    For all unordered lists $a$ and $b$, if $a + b$ has unique elements, then
    $b$ has unique elements.
\end{theorem}

\begin{theorem} \label{ulist_unique_conc}
    For all unordered lists $a$ and $b$, if $a + b$ has unique elements, then
    for every $x \in a$, we have $x \notin b$.
\end{theorem}

\begin{theorem} \label{ulist_image_unique}
    For all unordered lists $a$ and functions $f : \U \to \V$, if $f(a)$ has
    unique elements, then $a$ does as well.
\end{theorem}

\begin{theorem} \label{ulist_image_unique_inj}
    For all unordered lists $a$ and injective functions $f : \U \to \V$, if $a$
    has unique elements, then $f(a)$ does as well.
\end{theorem}

\begin{theorem} \label{ulist_sub_ex}
    For all unordered lists $a$ and $b$, if $a$ has unique elements, and if for
    all $x$, $x \in a$ implies $x \in b$, then there exists a list $c$ such that
    $b = a + c$.
\end{theorem}
\begin{proof}
    The proof will be by induction on $a$.  When $a = \llbracket \rrbracket$,
    then $c = b$ works, so the base case is true.  Now assume that there is a
    value $x$ such that $x \notin a$, that $a$ has unique elements, and that for
    all $b$ such that $x \in a$ implies $x \in b$, there exists a list $c$ such
    that $b = a + c$.  We also have that for all $y \in x : a$ we have $y \in
    b$.  We must prove that there exists a list $c$ such that $b = x : a + c$.
    First, because $x \in x : a$, we have $x \in b$, so there exists a list $b'$
    such that $b = x : b'$.  Then for all $y \in a$, we have $y \in x : a$, so
    $y \in x : b'$.  We can't have $y = x$ because $y \in a$ and $x \notin a$,
    so $y \in b'$.  Thus, for all $y \in a$, we have $y \in b'$, so by the
    inductive hypothesis we have a list $c$ such that $b' = a + c$.  Then $b = x
    : b' = b' : a + c$, showing that the theorem is true by induction.
\end{proof}

\begin{theorem}
    For all sets $S$ and values $x$, if $x \in S$, then $\llbracket
    x\rrbracket^S = \llbracket x\rrbracket$.
\end{theorem}

\begin{theorem}
    For all sets $S$ and values $x$, if $x \in S$, then $\llbracket
    x\rrbracket^S = \llbracket \rrbracket$.
\end{theorem}

\begin{theorem}
    For all sets $S$ and unordered lists $a$ and $b$, $(a + b)^S = a^S + b^S$.
\end{theorem}

\begin{lemma} \label{ulist_filter_in_both}
    For all sets $S$, unordered lists $a$, and values $x$, if $x \in a^S$, then
    $x \in a$ and $x \in S$.
\end{lemma}

\begin{theorem} \label{ulist_filter_in}
    For all sets $S$, unordered lists $a$, and values $x$, if $x \in a^S$, then
    $x \in a$.
\end{theorem}

\begin{theorem} \label{ulist_filter_in_set}
    For all sets $S$, unordered lists $a$, and values $x$, if $x \in a^S$, then
    $x \in S$.
\end{theorem}

\begin{theorem} \label{ulist_filter_unique}
    For all sets $S$ and unordered lists $a$, if $a$ has unique elements, then
    $a^S$ has unique elements.
\end{theorem}

\begin{theorem} \label{ulist_filter_image_in}
    For all sets $S : \set{\U}$, functions $f : \U \to \V$, unordered lists $a$,
    and values $y : \V$, if $y \in f(a^S)$, then $y \in f(a)$.
\end{theorem}

\begin{theorem} \label{ulist_filter_image_unique}
    For all sets $S : \set{\U}$, function $f : \U \to \V$, and unordered lists
    $a$, if $f(a)$ has unique elements, then $f(a^S)$ has unique elements.
\end{theorem}

\begin{theorem} \label{ulist_filter_inter}
    For all sets $S$ and $T$ and unordered lists $a$, $(a^S)^T = a^{S \cap T}$.
\end{theorem}

\begin{theorem} \label{ulist_filter_filter}
    For all sets $S$ and unordered lists $a$, $(a^S)^S = a^S$.
\end{theorem}

\begin{theorem}
    For all sets $S$ and values $x$, $\llbracket x\rrbracket \in S$ if and only
    if $x \in S$.
\end{theorem}

\begin{theorem} \label{ulist_prop_conc}
    For all sets $S$ and unordered lists $a$ and $b$, $(a + b) \in S$ if and
    only if $a \in S$ and $b \in S$.
\end{theorem}

\begin{theorem} \label{ulist_prop_sub}
    For all sets $S$ and $T$ with $S \subseteq T$ and all unordered lists $a$,
    if $a \in S$, then $a \in T$.
\end{theorem}

\begin{theorem} \label{ulist_prop_filter}
    For all unordered lists $a$ and sets $S$, $a^S \in S$.
\end{theorem}

\begin{theorem} \label{ulist_prop_in}
    For all unordered lists $a$ and sets $S$, if $a \in S$, then for all $x \in
    a$, we have $x \in S$.
\end{theorem}

\begin{theorem} \label{ulist_in_prop}
    For all unordered lists $a$ and sets $S$, if $x \in S$ whenever $x \in a$,
    then $a \in S$.
\end{theorem}

\begin{theorem} \label{ulist_prop_in_sub}
    For all unordered lists $a$ and $b$ and sets $S$, if for all $x \in a$ we
    have $x \in b$, and if $b \in S$, then $a \in S$.
\end{theorem}

\begin{theorem} \label{ulist_prop_other_filter}
    For all unordered lists $a$ and sets $S$ and $T$, if $a \in S$, then $a^T
    \in S$.
\end{theorem}

\begin{theorem} \label{ulist_prop_split}
    For all unordered lists $a$ and sets $S$, if, for all values $x$ and lists
    $b$ such that $a = x : b$, we have $x \in S$, then we have $a \in S$.
\end{theorem}
\begin{proof}
    By Theorem \ref{ulist_in_prop}, we need to prove that whenever $x \in a$, we
    have $x \in S$.  By Theorem \ref{in_ulist_split}, we have a list $b$ such
    that $a = x : b$.  Thus, by assumption, $x \in S$.
\end{proof}

\begin{theorem}
    For all values $x$, $|\llbracket x\rrbracket| = 1$.
\end{theorem}

\begin{theorem} \label{ulist_size_conc}
    For all unordered lists $a$ and $b$, $|a + b| = |a| + |b|$.
\end{theorem}

\begin{theorem} \label{ulist_image_size}
    For all unordered lists $a$ and functions $f : \U \to \V$, we have $|f(a)| =
    |a|$.
\end{theorem}

\begin{theorem}
    For all values $x$, $|\llbracket x\rrbracket |_x = 1$.
\end{theorem}

\begin{theorem}
    For all values $x$ and $y$ with $x \neq y$, $|\llbracket x\rrbracket|_y =
    0$.
\end{theorem}

\begin{theorem} \label{ulist_count_conc}
    For all unordered lists $a$ and $b$ and values $x$, $|a + b|_x = |a|_x +
    |b|_x$.
\end{theorem}

\begin{theorem} \label{ulist_count_in}
    For all unordered lists $a$ and values $x$, $x \in a$ if and only if $0 \neq
    |a|_x$.
\end{theorem}

\begin{theorem} \label{ulist_count_nin}
    For all unordered lists $a$ and values $x$, $x \notin a$ if and only if $0 =
    |a|_x$.
\end{theorem}

\begin{theorem} \label{ulist_count_unique}
    For all unordered lists $a$, if $a$ has unique elements, then for all values
    $x$, $|a|_x \leq 1$.
\end{theorem}

\begin{theorem} \label{ulist_count_in_unique}
    For all unordered lists $a$ and values $x$, if $a$ has unique elements and
    $x \in a$, then $|a|_x = 1$.
\end{theorem}

\begin{theorem}
    For all elements $x$, $\sum \llbracket x\rrbracket = x$.
\end{theorem}

\begin{theorem} \label{ulist_sum_conc}
    For all unordered lists $a$ and $b$, $\sum (a + b) = \sum a + \sum b$.
\end{theorem}

\begin{theorem} \label{ulist_sum_neg}
    For all unordered lists $a$, $\sum (-a) = -\sum a$.
\end{theorem}

\begin{theorem} \label{ulist_sum_minus}
    For all unordered lists $a$ and $b$, $\sum (a + (-b)) = \sum a - \sum b$.
\end{theorem}

\begin{theorem}
    For all unordered elements $x$, $\prod \llbracket x\rrbracket = x$.
\end{theorem}

\begin{theorem} \label{ulist_prod_conc}
    For all unordered lists $a$ and $b$, $\prod (a + b) = (\prod a) (\prod b)$.
\end{theorem}

\begin{theorem} \label{ulist_finite}
    For all types $\U$, if $\U$ is simply finite, then there exists an unordered
    list $a : \L_U(\U)$ such that $a$ has unique elements and that $x \in a$ for
    all $x : \U$.
\end{theorem}
\begin{proof}
    By Theorem \ref{simple_finite_bij}, there exists a natural number $n$ such
    that there is a bijective function $f : \U \to \T(n)$.  The proof will be by
    induction on $n$.  When $n = 0$, there can't be any values in $\U$, so the
    empty list works.  Thus, the base case is true.  Now assume that for all
    types $\V$ and bijective functions $g : \V \to \T(n)$, there exists an
    unordered list $a$ such that $a$ has unique elements and that $x \in a$ for
    all $x : \V$, and that we have a bijective function $f : \U \to \T(S(n))$.
    We must prove that there exists a list $a$ such that $a$ has unique elements
    and that $x \in a$ for all $x : \U$.

    Because $f$ is surjective, there exists some $x : \U$ such that $f(x) = n$.
    Define a new type $\V$ given by $\T(\{a : \U \mid a \neq x\})$.  Then we can
    define a new function $f' : \V \to \T(S(n))$ as the restriction of $f$ on
    $\V$.  Now for any $a \in \V$, if $f'(a) = f(a) = n$, we would have $a = x$
    by the injectivity of $f$.  Thus, $f'(a) \neq n$ and $f'(a) < S(n)$ for all
    $a : \V$, meaning that $f'(a) < n$ for all $a : \V$.  Thus we can define a
    function $g : \V \to \T(n)$ given by $g(a) = f(a)$.

    We will now prove that this function $g$ is bijective.  For injectivity, let
    $a$ and $b$ be values in $\V$ such that $g(a) = g(b)$.  This means that
    $f(a) = f(b)$, so $a = b$ by the injectivity of $f$, showing that $g$ is
    also injective.  For surjectivity, let $y$ be a natural number less than
    $n$.  Then $y < S(n)$ as well, so by the surjectivity of $f$, there exists
    an $a : \U$ such that $f(a) = y$.  Now if $a = x$, we would have $f(x) =
    f(a)$, which would mean that $n = y$, which contradicts $y < n$.  Thus, $a
    \neq x$.  This means that $a$ is in the set that defines $\V$.  Thus, $g(a)
    = f(a) = y$, showing that $g$ is surjective.

    We now have a type $\V$ with a bijective function $g : \V \to T(n)$, so by
    the inductive hypothesis there exists a list $a$ of values in $\V$ such that
    $a$ has unique elements and that $y \in a$ for all $y \in \V$.  We will show
    that the list $x : \stv{a}$ is the list that we want, that is, that $x
    :\stv{a}$ has unique elements and that $y \in x :\stv{a}$ for all $y : \U$.

    First, because $a$ has unique values, by Theorem
    \ref{ulist_image_unique_inj}, $\stv a$ has unique elements as well.  Also,
    if $x \in \stv a$, by Theorem \ref{image_in_ulist} we would have some value
    $x' : \V$ such that $\stv x' = x$ and $x' \in a$.  However, $\stp x'$ says
    that $\stv x' \neq x$, which is a contradiction.  Thus, $x \notin \stv a$.
    Because $\stv a$ has unique elements and $x \notin \stv a$, $x : \stv a$ has
    unique elements.

    To prove that $y \in x : \stv a$ for all $y : \U$, let $y$ be any value.  If
    $x = y$, then the result is trivial.  If $x \neq y$, we see that $y$ is a
    value in $\V$, which means that $y \in a$.  By Theorem \ref{in_ulist_image},
    we have $y \in [a|]$ as required.
\end{proof}

\end{document}
