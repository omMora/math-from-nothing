\documentclass[../math.tex]{subfiles}
\externaldocument{../math.tex}
\externaldocument{set}

\begin{document}

\setcounter{chapter}{5}

\chapter{Category Theory}

Category theory is the general study of mathematical structures.  It can provide
a unified way of describing many seemingly different aspects of math, and using
it can often simplify many definitions and theorems.  However, I don't actually
know much category theory, so I haven't developed much of it here.  I've mostly
just defined categories, functors, and natural transformations and a few basic
facts about them, and haven't even proven the Yoneda lemma.  I know that there
are many things later on that could be simplified by the use of category theory,
but the fact is I just haven't studied it enough yet.

Another thing that I should point out is that in the Coq code, I am unable to
make arguments using commutative diagrams.  Because this document is supposed to
mirror the Coq code, I will not be using commutative diagrams here either.  This
is not to say that I prefer not using commutative diagrams.  It's just that not
using them makes this document a better representation of the Coq code.

A quick note on the order here: In the Coq code, functors and natural
transformations are defined the moment after categories are.  That's because
it's helpful to have categories be objects in the category of categories, to
have functors be both the morphisms there and the objects in the category of
functors, and to have natural transformations be the morphisms in the category
of functors.  By defining these notions first, they can be used as the
definition of categories, functors, and natural transformations, allowing for
similar notation to be used for all of them.  However, here we don't have to
worry about making sure Coq will know what categories are objects are in, so
I'll present things in a different order here.  Instead, I will define
categories and several of the notions that are definable with just categories,
then functors and their notions after that, and then natural transformations.

\section{Categories}

\begin{definition}
    A Category is a combination of the following things:
    \begin{itemize}
        \item A $\Type$, which we call the objects of the category.  Given a
            particular category $C$, we abuse the notation $A : C$ to mean that
            $A$ is an object in $C$.
        \item A function from objects $A$ and $B$ to a $\Type$ that we call $A
            \to B$, the values of which are called the morphisms from $A$ to
            $B$.  Given a morphism $A \to B$, we call $A$ the domain and $B$ the
            codomain.
        \item A function that takes in objects $A$, $B$, and
            $C$ and morphisms $f : B \to C$ and $g : A \to B$, and produces a
            new morphism $f \circ g : A \to C$, called the composition of $f$
            and $g$.
        \item A function that takes in an object $A$ and produces a morphism
            $\1 : A \to A$ called the identity morphism.  If the object
            $A$ needs to be explicit we will write $\1_A$.
        \item A proof that composition of morphisms is associative, that is, for
            all objects $A$, $B$, $C$, and $D$ and all morphisms $f : C \to D$,
            $g : B \to C$, and $h : A \to B$, we have $f \circ (g \circ h) = (f
            \circ g) \circ h$.
        \item A proof that every identity morphism is a left identity, that is,
            for all objects $A$ and $B$ and morphisms $f : A \to B$, we have
            $\1_B \circ f = f$.
        \item A proof that every identity morphism is a right identity, that is,
            for all objects $A$ and $B$ and morphisms $f : A \to B$, we have
            $f \circ \1_A= f$.
    \end{itemize}
\end{definition}

\begin{definition}
    Given a category $C$ and two objects $A$ and $B$, if $f : A \to B$ and $g :
    B \to A$ are such that $f \circ g = \1$ and $g \circ f = \1$, then we
    say that $f$ and $g$ are inverses.  If $f$ is a morphism such that there
    exists a $g$ such that $f$ and $g$ are inverses, then we call $f$ an
    isomorphism.  If there exists an isomorphism from $A$ to $B$, then we say
    that $A$ and $B$ are isomorphic and we write $A \cong B$.
\end{definition}

\begin{theorem}
    The identity morphism is an isomorphism.
\end{theorem}
\begin{proof}
    The identity morphism is the inverse.
\end{proof}

\begin{theorem}
    The composition of two isomorphisms is an isomorphism.
\end{theorem}
\begin{proof}
    Let $A$, $B$, and $C$ be objects and let $f : B \to C$ and $g : A \to B$ be
    isomorphisms.  This means that we have morphisms $f' : C \to B$ and $g' : B
    \to A$ such that $f \circ f' = \1$, $f' \circ f = \1$, $g \circ g' = \1$,
    and $g' \circ g = \1$.  Then we have
    \[
        f \circ g \circ g' \circ f' = f \circ f' = \1
    \]
    and
    \[
        g' \circ f' \circ f \circ g = g' \circ g = \1,
    \]
    showing that $g' \circ f'$ is an inverse of $f \circ g$.
\end{proof}

\begin{theorem}
    If a morphism $f$ has two inverses $g_1$ and $g_2$, then $g_1 = g_2$.
\end{theorem}
\begin{proof}
    By definition, we have $f \circ g_1 = \1$.  Composing on the left with $g_2$
    on both sides we get
    \[
        g_2 \circ f \circ g_1 = g_1 = g_2.
    \]
\end{proof}

\begin{theorem}
    Objects being isomorphic is an equivalence relation.
\end{theorem}
\begin{proof}
    Reflexivity follows from the identity morphism being an isomorphism.  For
    symmetry, given an isomorphism $f : A \to B$, the inverse $g : B \to A$ is
    an isomorphism.  Transitivity follows from the composition of isomorphisms
    being an isomorphism.
\end{proof}

\begin{definition}
    Given a category $C$, we can define its dual category $\dual{C}$ as the
    category with objects the same as $C$, with morphisms from $A$ to $B$ as
    being the morphisms in $C$ from $B$ to $A$, with the composition $f \circ g$
    being the composition $g \circ f$ is $C$, and with identities being the same
    as the identities in $C$.
\end{definition}

\begin{theorem}
    A morphism is an isomorphism if and only if it is an isomorphism in the dual
    category.
\end{theorem}
\begin{proof}
    This follows directly from the symmetry of isomorphism.
\end{proof}

\begin{theorem}
    For all categories $C$, $C = \dual{{\dual{C}}}$.
\end{theorem}
\begin{proof}
    Trivial.
\end{proof}

\begin{definition}
    Given two categories $C_1$ and $C_2$, we define their product $C_1 \times
    C_2$ as the category where the objects are defined to be $C_1 \times C_2$
    (where $C_1$ and $C_2$ are considered to be the objects in this case), the
    morphisms $A \to B$ are defined to be $(P_1(A) \to P_1(B)) \times (P_2(A)
    \to P_2(B))$, the composition of two morphisms $f$ and $g$ is defined to be
    $(P_1(f) \circ P_1(g), P_2(f) \circ P_2(g))$, and the identity morphisms are
    defined to be $(\1, \1)$.
\end{definition}

\begin{definition}
    Given a category $C$, a subcategory of $C$ is subsets $O$ of the objects of
    $C$ and $M$ of the morphisms of $C$ such that the composition of morphisms
    in $M$ is in $M$ and that all identity morphisms are in $M$.  A subcategory
    is trivially a category formed with $\T(O)$ and $\T(M)$.
\end{definition}

\begin{definition}
    A subcategory is called a full subcategory if for all objects $A$ and $B$,
    $M(A, B) = \mathbf U$.
\end{definition}

\section{Functors}

\begin{definition}
    Given two categories $C_1$ and $C_2$, a functor $F$ from $C_1$ to $C_2$ is a
    combination of the following things:
    \begin{itemize}
        \item A function $C_1 \to C_2$, which will be denoted by treating the
            functor $F$ itself as a function.  For example, if $A : C_1$, then
            $F(A) : C_2$.
        \item For all objects $A$ and $B$ in $C_1$, a function from $A \to B$ to
            $F(A) \to F(B)$.  Again, this will be denoted by treating $F$ as a
            function, so if we have a morphism $f : A \to B$, then $F(f) : F(A)
            \to F(B)$.  Thus, functors will be used to represent two different
            functions that must be determined from context.  If we want to talk
            about the function taking morphisms from $A$ to $B$ to morphisms
            from $F(A)$ to $F(B)$ itself, we will use $F_{A,B}$.
        \item A proof that for all objects $A$, $B$, and $C$ in $C_1$ and
            morphisms $f : B \to C$ and $g : A \to B$, we have $F(f \circ g) =
            F(f) \circ F(g)$.
        \item A proof that for all objects $A : C_1$, we have $F(\1_A) =
            \1_{F(A)}$.
    \end{itemize}
\end{definition}

\begin{theorem}
    Given a category $C$, the identity map on both objects and morphisms is a
    functor that we call the identity functor.
\end{theorem}
\begin{proof}
    Trivial.
\end{proof}

\begin{theorem}
    Given three categories $C_1$, $C_2$, and $C_3$ and two functors $F : C_2 \to
    C_3$ and $G : C_1 \to C_2$, the functions $F(G(A))$ and $F(G(f))$ are a
    functor that we call the composition of $F$ and $G$.
\end{theorem}
\begin{proof}
    As for composition,
    \[
        F(G(f \circ g)) = F(G(f) \circ G(g)) = F(G(f)) \circ F(G(g)).
    \]
    As for the identity,
    \[
        F(G(\mathds 1)) = F(\mathds 1) = \mathds 1.
    \]
\end{proof}

\begin{theorem}
    Categories form a category \Cat where the objects are categories, the
    morphisms are functors between them, composition of functors is as defined
    above, and the identity functor is as defined above.
\end{theorem}
\begin{proof}
    Composition of functors is associative by the associativity of function
    composition.  The identity functor is an identity because it is defined to
    be the identity function.
\end{proof}

Thus, without confusion we may use the notation $\1$ to represent the
identity functor and $F \circ G$ to represent the composition of two functors.

\begin{theorem}
    Functors preserve isomorphisms, that is, if $F$ is a functor and $A \cong
    B$, then $F(A) \cong F(B)$.
\end{theorem}
\begin{proof}
    Because $A \cong B$, there exist morphisms $f : A \to B$ and $g : B \to A$
    such that $f \circ g = \1$ and $g \circ f = \1$.  Then we have
    \[
        F(f) \circ F(g) = F(f \circ g) = F(\1) = \1
    \]
    and
    \[
        F(g) \circ F(f) = F(g \circ f) = F(\1) = \1,
    \]
    showing that $F(f)$ and $F(g)$ are inverses.  Thus, $F(A) \cong F(B)$.
\end{proof}

\begin{definition}
    A functor $F$ is faithful if for all objects $A$ and $B$, $F_{A,B}$ is
    injective.
\end{definition}

\begin{definition}
    A functor $F$ is full if for all objects $A$ and $B$, $F_{A,B}$ is
    surjective.
\end{definition}

\begin{definition}
    A functor $F : C_1 \to C_2$ is essentially surjective if for all objects $B
    : C_2$, there exists an $A : C_1$ such that $F(A) \cong B$.
\end{definition}

\begin{theorem}
    The identity functor is faithful.
\end{theorem}
\begin{proof}
    This follows from the identity function being injective.
\end{proof}

\begin{theorem}
    The identity functor is full.
\end{theorem}
\begin{proof}
    This follows from the identity function being surjective.
\end{proof}

\begin{definition}
    Given a category $C$ and a subcategory $S$, the maps taking objects and
    morphisms from $S$ to $C$ is a functor called the inclusion functor.
\end{definition}

\begin{theorem}
    The inclusion functor is faithful.
\end{theorem}
\begin{proof}
    This follows from Theorem \ref{set-type-inj}.
\end{proof}

\begin{theorem}
    If $S$ is a full subcategory of $C$, then the inclusion functor from $S$ to
    $C$ is full.
\end{theorem}
\begin{proof}
    Let $f$ be a morphism in $S$ from objects $A$ to $B$.  We must prove that
    $f$ is a morphism in $C$ as well.  This follows from $S$ being a full
    subcategory of $C$.
\end{proof}

\section{Natural Transformations}

\begin{definition}
    Given two categories $C_1$ and $C_2$ and two functors $F$ and $G$ from $C_1$
    to $C_2$, a natural transformation $\alpha : F \to G$ is a combination of
    the following things:
    \begin{itemize}
        \item A function from objects $A : C_1$ to morphisms $F(A) \to G(A)$,
            which will be denoted $\alpha(A)$.  The image of this function is
            called the components of $\alpha$.
        \item A proof that for all objects $A$ and $B$ in $C_1$ and morphisms $f
            : A \to B$, we have $\alpha(B) \circ F(f) = G(f) \circ \alpha(A)$.
    \end{itemize}
\end{definition}

\begin{theorem}
    Given categories $C_1$ and $C_2$ and a functor $F : C_1 \to C_2$, the
    function that takes objects $A : C_1$ to $\1_{F(A)}$ is a natural
    transformation from $F$ to $F$ called the identity natural transformation.
\end{theorem}
\begin{proof}
    We mult prove that for all objects $A$ and $B$ in $C_1$ and morphisms $f : A
    \to B$, we have $\1 \circ F(f) = F(f) \circ \1$, which follows from $\1$
    being an identity.
\end{proof}

\begin{theorem}
    Given categories $C_1$ and $C_2$, functors $F$, $G$, and $H$ from $C_1$
    to $C_2$, and natural transformations $\alpha : G \to H$ and $\beta : F \to
    G$, the function taking objects $A : C_1$ to $\alpha(A) \circ \beta(A)$ is a
    natural transformation from $F$ to $H$ that we call the vertical composition
    of $\alpha$ and $\beta$.
\end{theorem}
\begin{proof}
    \[
        \alpha(B) \circ \beta(B) \circ F(f) =
        \alpha(B) \circ G(f) \circ \beta(A) =
        H(f) \circ \alpha(A) \circ \beta(A).
    \]
\end{proof}

\begin{theorem}
    Given two categories $C_1$ and $C_2$, functors form a category $\Fct(C_1,
    C_2)$ where the objects are functors from $C_1$ to $C_2$, the morphisms are
    natural transformations between them, composition of natural transformations
    is as defined above, and the identity natural transformation is as defined
    above.
\end{theorem}
\begin{proof}
    Every property needed follows from the corresponding property in $C_2$.
\end{proof}

Thus, like with functors, without confusion we may use the notation $\1$
to represent the identity natural transformation and $\alpha \circ \beta$ to
represent the vertical composition of two natural transformations.

\begin{theorem}
    Given three categories $C_1$, $C_2$, and $C_3$, four functors $F_1$ and
    $G_1$ from $C_1$ to $C_2$ and $F_2$ and $G_2$ from $C_2$ to $C_3$, and two
    natural transformations $\alpha : F_1 \to G_1$ and $\beta : F_2 \to G_2$,
    the function given by \[
        (\beta \ocircle \alpha)(A : C_1) = \beta(G_1(A)) \circ F_2(\alpha(A))
    \]
    is a natural transformation from $F_2 \circ F_1$ to $G_2 \circ G_1$ that we
    call the horizontal composition of $\beta$ and $\alpha$.
\end{theorem}
\begin{proof}
    For all objects $A$ and $B$ and morphisms $f : A \to B$,
    \begin{align*}
         {}& \beta(G_1(B)) \circ F_2(\alpha(B)) \circ F_2(F_1(f)) \\
        ={}& \beta(G_1(B)) \circ F_2(\alpha(B) \circ F_1(f)) \\
        ={}& G_2(\alpha(B) \circ F_1(f)) \circ \beta(F_1(A)) \\
        ={}& G_2(G_1(f) \circ \alpha(A)) \circ \beta(F_1(A)) \\
        ={}& G_2(G_1(f)) \circ G_2(\alpha(A)) \circ \beta(F_1(A)) \\
        ={}& G_2(G_1(f)) \circ \beta(G_1(A)) \circ F_2(\alpha(A)).
    \end{align*}
\end{proof}

\begin{theorem} \label{nat_trans_interchange}
    For all categories $C_1$, $C_2$, and $C_3$, functors $F_1$, $G_1$, and $H_1$
    from $C_1$ to $C_2$ and $F_2$, $G_2$, and $H_2$ from $C_2$ to $C_3$, and
    natural transformations $\alpha_1 : F_1 \to G_1$, $\beta_1 : G_1 \to H_1$,
    $\alpha_2 : F_2 \to G_2$, and $\beta_2 : G_2 \to H_2$,
    \[
        (\beta_2 \circ \alpha_2) \ocircle (\beta_1 \circ \alpha_1) =
        (\beta_2 \ocircle \beta_1) \circ (\alpha_2 \ocircle \alpha_1).
    \]
\end{theorem}
\begin{proof}
    Let $A : C_1$.  Then
    \begin{align*}
         {}& ((\beta_2 \circ \alpha_2) \ocircle (\beta_1 \circ \alpha_1))(A) \\
        ={}& (\beta_2 \circ \alpha_2)(H_1(A))
            \circ F_2((\beta_1 \circ \alpha_1)(A)) \\
        ={}& \beta_2(H_1(A)) \circ \alpha_2(H_1(A))
            \circ F_2(\beta_1(A) \circ \alpha_1(A)) \\
        ={}& \beta_2(H_1(A)) \circ \alpha_2(H_1(A))
            \circ F_2(\beta_1(A)) \circ F_2(\alpha_1(A)) \\
        ={}& \beta_2(H_1(A)) \circ G_2(\beta_1(A))
            \circ \alpha_2(G_1(A)) \circ F_2(\alpha_1(A)) \\
        ={}& (\beta_2 \ocircle \alpha_1)(A)
            \circ (\alpha_2 \ocircle \alpha_1)(A) \\
        ={}& ((\beta_2 \ocircle \alpha_1)
            \circ (\alpha_2 \ocircle \alpha_1))(A).
    \end{align*}
\end{proof}

\begin{theorem}
    For all categories $C_1$, $C_2$, and $C_3$ and functors $F : C_2 \to C_3$
    and $G : C_1 \to C_2$,
    \[
        \1_F \ocircle \1_G = \1_{F \circ G}.
    \]
\end{theorem}
\begin{proof}
    For all objects $A : C_1$,
    \[
        (\1_F \ocircle \1_G)(A)
        = \1(G(A)) \circ F(\1(A))
        = \1 \circ F(\1)
        = \1
        = \1(A).
    \]
\end{proof}

\begin{definition}
    If a natural transformation is an isomorphism in the category of functors,
    then we say that it is a natural isomorphism, and that those two functors
    are naturally isomorphic.
\end{definition}

\begin{theorem}
    A natural transformation $\alpha$ is an isomorphism if and only if the
    components of $\alpha$ are all isomorphisms.
\end{theorem}
\begin{proof}
    Let $C_1$ and $C_2$ be categories with functors $F$ and $G$ from $C_1$ to
    $C_2$, and let $\alpha : F \to G$.

    First assume that $\alpha$ is a natural isomorphism and let $A : C_1$.  We
    must find an inverse morphism to $\alpha(A)$.  Because $\alpha$ is a natural
    isomorphism, there exists a $\beta$ such that $\alpha \circ \beta = \1$ and
    $\beta \circ \alpha = \1$.  Now
    \[
        \alpha(A) \circ \beta(A) = (\alpha \circ \beta)(A) = \1(A) = \1,
    \]
    and similarly for $\beta(A) \circ \alpha(A)$, so $\beta(A)$ is the inverse
    of $\alpha(A)$, showing that $\alpha(A)$ is an isomorphism.

    Now assume that that all of the components of $\alpha$ are isomorphisms.
    Let $\beta : G \to F$ be given by $\beta(A) = \text{inverse of
    $\alpha(A)$}$.  Then for all objects $A$ and $B$ in $C_1$ and morphisms $f :
    A \to B$,
    \begin{align*}
         {}& \beta(B) \circ G(f) \\
        ={}& \beta(B) \circ G(f) \circ \alpha(A) \circ \beta(A) \\
        ={}& \beta(B) \circ \alpha(B) \circ F(f) \circ \beta(A) \\
        ={}& F(f) \circ \beta(A),
    \end{align*}
    showing that $\beta$ is a natural transformation.  Then by definition,
    $\alpha \circ \beta = \1$ and $\beta \circ \alpha = \1$, showing that
    $\alpha$ is a natural ismorphism.
\end{proof}

\begin{theorem}
    Given categories $C_1$, $C_2$, and $C_3$, and functors $F$ and $G$ from
    $C_2$ to $C_3$ and functors $H$ and $I$ from $C_1$ to $C_2$, if $F \cong G$
    and $H \cong I$, then $F \circ H \cong G \circ I$.
\end{theorem}
\begin{proof}
    Let $\alpha_1 : F \to G$ and $\alpha_2 : G \to F$ be inverses, and let
    $\beta_1 : H \to I$ and $\beta_2 : I \to H$ be inverses.  We will prove that
    $\alpha_1 \ocircle \beta_1$ and $\alpha_2 \ocircle \beta_2$ are inverses.
    By Theorem \ref{nat_trans_interchange},
    \[
        (\alpha_1 \ocircle \beta_1) \circ (\alpha_2 \ocircle \beta_2)
        = (\alpha_1 \circ \alpha_2) \ocircle (\beta_1 \circ \beta_2)
        = \1 \ocircle \1 = \1.
    \]
    Similarly,
    \[
        (\alpha_2 \ocircle \beta_2) \circ (\alpha_1 \ocircle \beta_1)
        = (\alpha_2 \circ \alpha_1) \ocircle (\beta_2 \circ \beta_1)
        = \1 \ocircle \1 = \1.
    \]
    Thus, $F \circ H \cong G \circ I$.
\end{proof}

\end{document}
